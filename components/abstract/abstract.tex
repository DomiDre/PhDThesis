\providecommand{\main}{../..}
\documentclass[\main/dresen_thesis.tex]{subfiles}
\begin{document}
  The collective magnetism of nanostructures is studied for four type of magnetic nanostructures: loosely packed nanospheres, monolayers of long-range ordered nanocubes in square arrays, double layers of said nanocubes in square arrays with non-magnetic spacer material of variable thickness in between, and at last three-dimensional colloidal crystals of nanocubes.
  Each nanostructure is studied by first discussing the non-interacting nanoparticle properties as is determined by small-angle X-ray and neutron scattering on dilute dispersions.
  Then the structure and magnetism of the arranged particles is resolved by grazing-incident scattering and reflectometry, where the results are compared with the nanoparticle properties to conclude on collective effects from interparticle interaction.
  \\

  For the discussion of the monolayer structure, a evaporation-driven self-assembly procedure to achieve long-range order for oleic acid ligated nanoparticles into a two-dimensional lattice is presented.
  It is shown that the method is capable to be extended to double layer in combination with spin-coated inter layers, which is used to perform a systematic study on interlayer dipolar interaction.
  Furthermore, for the detailed evaluation of the nanoparticle structural and magnetic properties, the superball form factor is introduced and used for the case of cubically shaped nanoparticles to account for the deviation from a perfect cube shape by rounded surfaces.
\end{document}