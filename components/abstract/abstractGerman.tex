\providecommand{\main}{../..}
\documentclass[\main/dresen_thesis.tex]{subfiles}
\begin{document}
  Der kollektive Magnetismus von Nanostrukturen wird für vier Typen von magnetischen Nanostrukturen untersucht: locker gepackte Nanokugeln, Monoschichten von langreichweitig geordneteten Nanowürfeln in quadratischen Anordnungen, Doppelschichten der Nanowürfel in quadratischen Anordnungen mit nichtmagnetischem Zwischenmaterial variabler Dicke, so wie dreidimensionale kolloidale Kristalle aus Nanowürfeln.
  Jede Nanostruktur wird untersucht, indem zuerst die nicht wechselwirkenden Nanoteilcheneigenschaften diskutiert werden, welche durch Kleinwinkelstreuung mittels Röntgen und Neutronenstrahlung in verdünnten Dispersionen bestimmt wird.
  Anschließend werden Struktur und Magnetismus der angeordneten Nanoteilchen durch Streuexperimente unter streifendem Einfall und Reflektometrie gelöst, wobei die Ergebnisse mit den Eigenschaften der Nanoteilchen verglichen werden, um auf kollektive Effekte der Interpartikelwechselwirkung schließen zu können.
  \\

  Für die Diskussion der Monoschichtstruktur wird ein auf Verdampfung basierendes Selbstorganisationsverfahren vorgestellt, um eine weitreichende Ordnung für Ölsäure-ligierte Nanoteilchen in einem zweidimensionalen Gitter zu erreichen.
  Es wird gezeigt, dass das Verfahren in der Lage ist, sich in Kombination mit spinbeschichteten Zwischenschichten auf eine Doppelschicht auszudehnen, woraufhin eine systematische Untersuchung der dipolaren Wechselwirkung zwischen den Schichten durchgeführt wird.
  Zur detaillierten Bewertung der strukturellen und magnetischen Eigenschaften der Nanoteilchen wird der Superball-Formfaktor eingeführt und für kubisch geformte Nanoteilchen verwendet, um die Abweichung einer perfekten Würfelform durch abgerundete Oberflächen zu berücksichtigen.
\end{document}