\providecommand{\main}{../../..}
\documentclass[\main/dresen_thesis.tex]{subfiles}
\begin{document}
	\section*{Curriculum Vitae}
	\addcontentsline{toc}{chapter}{Curriculum Vitae}
	\vspace{1cm}
	\Large\textbf{Dominique Hans Joseph Dresen} \\[5mm]
	\begin{table}[H]
		\begin{tabular}{p{3.5cm}l}
			Geburtsdatum/-ort & 15.03.1989 in D\"usseldorf\\
			Staatsangehörigkeit & Deutsch\\[5mm]
		\end{tabular}
	\end{table}

	% \underline{\parbox{0.465\textwidth}{\hspace*{45mm}\Large{Ausbildung}}}\hspace{2mm}
	\begin{table}[H]
		\begin{tabular}{p{5cm}l}
			02/2019						& Abgabe der Dissertation\\ \\

			seit 11/2014 			& \textbf{Dissertation} Physikalische Chemie\\
												& an der Universität zu Köln \\
												& ""\\ 
												& bei Dr. Sabrina Disch \\ \\

			10/2012 - 09/2014 & \textbf{Masterstudium} Physik \\
												& an der RWTH Aachen \\
												& Masterarbeit: "Quantum transport of non-interacting\\
												& electrons in 2D systems of arbitrary geometries" \\
												& am Institut f\"ur Quanteninformation\\
												& bei Prof. Dr. Fabian Hassler\\
												& Note: 1,0\\ \\

			10/2011 - 03/2012 & \textbf{Bachelorstudium} Chemie \\
			& an der RWTH Aachen  \\ \\

			10/2009 - 09/2012 & \textbf{Bachelorstudium} Physik \\
											  & an der RWTH Aachen  \\
												& Bachelorarbeit: "Suche nach resonanter \\
												& Smyon-Produktion mit dem CMS-Experiment"\\
												& am III. Physikalischen Institut A\\
												& bei Prof. Dr. Thomas Hebbeker\\
												& Note: 1,0\\ \\

			% 10/2008 - 07/2009 & \textbf{Grundwehrdienst} \\

			08/1999 - 08/2008 & \textbf{Abitur} Gymnasium Schwertstrasse Solingen \\
												& Note: 2,2\\
			\end{tabular}
	\end{table}
\end{document}