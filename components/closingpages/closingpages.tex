\providecommand{\main}{../..}
\documentclass[\main/dresen_thesis.tex]{subfiles}
  \renewcommand{\thisPath}{\main/components/closingpages/}
\begin{document}
  \subfile{\thisPath/danksagung/danksagung}
  \clearpage

  \subfile{\thisPath/erklaerung/erklaerung}
  \clearpage

  \subfile{\thisPath/personalBibliography/personalBibliography}
  \clearpage

  \subfile{\thisPath/cv/cv}
\end{document}
