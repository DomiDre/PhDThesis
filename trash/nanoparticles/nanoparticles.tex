\providecommand{\main}{../..}
\documentclass[\main/dresen_thesis.tex]{subfiles}

\begin{document}
\chapter{Nanoparticles}\label{ch:nanoparticles}
Using nanoparticles as building blocks to produce superstructures of higher order has multiple interesting aspects that can be leveraged in the process. 
Once the preparation of self-assembled nanostructures is well understood, it is easy to switch - within the limits of the production process - the particle batch for one with nanoparticles of different shape, size or material, thus altering the physical properties of the structure.
This enables the design of nanostructures with specially tailored properties. 

In this thesis, multiple nanostructures from different nanoparticles will be presented for the study of their emergent magnetic properties. 
To do this, the magnetism of the nanostructures is always compared to the magnetism of the individual nanoparticles in dispersion. 
In this chapter, we will present the general methodology that was developed and applied to characterize each nanoparticle dispersion on the example of cobalt ferrite nanocubes.
The results of the individual nanoparticles used for the various nanostructures are then briefly produced at the beginning of the respective chapters.

\section{Synthesis by Thermal Decomposition}

\section{Nuclear Structure}

\section{Magnetic Structure}

\end{document}