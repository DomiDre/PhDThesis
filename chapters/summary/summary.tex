\providecommand{\main}{../..}
\documentclass[\main/dresen_thesis.tex]{subfiles}

\begin{document}
\chapter{Summary}\label{ch:summary}
  In this thesis, a broad approach to the study of collective magnetism in nanostructures with varying degree of long-range order has been presented.

  In each case the individual nanoparticle structural and magnetic properties are determined from dispersion measurements by the means of small-angle scattering.
  The results from small-angle scattering are self-consistently checked to complimentary experiments such as diffraction, electron microscopy and vibrating sample magnetometry.
  It is then proceeded to study nanostructures where the discussed nanoparticles are used as building blocks to form nanostructures.
  Concretely studied are loosely packed structures of nanospheres, long-range ordered monolayers of cubes in a square array, double layer structures of such monolayers that are separated by a non-magnetic spacer layer and three-dimensional long-range ordered layers of iron oxide nanocubes.
  Where for the first and last structure, literature known methods have been used to prepare the samples, the method for the preparation of the long-range ordered monolayers and the double layers have been developed within the scope of this work.
  \\

  The structure of the assembled nanoparticles is determined in each case by the use of grazing-incidence scattering and reflectometry, as well as by scanning electron microscopy.
  With the knowledge of the nanoparticle properties and the relative position of the nanoparticles, the magnetic properties are tackled to study the magnetic structure and work out whether effects emerging from dipolar interparticle interaction can be determined.

  For the loosely packed nanostructures and colloidal crystals, the iron oxide nanoparticles are found in an inverse spinell phase close to magnetite but with anti-phase boundaries throughout the particle volume.
  This becomes apparent in diffraction as discrepancy of the reflections that are associated with the tetrahedral and octahedral sublattices of the inverse spinell phase and a reduced particle magnetization in magnetometry.
  From the study of the iron oxide nanocubes in the colloidal crystal chapter at varied times after the synthesis, the progressing oxidation from a w\"ustite/inverse spinell core-shell structure to a fully oxidized phase is visible within the first half year after synthesis.

  The magnetic properties of the self-assembled nanostructures from iron oxide nanoparticles are studied by macroscopic magnetization measurements.
  From the temperature-dependent magnetization measurements, a shift in the blocking temperature is visible for the loosely packed nanospheres towards higher temperatures in direct comparison to the non-interacting nanospheres.
  A comparison of the blocking temperature for the colloidal crystals with varied thickness reveals for the sample with intermediate thickness a shift towards lower temperatures, which is associated with a structure and thickness dependent dipolar coupling.
  Both type of nanostructures are studied by low-temperature polarized neutron reflectometry experiments after zero-field and field cooling.
  For the loosely packed nanospheres, no significant splitting is observed in remanence in either case.
  The colloidal crystals show a splitting at remanence that can be described qualitatively similar to the homogeneously magnetized state at saturation, only weaker in magnitude.
  It can be seen that the field cooled state shows a stronger splitting than the zero-field cooled state for every colloidal crystal, which is connected to the anti-phase boundaries in the non-interacting nanocubes and thereby identified as an individual nanoparticle property.
  \\

  Care is taken in the design of the monolayers to use nanocubes with a single phase and well-defined magnetic properties, which is shown in the nanoparticle characterization.
  Due to the simplified structure given by monolayers, it is straight forward to simulate the expectation for their magnetic properties numerically.
  Signatures to observe a super antiferromagnetic ground state emerging from dipolar interparticle coupling within the square arrays are thereby determined using the BornAgain software and explicitly searched for by polGISANS experiments.
  From the experiment, no clear signs of a macroscopic super antiferromagnetic state in the square arrays is visible.
  By comparing the long-range ordered monolayer with an disordered monolayer, a broadening of the hysteresis is observed instead, which suggests a super ferromagnetic coupling within the particle array.
  Supportive to this observation is that a jump in magnetization, observed in the hysteresis around zero applied magnetic field at low temperatures, decreases in magnitude for the long-range ordered layer in comparison to the disordered layer.

  The monolayer preparation procedure is then extended to prepare double layers of single nanoparticles that are separated by a non-magnetic layer of controlled thickness to study dipolar interlayer coupling effects.
  The feasibility of the preparation of such samples is shown by cross-sectional scanning electron microscopy and X-ray and neutron reflectometry experiments, where the three layers are clearly visible in each case.
  By direct comparison of the magnetic properties of the double layer samples, no direct effect from interlayer coupling that is correlated to the varied interlayer distance is observable and it is concluded that the single array magnetic properties dominate.
  \\

  The combined study of the non-interacting nanoparticle properties and the magnetic properties in a self-assembled nanostructure allowed in each of the studied structures to deduce which observations are associated with interparticle interaction and which are individual nanoparticle properties.
  As it is apparent that nanoparticle properties can greatly vary depending on how the nanoparticles are prepared, this method allows to give unbiased conclusions and at the same time provides the parameters to include in a simulation of the discussed nanostructure.

  \paragraphNewLine{Outlook}
    The development of the monolayer preparation method for oleic acid-ligated nanoparticles by drop casting from an \textit{n}-heptane dispersion with octadecene and oleic acid additives can be considered a starting point.
    In this thesis cobalt ferrite nanocubes were used and the drop casting process of the studied nanoparticles happened undirected.
    But as it is furthermore shown that the orientation of the resulting square arrays can be controlled by the means of the magnetic field, a continued study on oriented monolayers should be pursued.
    Furthermore, the effect of the magnetocrystalline anisotropy on the magnetic properties of the monolayers can be further studied by varying the material from cobalt ferrite to iron oxide and mixed compositions in between.

    To gain further insight into the magnetic structure and the ground state of the monolayers, methods that resolve the magnetization locally are of interest for future studies, such as magnetic force microscopy and electron holography.
    Due to the coarse resolution of these experiments, it might be necessary to extend the monolayer preparation to larger nanoparticles to perform said experiments.

    Going furthermore to the study of double layers, this thesis has only shown the very first attempts on such a structure.
    It should be of interest to study the effects of varying interlayer spacing materials, as well as possibly the combination of two types of different materials to the material efficient and low-cost creation of hybrids that combine properties of multiple materials in a single sandwich.
    The number of conceivable combinations of materials with different magnetic, conducting or optical properties are large and further interesting future results can be expected.
\end{document}