\providecommand{\main}{../../../..}
\documentclass[\main/dresen_thesis.tex]{subfiles}

\begin{document}
  \label{sec:monolayers:structuralCharacterization:summary}
  Using two monolayer samples of different nanoparticle batches, the monolayer and long-ranged lateral order is shown by reflectometry and GISAXS.
  The reflectivity of each sample is described by the same model of an homogeneous thin layer with the thickness of the nanocubes as determined by small-angle scattering.
  Additionally layers of oleic acid around that and a silicon dioxide on top of the silicon substrate are used to describe the sample, which is enough to achieve good agreement between the calculated and observed reflectivity.
  Using the fitted packing fraction of the nanocubes from the reflectivity data, the average particle spacing in the square lattice is estimated, which is and $14.2(2) \unit{nm}$ for ML-Ac-CoFe-C and $13.4(2) \unit{nm}$ for ML-Ac-CoFe-C-2 according to the X-ray reflectivity.
  \\

  The lateral order in ML-Ac-CoFe-C-2 observed by grazing-incidence small-angle X-ray scattering is described by a model of nanocubes with a square array paracrystal interference function using the BornAgain software package.
  The layered structure of the sample as observed by X-ray reflectometry is likewise entered to appropriately describe the sample.
  By fitting the first order peak observed in the Yoneda band with a Voigt function, all parameters of the paracrystal are determined to reproduce the two-dimensional GISAXS data with good agreement.
  An average particle spacing of $13.28 \unit{nm}$ on the square array and a coherence length of $877 \unit{nm}$ with an nearest-neighbour position uncertainty of $1.63 \unit{nm}$ is observed.
  The average particle spacing is thus in good agreement with the value determined by X-ray reflectometry, which also raises the confidence in the developed reflectometry model.
  \\

  The structural characterization of the samples by scattering methods is for one an proof that the monolayer structure is achieved not only on the small part of the sample that is shown by scanning electron microscopy, but over a macroscopic area of the substrate.
  And for the other, it is indispensable in the following for the discussion of the magnetic properties in the samples.

\end{document}