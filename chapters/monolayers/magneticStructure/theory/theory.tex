\providecommand{\main}{../../../..}
\documentclass[\main/dresen_thesis.tex]{subfiles}

\begin{document}
  For nanoparticles with long range order in two dimensions, in general five structural lattices are possible by the crystallographic restriction theorem: the square, hexagonal, rectangular, rhombic and oblique lattice.
  In this work, the square and hexagonal lattice are discussed, which can be obtained from densely packing nanocubes and nanospheres, respectively, in the plane.
  When only dipolar interaction between magnetic nanoparticles is considered on these lattices, different energetic ground states emerge for the two lattice types.
  On the hexagonal lattice the ground state is super ferromagnetic, whereas it is super antiferromagnetic for the square lattice as will be discussed in the following for both lattices separately.


  \subsubsection{Square Lattice}
    

  \subsubsection{Hexagonal Lattice}

\end{document}