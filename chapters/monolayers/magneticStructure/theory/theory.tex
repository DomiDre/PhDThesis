\providecommand{\main}{../../../..}
\documentclass[\main/dresen_thesis.tex]{subfiles}

\begin{document}
  To discuss the expected magnetic structure of ordered nanostructures, dynamic micromagnetic simulations are performed and used to estimate the relevant scales of the different contributions.
  The evolution of a nanoparticle superspin $\hat{\mu}_i$ in an assembly can be modeled by using the Landau-Lifshitz-Gilbert equation
  \begin{align}
    \frac{\dint \hat{\mu}_i}{\dint t} \eq - \frac{\gamma}{1 + \alpha^2} \biggl( \hat{\mu}_i \times \vec{B}_i^\mathsf{eff} + \alpha \hat{\mu}_i \times \bigl(\hat{\mu}_i \times \vec{B}_i^\mathsf{eff} \bigr) \biggr),
  \end{align}
  with $\gamma$ the electron gyromagnetic ratio and $\alpha$ the Gilbert damping constant.
  For simplicity, it is assumed that all superspins are of equal magnitude $\mu \eq M_s V$, and $\hat{\mu}$ is of unit length.
  The effective magnetic field $\vec{B}_i^\mathsf{eff}$ at the position of the magnetic moment is a sum of multiple contributions
  \begin{align}
    \vec{B}_i^\mathsf{eff} \eq \vec{B}^\mathsf{ext} + \vec{B}^\mathsf{ani}_i + \vec{B}^\mathsf{dip-dip}_i + \vec{B}^\mathsf{therm}_i,
  \end{align}
  given by the externally applied field $\vec{B}^\mathsf{ext}$, the magnetic anisotropy $\vec{B}^\mathsf{ani}_i$, the dipole-dipole interaction with surrounding superspins $\vec{B}^\mathsf{dip-dip}_i$ and a white noise field $\vec{B}^\mathsf{therm}_i$ to simulate thermal effects.
  The exchange interaction is relevant for the formation of superspins, but is negligible for the interaction between two superspins as it only has a range of a few angstrom due to it's quantum mechanic nature.
  The fields are generally determined from their respective energy terms via the thermodynamic relation
  \begin{align}
    \vec{B}_i \eq - \frac{1}{\mu} \frac{\dint E}{\dint \hat{\mu}_i}
  \end{align}

  For the anisotropy field, either uniaxial or cubic anisotropy can be implemented where the latter is the observed anisotropy of iron oxide/cobalt ferrite in literature and the first is commonly observed for nanoparticles.
  The anisotropy field are given by
  \begin{align}
    \vec{B}^\mathsf{ani}_{i, \,\mathsf{uniaxial}} &\eq \frac{2 K V}{\mu} \bigl( \hat{\mu}_i \cdot \hat{n} \bigr) \hat{n},\\
    \vec{B}^\mathsf{ani}_{i, \,\mathsf{cubic}} &\eq \frac{2 V}{\mu} \begin{pmatrix}
      K_1 (\hat{\mu}_{i, x}^2 - 1) \hat{\mu}_{i, x} - K_2 (\hat{\mu}_{i, y} \hat{\mu}_{i, z})^2 \hat{\mu}_{i, x} \\
      K_1 (\hat{\mu}_{i, y}^2 - 1) \hat{\mu}_{i, y} - K_2 (\hat{\mu}_{i, x} \hat{\mu}_{i, z})^2 \hat{\mu}_{i, y} \\
      K_1 (\hat{\mu}_{i, z}^2 - 1) \hat{\mu}_{i, z} - K_2 (\hat{\mu}_{i, x} \hat{\mu}_{i, y})^2 \hat{\mu}_{i, z}
    \end{pmatrix}.
  \end{align}

  The dipole-dipole field is evaluated for each particle by summing the dipole fields produced by the other superspins
  \begin{align}
    \vec{B}^\mathsf{dip-dip}_i \eq \frac{\mu \mu_0}{4 \pi}  \sum_{i \neq j} \frac{3 \hat{r}_{ij}  (\mu_j \cdot \hat{r}_{ij}) - \mu_j}{r_{ij}^3},
  \end{align}
  with $\hat{r}_{ij}$ the unit vector describing the direction from the position of $\mu_j$ to $\mu_i$ and $r_{ij}$ the magnitude of the distance.

  A statistical noise field $\vec{B}^\mathsf{therm}_i$ is added to simulate interaction of the superspin with microscopic degrees of freedom and account for thermal effects.
  It is assumed that $\vec{B}^\mathsf{therm}_i$ follows a Gaussian stochastic process and has the statistical properties
  \begin{align}
    \braket{B^\mathsf{therm}_i (t)} &\eq 0,\\
    \braket{B^\mathsf{therm}_i (t) B^\mathsf{therm}_j (t^\prime)} &\eq 2 D \delta_{ij} \delta(t - t^\prime),
  \end{align}
  saying that the thermal field strength averages to 0 and is uncorrelated in time and space.
  The factor $D$ is determined by studying the stationary solutions of the Fokker-Planck equation associated with the stochastic Landau-Lifshitz-Gilbert equation to \cite{Garcia_1998_Lange}
  \begin{align}
    D \eq \frac{\alpha}{1 + \alpha^2} \frac{k_B T}{\gamma \mu}.
  \end{align}

  The numerical implementation is then performed in \textsc{Fortran90} by initializing first a system of magnetic moments, tabulating their respective positions and distances to one another, and then evolving the system in discrete time steps by recursively applying Heun's method \cite{Sueli_2003_Anint} to the set of ordinary differential equations.
  Random gaussian numbers needed for the simulation of thermal noise are generated using the Box-Mueller transform \cite{Box_1957_Anoteo}.
  After each time step, the magnetic moment vector directions $\hat{\mu}$ are normalized to avoid numerical instabilities.

  For nanoparticles with long range order in two dimensions, in general five structural lattices are possible by the crystallographic restriction theorem: the square, hexagonal, rectangular, rhombic and oblique lattice.
  In this work, the square and hexagonal lattice are discussed, which can be obtained from densely packing nanocubes and nanospheres, respectively, in the plane.
  When only dipolar interaction between magnetic nanoparticles is considered on these lattices, different energetic ground states emerge for the two lattice types.
  On the hexagonal lattice the ground state is super ferromagnetic, whereas it is super antiferromagnetic for the square lattice as will be discussed in the following for both lattices separately.

  \subsubsection{Square Lattice}
    

  \subsubsection{Hexagonal Lattice}

\end{document}