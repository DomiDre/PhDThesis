\providecommand{\main}{../../..}
\documentclass[\main/dresen_thesis.tex]{subfiles}
  \renewcommand{\thisPath}{\main/chapters/monolayers/summary}

\begin{document}
  The preceding chapter provides a full analysis of the preparation and characterization of magnetic monolayers starting from the single nanoparticle properties in dispersion, over the study on how to prepare monolayers with long-range order, towards the search for magnetic properties emerging from interparticle interaction.

  Two heating up approaches from literature for the cobalt ferrite nanocube synthesis were followed: from cobalt/iron oleates and from cobalt/iron acetylacetonates, where it is found that the oleate based synthesis results in homogeneously shaped nanocubes with a core-shell structure and only weak magnetic properties.
  The acetylacetonate based synthesis provides on the other hand nanocubes with strong magnetic properties and a pure single phase, but a broader size distribution.
  The preparation of monolayers is studied using the nanocubes with the small size distribution, to determine the optimal solvent/co-solvent combination and drying conditions.
  The results are transferred to the strongly magnetic nanocubes to prepare monolayers, which are studied in detail to resolve the structure and magnetism.
  \\

  It is confirmed by X-ray and neutron reflectometry that the samples have a monolayer structure.
  By grazing-incidence small-angle X-ray scattering the square array structure that is observed in electron microscopy is resolved and succesfully reproduced using the BornAgain software.
  Using the knowledge of the structural properties of the sample, as well as the single-particle properties of the nanocubes, the magnetism in a long-range monolayer is studied by magnetometry and polarized neutron scattering.
  No clear signs of a super antiferromagnetic state are found, which would theoretically be expected as the ground state of a dipolar coupled square array.
  From the hysteresis it is by direct comparison with the dispersion properties and the disordered monolayer that the hysteresis is enhanced, which points to a super ferromagnetic coupling within the sample.
  Although no super antiferromagnetic state is visible, it is visible that the remanent state of the monolayers is not perfectly describable by a homogeneously magnetized sample.
  \\

  Further work on the monolayer magnetism should be performed in the future to fully understand the observed ground state.
  Helpful for this endeavor will be measurements that resolve the magnetic structure locally, such as magnetic force microscopy and electron holography.
  Also it should be helpful to reduce the complexity of the monolayers further in the studies by reducing the particle size distribution and shape inhomogeneity of the synthesized nanocubes, as well as by orienting the square array lattices.
  To understand the magnetic ground state, polarized neutron reflectometry measurements with polarization analysis should point toward what is happening upon removal of the externally applied magnetic field and are pursued in the future.
\end{document}