\providecommand{\main}{../../..}
\documentclass[\main/dresen_thesis.tex]{subfiles}
  \renewcommand{\thisPath}{\main/chapters/monolayers/preparationMonolayers}

\begin{document}
  Long-range ordered monolayers can be prepared via drop-casting a nanocube dispersion on a flat wafer.
  Drop-casting is a material efficient and cheap method to prepare thin films.
  However, experimental experience shows that the homogeneity of such samples suffers due to wetting effects during evaporation.
  The general approach to improve the quality achieved by drop casting is the usage of co-solvents.
  Therefore, a method to prepare monolayers of oleic acid coated nanoparticles of approx. $10 - 20 \unit{nm}$ size by drop-casting with co-solvents is discussed in the following.
  The basic idea is to use a mixture of a fast-evaporating solvent, such as n-heptane, as primary dispersion medium and a slow-evaporating solvent, such as 1-octadecene, as addend.
  After complete evaporation of the fast-evaporating component, a gentle heating step is performed to remove the remaining dispersion medium.
  The various parameters during drop-casting are discussed and optimal value estimation are presented to achieve high-quality monolayers with long range order.

  \subsection{Nanoparticle Concentration}
    \subfile{\thisPath/nanoparticleConcentration/nanoparticleConcentration}

  \subsection{Solvent Properties}
    \subfile{\thisPath/solventProperties/solventProperties}

  \subsection{Drying Conditions}
    \subfile{\thisPath/dryingConditions/dryingConditions}

  \subsection{Cleaning of the Layer}
    \subfile{\thisPath/cleaningWafer/cleaningWafer}

\end{document}