\providecommand{\main}{../../../..}
\documentclass[\main/dresen_thesis.tex]{subfiles}

\begin{document}
  The optimal concentration of the nanoparticles in dispersion for a monolayer on a given wafer surface area can be roughly estimated geometrically from the average particle to particle distance on the square lattice $a_{p-p}$.
  Assuming a perfect square lattice expanding over the wafer area $A_{\textsf{wafer}}$, the number of nanocubes is determined by
  \begin{align}
    N \eq \frac{A_{\textsf{wafer}}}{a_{p-p}^2}.
  \end{align}
  When the volume of a dispersion $V_{\textsf{disp.}}$ is dropped on the wafer, the number of particles in the drop is given by
  \begin{align}
    N \eq \frac{c_V V_{\textsf{disp}}}{V_p},
  \end{align}
  where $c_V$ is the particle volume concentration of the dispersion and $V_p$ the volume of a single nanocube.
  The average nanoparticle volume is known from the small-angle scattering model, and the particle volume concentration can be estimated by
  \begin{align}
    c_V \eq \frac{c_m}{\rho},
  \end{align}
  using the the particle mass concentration $c_m$ of the dispersion that is estimated from gravimetry, or more exact from a thermogravimetric analysis, and the density $\rho$ of the particle that can be estimated using the literature value.
  Combining these equations, the particle mass concentration of the dispersion needs to be tuned to the value
  \begin{align}\label{eq:monolayers:preparation:particleConcentration}
    c_m \eq \rho \frac{A_{\textsf{wafer}}}{V_{\textsf{disp}}} \frac{V_p}{a_{p-p}^2}
  \end{align}
  Typically, for ferrite nanocubes with the size in the order of $10 \unit{nm}$ with an oleic acid shell in the order of $2 \unit{nm}$, the optimal mass concentration for a drop of $50 \unit{\text{\textmu} L}$ on a $10\times 10 \unit{mm^2}$ wafer is in the order of $0.1 \unit{mg/mL}$.
  Using this calculated estimate, the optimal particle concentration can be determined for a batch of nanoparticles by a short experimental drop casting series around this value.
  In the case of a non-square lattices, the \refeq{eq:monolayers:preparation:particleConcentration} has to account for the reduced coverage by including an additional factor for the packing density $\eta$ in the plane and $a_{p-p}^2$ has to be replaced by the average maximum cross-section the particle is taking in the lattice.
  For a perfect circle packing, which is expected for densely packed spheres in two dimensions, the packing density is straight forward to evaluate to $\eta \eq \pi \sqrt{3} / 6 \approx 0.9069$.
\end{document}