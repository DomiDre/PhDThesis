\providecommand{\main}{../../../..}
\documentclass[\main/dresen_thesis.tex]{subfiles}

\begin{document}
  After that treatment, the thin-film interference is no longer visible and only a light brownish color remains on the silicon wafer.
  From SEM it is revealed that there is still a slight organic residue of a few nm on the surface that charges during microscopy.
  This thin organic layer is attributable for example to the surplus of oleic acid in the dispersion.
  To remove it, the sample can be further annealed at $300 ^\circ C$, near the boiling temperature of oleic acid.
  However, when higher temperatures are used to remove this remaining organic layer, the nanostructure suffers due to the violent conditions.
  Washing the wafer with a polar solvent, such as ethyl acetate, on a spin coater on the other hand has proven to remove the organic layer gently and leaves the ordered layer unaffected.
  The idea is that the nanoparticles are non-soluble in polar solvents and therefore the nanostructure is mostly unaffected and sticks to the wafer, while the loose organic solvents on top can be thrown off the surface together with the polar solvent.
  As ethyl acetate also leaves some organic remains, the sample is washed again afterwards with 2-propanol to obtain a clean nanoparticle surface.
  This process increases the obtainable resolution within scanning electron microscopy, and makes it possible to get clear images of the cross section.
  Also during scattering experiments, it is best if any organic component is removed from the surface, to reduce the complexity of the samples for once and to reduce the amount of incoherent scattering during neutron experiments for the other.
\end{document}