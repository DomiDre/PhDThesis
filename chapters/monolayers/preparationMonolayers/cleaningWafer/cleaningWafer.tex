\providecommand{\main}{../../../..}
\documentclass[\main/dresen_thesis.tex]{subfiles}

\begin{document}
  \label{sec:monolayers:preparation:cleaningWafers}
  Even when the thin-film interference is no longer visible after the heat treatment, some light brownish color remains often on the silicon wafer.
  From SEM it is apparent that there is still a slight organic residue of a few nm on the surface, which becomes visible by imaging effects due to the electron beam charging the sample at the points where it scans the sample.
  This thin organic layer is attributable to the surplus of oleic acid in the dispersion.
  To remove it, the sample can be further annealed around $300 ^\circ C$, near the boiling temperature of oleic acid.
  However, when higher temperatures are used to remove this remaining organic layer, the nanostructure suffers due to the violent conditions.

  An alternative is to wash the wafer with a polar solvent, such as ethyl acetate.
  The polar solvents leave the long-range order unaffected, as the nanoparticles are only disperseable in non-polar solvents and the individual nanoparticles are stabilized by the nanostructure to remain on the substrate.
  The additional organic solvent, even though it is also not disperseable in polar solvents, can be removed this way by removing the polar solvent with the use of a spin coater.
  The idea is that the loose organic solvents are detached by the polar solvent and carried off due to centrifugal force (note here that the density of ethyl acetate is slightly below the density of oleic acid).
  As ethyl acetate might also leave some organic remains, the sample is further washed again afterwards with 2-propanol (HPLC) to increase the cleanness of the nanoparticle surface.
  This washing process increases the obtainable resolution within scanning electron microscopy, and makes it possible to even obtain clear images of the cross section of a monolayer.
  Also during scattering experiments, it is best if any organic component is removed from the surface, to reduce the complexity of the samples for one and to reduce the amount of incoherent scattering during neutron experiments due to hydrogen in the organic solvents for the other.
  By following this process, it is possible to obtain neutron scattering data from the monolayers  with a high signal/noise ratio in the first order peak, which will be used in the discussion of the magnetic structure.
\end{document}