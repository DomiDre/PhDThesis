\providecommand{\main}{../../../..}
\documentclass[\main/dresen_thesis.tex]{subfiles}

\begin{document}
  After the wafer has been treated at elevated temperature, it is still apparent from SEM that a layer of organic residue remains on the surface.
  This thin organic layer is attributable mostly to the surplus of oleic acid (and possibly 1-octadecene) in the dispersion and it can be removed by further annealing the sample at $300 ^\circ C$, near the boiling temperature of oleic acid.
  However, removing it by increased temperature is a aggressive method that destroys the square lattice structure underneath the organic layer.
  A more gentle method has been found by removing the organic residue by washing the wafer with polar solvents on a spin coater.
  The idea is that the nanoparticles are non-soluble in polar solvents and therefore the nanostructure is mostly unaffected and sticks to the wafer, while the loose organic solvents on top can be thrown off the surface together with the polar solvent.

  A series of washing the wafer with 

\end{document}