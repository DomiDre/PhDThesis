\providecommand{\main}{../../../..}
\documentclass[\main/dresen_thesis.tex]{subfiles}

\begin{document}
  As the dispersion is composed of a fast and slow evaporating component, the drop casting process subdivides into two parts.
  The first, where the primary fast evaporating phase of the dispersion evaporates and the second where the slowly evaporating phase is removed.
  For the first part, different conditions can be tested: it can be performed on a heating or cooling stage to change the surface temperature of the wafer, a magnetic field can be applied or the drying time can be enlarged by performing the drying procedure within an enclosed container.
  However, varying these conditions for example by the means of Peltier elements and Halbach arrays has shown no improvements of the sample quality.
  Therefore for all presented samples the first evaporation step is always performed at ambient conditions, in an open container, on an even surface, within a calm room as this has shown to provide the best results in all initial tests.

  \begin{figure}[tb]
    \centering
    \includegraphics[width=0.7\textwidth]{monolayers_preparation_oilyFilm}
    \caption{\label{fig:monolayers:preparation:dryingConditions:oilyFilm}Thin oily film visible on a silicon wafer after the fast evaporating component of the dispersion has been dried during monolayer preparation.}
  \end{figure}
  After the fast evaporating component of the dispersion is dry, a thin oily film is visible on the wafer by thin-film interference as shown in \reffig{fig:monolayers:preparation:dryingConditions:oilyFilm}.
  Even after waiting several weeks, this thin-film of 1-octadecene/oleic acid does not evaporate at room temperature and has to be removed actively at elevated temperatures.
  While temperatures of at least $140 \unit{^\circ C}$ are necessary to remove the organic components to most parts, quickly heating to that temperature (via a heating plate for example) leads to a inhomogeneous evaporation of the droplet.
  To have a homogeneous evaporation, the best found method is to place the silicon wafer within a glass Petri dish that is covered with perforated aluminium foil.
  Then, the best result is observed when the thin film is left within an oven at $80 \unit{^\circ C}$ for at least $12\unit{h}$ and only then heated to $140 \unit{^\circ C}$ for at least $6\unit{h}$.
  It is important to balance the oven with a water level, as any slope leads to a accumulation at the lower side of the silicon wafer.

  After that treatment, the thin-film interference is no longer visible and only a light brownish color remains on the silicon wafer.
  From SEM it is revealed that there is still a slight organic residue of a few nm on the surface that charges during microscopy.
  When higher temperatures are used to remove this remaining organic layer, the nanostructure suffers due to the violent conditions.
  Washing the wafer with a polar solvent, such as ethyl acetate, on a spin coater has proven to remove the organic layer and leave the ordered layer unaffected.
  As ethyl acetate also leaves some organic remains, the sample is washed again with 2-propanol (HLPC) to obtain a clean nanoparticle surface.
\end{document}