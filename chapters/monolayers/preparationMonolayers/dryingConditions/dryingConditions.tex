\providecommand{\main}{../../../..}
\documentclass[\main/dresen_thesis.tex]{subfiles}

\begin{document}
  As the dispersion is composed of a fast and slow evaporating component, the drop casting process subdivides into two parts.
  The first, where the primary fast evaporating phase of the dispersion evaporates and the second where the slowly evaporating phase is removed.
  For the first part, different conditions can be tested: it can be performed on a heating or cooling stage to change the surface temperature of the wafer, a magnetic field can be applied or the drying time can be enlarged by performing the drying procedure within an enclosed container.
  Quick and dirty tests of the various conditions however have shown no qualitative improvements towards the goal of increasing the sample quality. 
  And simply drying the droplet at ambient conditions on an even surface within a calm room has shown to provide the best results in all initial tests.

  After the fast evaporating component has been dried, a thin oily film is visible on the wafer by thin-film interference.
  Even after waiting several weeks, this thin-film of 1-octadecene/oleic acid does not evaporate and has to be removed at elevated temperatures.
  Varying the temperature, the best result is obtained when the thin film is left at $80 \unit{^\circ C}$ for at least $12\unit{h}$ (over night).
  After that treatment, the thin-film interference is no longer visible.
  However, within scanning electron microscopy, it becomes apparent that there is still a organic residue


  The droplet of dispersion and solvents can be dried under varied conditions.
  The heating temperature...
\end{document}