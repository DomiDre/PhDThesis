\providecommand{\main}{../../../..}
\documentclass[\main/dresen_thesis.tex]{subfiles}

\begin{document}
  By varying the solvent/co-solvent combination and drying procedure, a method to prepare long-range ordered monolayers of oleic acid ligated nanoparticles is presented.
  An experimental series on Ol-CoFe-C with alkane/alkene combinations as solvent/co-solvent, reveals that a combination of \textit{n}-heptane together with an additive of $2 \%$ octadecene in dispersion results in the highest long-range order.
  The protocol for the preparation of the monolayers on a substrate is hereby given by first evaporating the primary solvent \textit{n}-heptane at ambient conditions, and then removing most parts of the 1-octadecene in a balanced oven at $80 ^\circ$.
  For further reduction of the organic remnants, the sample should be heated to $140 ^\circ$ and can subsequently be rinsed with ethyl acetate on a spin coater.
  \\

  The developed preparation method can be transferred to spherical nanoparticles, which results in hexagonal order, and to nanocubes of different materials such as magnetite.
  Furthermore, it is observed that the self-assembly process is sensitive to the amount of oleic acid in the dispersion.
  From a series of Ac-CoFe-C-3 nanocube dispersions with varied oleic acid content, a optimal fraction in the order of $5 - 10 \cdot 10^{-3}$ is observed.
  This corresponds to a thin layer of $25 - 50 \unit{nm}$ on the substrate.
  Finally, it is shown that it is possible to direct the orientation of the nanocubes in the monolayer by applying a magnetic field during the drying procedure.
  \\

  An approach to explain this specific combination can be given by the freezing point of octadecene and the evaporative cooling effect of \textit{n}-heptane.
  Octadecene and oleic acid freeze at a temperature of approx. $15 \unit{^\circ C}$, whereas evaporating \textit{n}-heptane and lower order alkanes can cool a surface by over $\Delta T \eq 10 \unit{^\circ C}$ due to evaporative cooling \cite{Tuckermann_2002_Evapo}.
  During evaporation of the quickly evaporating \textit{n}-heptane at room temperature ($T \eq 22 \unit{^\circ C}$ measured for the laboratory surfaces), the surface of the droplet is evaporatively cooled near the freezing temperature of 1-octadecene, which increases the viscosity of the liquid at the surface.
  This generates the condition needed for monolayer formation as described in \cite{Bigioni_2006_Kinet}, where particles are pinned to the surface of the evaporating droplet at which they can find one another to form a single ordered layer.
  Increasing the order of the alkane reduces the evaporative cooling effect from the drop casting experiment, but also increases the time the particles have to order themselves on the droplet surface.
  On the other hand, exchanging octadecene for a lower order alkene reduces the freezing point, as to which no noticeable long-range order is observed for such combinations.
  \\

  In summary, the presented data show that a combination of \textit{n}-heptane, 1-octadecene and oleic acid, with fine-tuned ratios, lead to long-range order for oleic acid capped nanoparticles into two dimensional structures, where the specific lattice order depends on the particle shape.
\end{document}