\providecommand{\main}{../../../..}
\documentclass[\main/dresen_thesis.tex]{subfiles}

\begin{document}
  By varying the solvent/co-solvent combination and drying procedure, a method to prepare long-range ordered monolayers of oleic acid ligated nanoparticles is shown.
  An experimental series on Ol-CoFe-C with alkane/alkene combinations as solvent/co-solvent, show by SEM and GISAXS that a combination of \textit{n}-heptane together with an addend of $2 \%$ octadecene in dispersion results in the highest long-range order.
  The protocol for evaporation of the dispersion on a substrate is hereby given to evaporate the fast evaporating \textit{n}-heptane at ambient conditions, and then removing the slow-evaporating component in a balanced oven at $80 ^\circ$.
  For further reduction of the organic remnants, the sample should be heated to $140 ^\circ$ and can subsequently washed with non-polar solvents on a spin coater.
  \\

  The developed preparation method can be transferred to spherical nanoparticles, which results in hexagonal order and to nanocubes from another phase such as iron oxide.
  Variation of nanoparticles, show that dispersions that do not order by the presented method, can be ordered by adding a small fraction of oleic acid in the order of $10^{-4}$, which is discussed on the example of the nanocubes in the Ac-CoFe-C-3 batch.
  This coincides with the volume fraction of oleic acid that was found by the SANS characterization of the dispersions earlier during the nanoparticle characterization.
  As additional variation, it is shown that it's possible to direct the orientation of the nanocubes in the monolayer by applying a magnetic field during the drying procedure.
  \\

  An approach to explain this specific combination can be given by the freezing point of octadecene and the evaporative cooling effect of \textit{n}-heptane.
  Octadecene and oleic acid freezes at a temperature of approx. $15 \unit{^\circ C}$, whereas evaporating \textit{n}-heptane and lower order alkanes can cool a surface by over $\Delta T \eq 10 \unit{^\circ C}$ due to evaporative cooling \cite{Tuckermann_2002_Evapo}.
  During evaporation of the quickly evaporating \textit{n}-heptane at room temperature ($T \eq 22 \unit{^\circ C}$ measured for the laboratory surfaces), the surface of the droplet is evaporative cooled near the freezing temperature of 1-octadecene, which increases the viscosity of the liquid at the surface.
  This generates the condition needed for monolayer formation as described in \cite{Bigioni_2006_Kinet}, where particles are pinned to the surface of the evaporating droplet at which they can find one another to form a single ordered layer.
  Increasing the order of the alkane reduces the evaporative cooling effect from the drop casting experiment, but also increases the time the particles have to order themselves on the droplet surface.
  On the other hand, exchanging octadecene for a lower order alkene reduces the freezing point, as to which no noticeable long-range order is observed for such combinations.
  \\

  In summary, the presented data show that a combination of \textit{n}-heptane, 1-octadecene and oleic acid, with fine-tuned ratios, lead to long-range order for oleic acid capped nanoparticles into two dimensional structures, where the specific lattice order depends on the particle shape.
\end{document}