\providecommand{\main}{../..}
\documentclass[\main/dresen_thesis.tex]{subfiles}

\begin{document}
\chapter{Monolayer}\label{ch:monolayers}
When nanoparticles are aligned as building blocks to form superstructures, the simplest solution is to align them in a plane. 
For nanocubes the arrangement of highest order in the plane is the square lattice, while for nanospheres it's the hexagonal lattice.
In the following, the preparation of such self-assembled lattices with long-range order is presented.
All parameters that have to be considered are discussed and rule of thumbs to estimate their optimal value are explained, such that anybody can transfer the method to another batch of nanoparticles.
For selected monolayer, the quantitative evaluation of the structures is shown, as well as the characterization of their magnetic properties. Emergent magnetic properties that are not observed for the individual nanoparticles are described and compared to model calculations.


\section{Preparation of Monolayer}
An method to prepare monolayers of oleic acid coated nanoparticles of approx. $10 - 20 \unit{nm}$ size by drop-casting is presented in the following. 
Drop-casting is a material efficient and cheap method to prepare thin films. 
However, experimental experience shows that the homogeneity of the samples suffers due to uneven wetting during evaporation. 
The general approach to improve the quality is the usage of co-solvents. 
The idea in the following is to use a combination of a fast-evaporating solvent as main dispersion medium and a slow-evaporating solvent as addend.
After complete evaporation of the fast-evaporating component, a heating step is performed to remove the remaining dispersion medium.
In the following, the various parameters during drop-casting are discussed and optimal value estimation are presented to achieve high-quality monolayers with long range order.

\subsection{Nanoparticle Concentration}

It is found that the quality of the monolayer sample is quantitatively enhanced by using the combination of Heptane and Octadec-1-ene.

\section{Nuclear Structure}
From literature, multiple synthesis routes are known to prepare ferrite nanoparticles with oleic acid coating in this size range. It will be shown in the following chapters that this preparation method is applicable to all tested nanoparticles.

\section{Magnetic Structure}

\section{Emergent Effects}

\section{Model}

\end{document}