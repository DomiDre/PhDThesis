\providecommand{\main}{../..}
\documentclass[\main/dresen_thesis.tex]{subfiles}
  \renewcommand{\thisPath}{\main/chapters/monolayers}

\begin{document}
  \chapter{Monolayer}\label{ch:monolayers}
    For nanocubes the arrangement of highest order in the plane is the square lattice, while for nanospheres it's the hexagonal lattice.
    In the following, the preparation of such self-assembled lattices with long-range order is presented.
    All parameters that have to be considered are discussed and rule of thumbs to estimate their optimal value are explained, such that the method can be transferred to other batch of nanoparticles.
    For selected monolayer, the quantitative evaluation of the structures is shown, as well as the characterization of their magnetic properties. Emergent magnetic properties that are not observed for the individual nanoparticles are described and compared to model calculations.

    \section{Cobalt Ferrite Nanocubes}
      \subfile{\thisPath/nanoparticles/nanoparticles}

    \section{Preparation of Monolayers}
      Drop-casting is a material efficient and cheap method to prepare thin films.
      However, experimental experience shows that the homogeneity of the samples suffers due to uneven wetting during evaporation.
      The general approach to improve the quality achieved by drop casting is the usage of co-solvents.
      A method to prepare monolayers of oleic acid coated nanoparticles of approx. $10 - 20 \unit{nm}$ size by drop-casting with co-solvents is presented in the following.
      The basic idea is to use a combination of a fast-evaporating solvent as main dispersion medium and a slow-evaporating solvent as addend.
      After complete evaporation of the fast-evaporating component, a heating step is performed to remove the remaining dispersion medium.
      The various parameters during drop-casting are discussed and optimal value estimation are presented to achieve high-quality monolayers with long range order.

      \subsection{Nanoparticle Concentration}
        It is found that the quality of the monolayer sample is quantitatively enhanced by using the combination of Heptane and 1-Octadecene.

    \section{Nuclear Structure}
      From literature, multiple synthesis routes are known to prepare ferrite nanoparticles with oleic acid coating in this size range. It will be shown in the following chapters that this preparation method is applicable to all tested nanoparticles.

    \section{Magnetic Structure}

    \section{Emergent Effects}

    \section{Model}

\end{document}