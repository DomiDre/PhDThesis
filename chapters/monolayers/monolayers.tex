\providecommand{\main}{../..}
\documentclass[\main/dresen_thesis.tex]{subfiles}
  \renewcommand{\thisPath}{\main/chapters/monolayers}

\begin{document}
  \chapter{Monolayers}\label{ch:monolayers}
    A primary goal of this work is the study of magnetic dipolar interactions between nanoparticles.
    Due to the highly directional nature of dipolar magnetism, random orientation of the nanoparticles and random packing, as described in the previous chapter, cloud the observation of macroscopic effects arising from the interaction.
    Therefore an intermediate goal towards the observation of dipolar interaction is the successful preparation of highly ordered nanostructures.
    Simple arrangements that have been discussed extensively in theory, are the square lattice and hexagonal lattice in the plane.
    Here, dipolar coupling leads to an emergent macroscopic states for the nanostructure, namely a super antiferromagnetic state for the square lattice and a super ferromagnetic state for the hexagonal lattice \cite{Politi_2002_Dipol, Russier_2001_Calcu, Varon_2013_Dipol}.
    In this work, a procedure to fabricate such long range ordered nanostructures from nanocubes and nanospheres has been developed.

    In the following, the preparation of such self-assembled lattices with long-range order is presented in detail on the case of the square lattice from cobalt ferrite nanocubes.
    The synthesis and characterization of magnetic nanocubes is described first, and subsequently the production of monolayer samples from same nanoparticles.
    Multiple parameters that have to be considered are discussed and rule of thumbs to experimentally obtain their optimal value by small variations around the estimates are explained.
    The transfer to other iron spinel nanocubes and nanospheres based on this procedure is also presented, as well as possible variations in the procedure to improve and go beyond the analyzed samples.
    Finally, for selected nanoparticles and monolayers, the quantitative evaluation of the structures are shown, as well as the characterization of their magnetic properties. Emergent magnetic properties that are not observed for the individual nanoparticles are described and compared to model calculations.

    \section{Cobalt Ferrite Nanocubes}
      \subfile{\thisPath/nanoparticles/nanoparticles}

    \FloatBarrier
    \section{Preparation of Monolayers}
      \subfile{\thisPath/preparationMonolayers/preparationMonolayers}

    \FloatBarrier
    \section{Magnetic Structure}
      \subfile{\thisPath/magneticStructure/magneticStructure}

    \section{Emergent Effects}

    \section{Model}

\end{document}