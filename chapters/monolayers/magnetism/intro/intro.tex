\providecommand{\main}{../../../..}
\documentclass[\main/dresen_thesis.tex]{subfiles}

\begin{document}
  For nanoparticles with long range order in two dimensions, in general five structural lattices are possible by the crystallographic restriction theorem: the square, hexagonal, rectangular, rhombic and oblique lattice.
  In this work, the square and hexagonal lattice are discussed, which can be obtained from densely packing nanocubes and nanospheres, respectively, in the plane.
  When only dipolar interaction between magnetic nanoparticles is considered on these lattices, different energetic ground states emerge for the two lattice types.
  On the hexagonal lattice the ground state is super ferromagnetic, whereas it is super antiferromagnetic for the square lattice as will be discussed in the following for both lattices separately.


  The energy gain only due to dipolar interaction for a transition from a perfectly aligned super ferromagnetic state towards an super antiferromagnetic state can be calculated for zero temperature by performing a double sum over all lattice sites from the point of view of a single magnetic moment.
  Consider the magnetic moment situated in a plane at the origin $(0, \,0)$ with its moment aligned to the $y$ axis $\mu_{00} \eq \mu \hat{e}_y$, and the other magnetic moments at the coordinates $(i a_{pp}, j a_{pp})$, where $i,\, j$ are integer numbers and $a_{pp}$ is the spacing between two particles on the square lattice.
  In the super ferromagnetic case all magnetic moments are considered to point along the $y$ direction $\mu_{ij} \eq \mu \hat{e}_y$, whereas for the super antiferromagnetic case the direction is alternating with the $x$-coordinate $\mu_{ij} \eq (-1)^i \mu \hat{e}_y$ as depicted in Figure ??.
  % figure einfügen.
  The energy is then determined by calculating the Zeeman energy of $\mu_{00}$ with the dipolar magnetic field described in \refeq{eq:theoreticalBackground:magnetism:dipolarField} generated by all the other moments $\mu_{ij}$, which reads
  \begin{align}
    E \eq \sideset{}{'}\sum_{i,j=-\infty}^\infty \frac{\mu_0}{4 \pi} \frac{3(\vec{\mu_{ij}} \cdot \hat{r})(\vec{\mu_{00}} \cdot \hat{r}) - \vec{\mu_{ij}} \cdot \vec{\mu_{00}} }{r^3},
  \end{align}
  where $r \eq \sqrt{i^2 + j^2} a_{pp}$.
  The explicit sum for super ferromagnetic and super antiferromagnetic states is then given by
  \begin{align}
    E^\mathrm{SFM} \eq \frac{\mu_0 \mu^2}{4 \pi a^3_{pp}} \sideset{}{'}\sum_{i,j=-\infty}^\infty  \frac{3j^2}{(i^2 + j^2)^{5/2}} - \frac{1}{(i^2 + j^2)^{3/2}} ,\\
    E^\mathrm{SAFM} \eq \frac{\mu_0 \mu^2}{4 \pi a^3_{pp}} \sideset{}{'}\sum_{i,j=-\infty}^\infty (-1)^i \biggl( \frac{3j^2}{(i^2 + j^2)^{5/2}} - \frac{1}{(i^2 + j^2)^{3/2}} \biggr).
  \end{align}
  The first addend can be transformed by making use of the symmetry between $i,\,j$ via
  \begin{align}
    \sideset{}{'}\sum_{i,j=-\infty}^\infty \frac{3j^2}{(i^2 + j^2)^{5/2}} \eq& \sideset{}{'}\sum_{i,j=-\infty}^\infty \frac{3j^2 + 3i^2 - 3i^2}{(i^2 + j^2)^{5/2}}\\
    \eq& \sideset{}{'}\sum_{i,j=-\infty}^\infty \frac{3}{(i^2 + j^2)^{3/2}} - \frac{3 i^2}{(i^2 + j^2)^{5/2}} \\
    \eq& \frac{1}{2} \sideset{}{'}\sum_{i,j=-\infty}^\infty \frac{3}{(i^2 + j^2)^{3/2}}
  \end{align}
\end{document}