\providecommand{\main}{../../../..}
\documentclass[\main/dresen_thesis.tex]{subfiles}

\begin{document}
  The magnetic structure of the two monolayers, which have been characterized structurally by reflectometry (ML-Ac-CoFe-C) and GISAXS (ML-Ac-CoFe-C-2), are resolved using vibrating sample magnetometry, reflectometry and grazing-incidence scattering with polarized neutrons respectively.
  \\

  From a simple calculation, it is estimated that the relevant energy scale to observe dipolar interactions in terms of a super antiferromagnetic state is below $87 \unit{K}$ and at small magnetic fields in the order of $6 \unit{mT}$.
  For both samples it is determined that the blocking temperature is above room temperature and at low temperature of $5 \unit{K}$ a wide hysteresis is observed with a coercive field of $2.21(1) \unit{T}$ for ML-Ac-CoFe-C and $2.29(1) \unit{T}$ for ML-Ac-CoFe-C-2.
  Comparing the hysteresis of ML-Ac-CoFe-C, measured at $10 \unit{K}$, with the hysteresis of a frozen dispersion of the same nanocubes and a thin layer that is not ordered to square arrays, it is observed that the hysteresis of the monolayer is slightly widened and does not fall drop off as quickly in magnetization upon application of a reversing magnetic field.
  From this it is assumed that the particle interaction stabilizes the magnetic state and therefore acts primarily like a super ferromagnet.
  \\
  A more thorough investigation of the magnetic state is performed first by polarized neutron reflectometry on ML-Ac-CoFe-C.
  In the saturated state, the sample is well-described by a homogeneously magnetized magnetic scattering length density profile that is correlated to the nuclear structure.
  The absolute measured magnitude with $316(7) \unit{kA \, m^{-1}}$ is $15 \%$ smaller than the magnitude that was estimated by VSM to $358(13) \unit{kA \, m^{-1}}$.
  It is concluded that the magnitude from VSM is overestimated, as it has been scaled to it's magnetic volume from assuming that the measured magnetic moment corresponds to the volume of a single particle with the size used as determined from SAXS.
  Instead it is observed that the magnetic volume of a single particle appears on average enlarged, which can again be attributed to a ferromagnetic coupling between the nanoparticles.

  Measuring the sample at a negative field of $-100 \unit{mT}$, the data deviates from the model of a homogeneously  magnetized layer and also no improvement in the fit could be obtained by assuming a reduced size of the magnetic layer.
  This hints that a more complex magnetic structure is actually present in the sample, where the exact determination was not found within the scope of this thesis.
  The magnitude of the magnetization estimated by the simple model in this state is however in full agreement with the VSM measurement, when the change in scale is accounted for.
  \\

  Another approach that focuses on the in-plane order is done by studying ML-Ac-CoFe-C-2 with polGISANS.
  Using the structural model determined from GISAXS, the BornAgain software is used to simulate the expectation for different magnetic configurations on a square lattice.
  These are compared to the observed data measured for the monolayer after zero-field cooling to $5 \unit{K}$ and in remanence after saturating the sample in-plane in the direction of the neutron beam and perpendicular to it.
  A clear signal of an antiferromagnetic state is not observed in either case, and it has to be assumed if it's present that it is greatly broadened out due to a small domain size.
  Measuring the sample at a negative field of $-100 \unit{mT}$ after sample saturation, it is observed by a splitting between the two neutron states that a large fraction of the sample is still aligned with the originally applied saturating field, as would have been expected from the hysteresis.
  This splitting is curiously not observed at the remanent state measured at guide field, which if beam depolarization could be excluded is not fully understood yet.
  \\

  With this study, the first experimental study of the magnetism for nanocubes in a square array that we are aware of is presented.
  It is shown that the sample quality and material is good enough to obtain quantifiable data and opens the door for more complex studies with varied particle materials, nanoparticle sizes and shapes, orientation of the nanoparticle arrays \etc.
  Improving the sample quality further and maybe by allowing for higher counting times during polarized GISANS measurements, a full understanding on the magnetic state of nanoparticles in a long-range ordered square array should be obtainable.
\end{document}