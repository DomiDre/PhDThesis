\providecommand{\main}{../../../..}
\documentclass[\main/dresen_thesis.tex]{subfiles}

\begin{document}
  \label{sec:monolayers:nanoparticle:dropcastingExperiments}

  In the following, the steps in every drop casting experiment are described, as well as the preparation conditions are given for the presented samples in this chapter.
  The general procedure for a drop casting experiment is the following:
  first the substrate and the nanoparticle dispersion need to be prepared.
  The nanoparticle concentration has to be chosen for the dispersion, as well as the the solvent and possible co-solvents, which influence the evaporation driven self-assembly.
  The environment conditions during the solvent evaporation provide further parameters, which can be tuned to control the drop casting experiment.
  And finally, after the droplet has dried, the wafer can be further treated to purify the sample.

  Each step is described in the following, as well as the used methods to characterize their influence on the samples.

  \paragraphNewLine{Preparation of the silicon substrate}
    Single-crystalline silicon substrates with (100) orientation were purchased from Crystec GmbH with a thickness of $0.525 \unit{mm}$ and precutted to $10 \times 10 \unit{mm^2}$ squares.
    The wafers are polished on one side, which is used as surface for the drop casting experiments.
    To remove the protective film from the polished side, the substrates are cleaned using an Elmasonic P60 ultrasonic bath, where the wafer is first submerged in acetone for 10 minutes.
    After that the wafer is removed from the acetone, it is dried by spinning on a KLM Spin-Coater SCC briefly, after which it is submerged in 2-propanol of HPLC grade and sonificated for another 10 minutes.
    The spin-coater is used once more to dry the wafers and the silicon substrates are used within the same hour for the drop casting experiment.

  \paragraphNewLine{Determination of Nanoparticle Concentration}
    The optimal concentration of the nanoparticles in dispersion for a monolayer on a given wafer surface area can be roughly estimated geometrically from the average particle to particle distance on the square lattice $a_{p-p}$.
    Assuming a perfect square lattice expanding over the wafer area $A_{\textsf{wafer}}$, the number of nanocubes $N$ is determined by
    \begin{align}
      N \eq \frac{A_{\textsf{wafer}}}{a_{p-p}^2}.
    \end{align}
    When the volume of a dispersion $V_{\textsf{disp.}}$ is dropped on the wafer, the number of particles in the drop is given by
    \begin{align}
      N \eq \frac{c_V V_{\textsf{disp.}}}{V_p},
    \end{align}
    where $c_V$ is the particle volume concentration of the dispersion and $V_p$ the volume of a single nanocube.
    The average nanoparticle volume is known from the small-angle scattering model, and the particle volume concentration can be estimated by
    \begin{align}
      c_V \eq \frac{c_m}{\rho},
    \end{align}
    using the the particle mass concentration $c_m$ of the dispersion that is estimated from gravimetry, or more exact from a thermogravimetric analysis, and the density $\rho$ of the particle that can be estimated using the literature value.
    Combining these equations, the particle mass concentration of the dispersion needs to be tuned to the value
    \begin{align}\label{eq:monolayers:preparation:particleConcentration}
      c_m \eq \rho \frac{A_{\textsf{wafer}}}{V_{\textsf{disp.}}} \frac{V_p}{a_{p-p}^2}
    \end{align}
    Typically, for ferrite nanocubes with a size in the order of $10 \unit{nm}$ and with an oleic acid shell in the order of $2 \unit{nm}$, the optimal mass concentration for a drop of $50 \unit{\text{\textmu} L}$ on a $10\times 10 \unit{mm^2}$ wafer is in the order of $0.1 \unit{mg/mL}$.
    Using this calculated estimate, the optimal particle concentration can be determined for a batch of nanoparticles by a short experimental drop casting series around this value.
    In the case of a non-square lattices, \refeq{eq:monolayers:preparation:particleConcentration} has to account for the reduced coverage by including an additional factor for the packing density $\eta$ in the plane and $a_{p-p}^2$ has to be replaced by the average maximum cross-section the particle is taking in the lattice.
    For a perfect circle packing, which is expected for densely packed spheres in two dimensions, the packing density is straight forward to evaluate to $\eta \eq \pi \sqrt{3} / 6 \approx 0.9069$ and the cross-section is given by $\pi r^2$.

    The variation of the concentration is not presented in the scope of this thesis to focus in more detail on the other parameters.

    \paragraphNewLine{Drying Process during Drop-Casting}
      For the drop casting experiment, the cleaned silicon substrates are always placed in a clean Petri dish.
      $50 \musf L$ of dispersion is then dropped on the wafer surface and the primary solvent is evaporated at ambient conditions of $22 \unit{^\circ C}$ in the open container.
      After the evaporation of the primary solvent, the Petri dish is covered tightly with tinfoil, which is perforated along the edges by 8 holes, which are symmetrically poked by a syringe needle with a $0.9 \unit{mm}$ outer diameter.
      The Petri dish is placed for at least $12 \unit{h}$ in an oven at $80 \unit{^\circ C}$.
      Without moving the samples, the oven is set for another $6 \unit{h}$ at $140\unit{^\circ C}$.

    \paragraphNewLine{Variation of Alkanes / Alkenes as Solvent / Co-Solvent}
      To study the influence of the solvent and co-solvent, the nanoparticles Ol-CoFe-C are drop casted at the same nanoparticle concentration from varied combinations.
      As primary solvents, the alkanes pentane, hexane and heptane are studied, for which nanoparticle dispersions are prepared at $0.1 \unit{mg mL}$.
      As co-solvent, the alkenes tetradecene (\ch{C14H28}) and octadecene (\ch{C18H36}) are compared.
      In all cases the volume fraction of the co-solvent is set to $2 \%$.
      Additional experiments performed in combination with hexadecene (\ch{C16H32}), dodecene (\ch{C12H24}) and decene (\ch{C10H20}), as well as varied co-solvent volume fractions are left out from this thesis for brevity.


      \begin{table}[!htbp]
        \centering
        \caption{\label{tab:monolayers:charMethod:varyAlkaneAlkene}Monolayers prepared from Ol-CoFe-C for the discussion of the solvent variation.}
        \begin{tabular}{ l | l | c }
          \textbf{Sample} & Solvent              & Co-Solvent \\
          \hline
          ML-SV-HexNone   & $\mathit{n}$-hexane  & -          \\
          ML-SV-HexTet    & $\mathit{n}$-hexane  & tetradecene\\
          ML-SV-PenOct    & $\mathit{n}$-pentane & octadecene \\
          ML-SV-HexOct    & $\mathit{n}$-hexane  & octadecene \\
          ML-SV-HepOct    & $\mathit{n}$-heptane & octadecene \\
          \hline
        \end{tabular}
      \end{table}

      The discussed samples are listed in \reftab{tab:monolayers:charMethod:varyAlkaneAlkene} and are named with the structure ML-SV-$XY$, where $X$ is the abbreviated primary solvent and Y the abbreviated co-solvent.
      By using SEM and GISAXS, the lateral order of the samples is quantified.
      As reference to a sample with no added co-solvent, the sample ML-SV-HexNone is discussed, which is drop casted from $\mathit{n}$-hexane.
      As combination of alkan and alkene, hexane with tetradecene, and $\mathit{n}$-pentane, $\mathit{n}$-hexane and $\mathit{n}$-heptane with 1-octadecene are discussed.

      The GISAXS experiments were performed at GALAXI ($\lambda \eq 1.34 \angstrom$) by measuring the scattering under an grazing-incidence angle of $\alpha_i \eq 0.11^\circ$ over an integrated time of $4 \unit{h}$ each.

      Within the theory of paracrystals, GISAXS provides a direct measure of the lattice structure, lattice constant and the coherence length \cite{Renaud_2009_Probi}.
      For the studied nanocubes a square array is expected as long-range ordered structure.
      To approve the achieved order, the GISAXS data in the Yoneda band is regarded and the peak positions as well as the peak broadening is determined.
      For Gaussian disorder of the nearest neighbor positions on a square array, the peak broadening is described by a Lorentzian function as shown in \refapp{ch:appendix:calculations:paracrystal} and reads for the (10) peak
      \begin{align}
        I(q_y) \eq \frac{A} {1 + \biggl(2\frac{q_y - q_{hk}}{\Delta q_{hk}}\biggr)^2},
      \end{align}
      where $A$ is the intensity of the (10) peak, $q_{10}$ the peak center and $\Delta q_{10}$ the FWHM.
      For a square lattice, the expected peak positions are given by
      \begin{align}\label{eq:monolayers:preparation:solventVariation:squareLatticeIndices}
        q_{hk} \eq \frac{2 \pi}{a} \sqrt{h^2 + k^2},
      \end{align}
      where $h$, $k$ are integers and $a$ the lattice constant of the square array.

      By fitting the (10) peak in the Yoneda band to a Lorentzian, $q_{10}$ and $\Delta q_{10}$ are determined for each sample, where from $q_{10}$ the lattice constant is obtained by
      \begin{align}
        a &\eq \frac{2 \pi}{q_{10}}.
      \end{align}
      By comparing the expected peak positions for the given lattice constant, to the observed data in the Yoneda band, the square lattice structure is evaluated.
      From $\Delta q_{10}$ the coherence length is given by
      \begin{align}
        d_{coh.} &\eq \frac{4 \pi^2}{\Delta q_{10} - \Delta q_{D.B.}},
      \end{align}
      where the FWHM of the peak is corrected for $\Delta q_{D.B.} \eq 0.0195(7) \unit{nm^{-1}}$, the FWHM of the direct beam given for GALAXI in \reffig{fig:lss:galaxi:directBeam}.

    \paragraphNewLine{Oleic Acid Addend}
      Monolayers from Ac-CoFe-C-3 are prepared with varied oleic acid content to study its influence on the monolayer formation.
      Ac-CoFe-C-3 is chosen for this variation study as it is a nanoparticle batch that shows bad self-assembly properties in comparison to the other presented nanoparticles.

      To study the influence of oleic acid, $5 \unit{mL}$ dispersion with $c \eq 0.12 \unit{mg \, mL^{-1}}$ is prepared from Ac-CoFe-C-3 in $\mathit{n}$-heptane, to which $100 \unit{\musf L}$ 1-octadecene is added.
      After ultra sonification, the dispersion is seperated into 5 fractions of, $1 \unit{mL}$ each, to which varied amount of oleic acid is added.

      Each sample is prepared by drop casting $50 \unit{\musf L}$ of each dispersion on a separate silicon substrate, which is placed in a Petri dish.
      After the evaporation of the $\mathit{n}$-heptane, the Petri dish is covered by tinfoil and the samples are kept in an oven at $80 \unit{^\circ C}$ for $\unit{12 \unit{h}}$ and subsequently kept at $140 \unit{^\circ C}$ for $\unit{6 \unit{h}}$.
      The quality of the five samples are then studied by scanning electron microscopy to evaluate the effect of the oleic acid on the order formation.

      The prepared samples are listed in \reftab{tab:monolayers:charMethod:OAVariation} as ML-OA-x, where x is the volume fraction of oleic acid, which are $0$, $5 \cdot 10^{-6}$, $1 \cdot 10^{-5}$, $5 \cdot 10^{-5}$, $5 \cdot 10^{-5}$ and , $1 \cdot 10^{-4}$.

      \begin{table}[!htbp]
        \centering
        \caption{\label{tab:monolayers:charMethod:OAVariation}Monolayer samples prepared from Ac-CoFe-C-3 with varied oleic acid volume fraction $c_V^\mathrm{OA}$ in the dispersion during drop casting.}
        \begin{tabular}{ l | c }
          \textbf{Sample} & $c_V^\mathrm{OA} \, / \%$\\
          \hline
          ML-OA-$0$ & $0$\\
          ML-OA-$5 \cdot 10^{-6}$ & $5 \cdot 10^{-4}$\\
          ML-OA-$1 \cdot 10^{-5}$ & $1 \cdot 10^{-3}$\\
          ML-OA-$5 \cdot 10^{-5}$ & $5 \cdot 10^{-3}$\\
          ML-OA-$1 \cdot 10^{-4}$ & $1 \cdot 10^{-2}$\\
          \hline
        \end{tabular}
      \end{table}

    \paragraphNewLine{Drop-Casting of Ac-CoFe-C and Ac-CoFe-C-2}
      For the study of magnetic monolayers, samples from Ac-CoFe-C and Ac-CoFe-C-2 are drop casted on a silicon substrate.
      ML-Ac-CoFe-C was prepared by dispersing Ac-CoFe-C in $\mathit{n}$-heptane at a concentration of $0.12 \unit{mg \, mL^{-1}}$.
      And for ML-Ac-CoFe-C-2, Ac-CoFe-C-2 was dispersed in $\mathit{n}$-hexane at a concentration of $0.13 \unit{mg \, mL^{-1}}$.
      In both cases $2 \unit{\%}$ of octadecene are added in volume fraction to the dispersion, before the drop casting experiments is performed with $50 \musf L$ of dispersion as described before.

      The samples are qualitatively characterized by scanning electron microscopy.
      Then ML-Ac-CoFe-C is used for the study of the vertical electronic, nuclear and magnetic structure by XRR and PNR, and ML-Ac-CoFe-C-2 is used for the study of the lateral structure in a magnetic monolayer by GISAXS and GISANS.
      For ML-Ac-CoFe-C-2 an additional XRR study is performed analogue to ML-Ac-CoFe-C.
      The characterization methods employed for both studies are described in the following section.

    % \paragraphNewLine{Summary of Drop-Casted Monolayers}
    %   From the presented nanoparticles, monolayers are prepared using varied primary solvents and varied co-solvents.
    %   In all cases, the co-solvent has a $2 \,\%$ volume fraction in dispersion.
    %   Additionally a series is studied, where oleic acid is added additionally with varied concentration to study the achievement of long-range order for a nanoparticle batch that does not order properly without.
    %   A summary of all prepared nanoparticle concentrations, solvent and co-solvent combinations, as well as oleic acid addends is given in \reftab{tab:monolayers:charMethod:sampleConcentrations}.

      % \begin{table}[!htbp]
      %   \centering
      %   \caption{\label{tab:monolayers:charMethod:sampleConcentrations}Monolayer samples discussed in this chapter and from which nanoparticles (NP) they are prepared. Furthermore it is stated in which solvent they are dispersed, which concentration $c_m^\mathrm{NP}$ is set (in units of $\unit{mg \, mL^{-1}}$), which co-solvent was used and finally what volume fraction of oleic acid $c_V^\mathrm{OA}$ was additionally added to the dispersion.}
      %   \begin{tabular}{ l | l | l | c | c | c }
      %     \textbf{Sample} & NP          & Solvent              & $c_m^\mathrm{NP}$ & Co-Solvent  & $c_V^\mathrm{OA} \, / \%$\\
      %     \hline
      %     ML-SV-HexNone            & Ol-CoFe-C   & $\mathit{n}$-hexane  & $0.10$             & -  & $-$\\
      %     ML-SV-HexTet             & Ol-CoFe-C   & $\mathit{n}$-hexane  & $0.10$             & tetradecene & $-$\\
      %     ML-SV-PenOct             & Ol-CoFe-C   & $\mathit{n}$-pentane & $0.10$             & octadecene  & $-$\\
      %     ML-SV-HexOct             & Ol-CoFe-C   & $\mathit{n}$-hexane  & $0.10$             & octadecene  & $-$\\
      %     ML-SV-HepOct             & Ol-CoFe-C   & $\mathit{n}$-heptane & $0.10$             & octadecene  & $-$\\
      %     ML-OA-0                  & Ac-CoFe-C-3 & $\mathit{n}$-heptane & $0.12$            & octadecene  & $0$\\
      %     ML-OA-$5 \cdot 10^{-6}$  & Ac-CoFe-C-3 & $\mathit{n}$-heptane & $0.12$            & octadecene  & $5 \cdot 10^{-4}$\\
      %     ML-OA-$1 \cdot 10^{-5}$  & Ac-CoFe-C-3 & $\mathit{n}$-heptane & $0.12$            & octadecene  & $1 \cdot 10^{-3}$\\
      %     ML-OA-$5 \cdot 10^{-5}$  & Ac-CoFe-C-3 & $\mathit{n}$-heptane & $0.12$            & octadecene  & $5 \cdot 10^{-3}$\\
      %     ML-OA-$1 \cdot 10^{-4}$  & Ac-CoFe-C-3 & $\mathit{n}$-heptane & $0.12$            & octadecene  & $1 \cdot 10^{-2}$\\
      %     ML-Ac-CoFe-C             & Ac-CoFe-C   & $\mathit{n}$-heptane & $0.12$            & octadecene  & $-$\\
      %     ML-Ac-CoFe-C-2           & Ac-CoFe-C-2 & $\mathit{n}$-hexane  & $0.13$            & octadecene  & $-$\\
      %     \hline
      %   \end{tabular}
      % \end{table}
\end{document}



