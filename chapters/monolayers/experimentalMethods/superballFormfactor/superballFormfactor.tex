\providecommand{\main}{../../../..}
\documentclass[\main/dresen_thesis.tex]{subfiles}

\begin{document}
  \label{sec:monolayers:nanoparticle:structuralCharacterization}
    \begin{figure}[tb]
      \centering
      \includegraphics{appendix_superballShapes}
      \caption{\label{fig:monolayers:nanoparticle:superballShapes}Visualization of the superball shape for varied $p$.}
    \end{figure}
    To evaluate the small-angle scattering data quantitatively, the experimental data is used to fit the parameters of a superball form factor that was derived and implemented in the course of this thesis.
    A superball is a mathematical shape that can be used to describe rounded cubes and it's volume is defined by
    \begin{align}
      x^{2p} + y^{2p} + z^{2p} < R^{2p},
    \end{align}
    where $R$ is the radius of the superball and $p$ describes whether the body is closer to a sphere or a cube.
    \reffig{fig:monolayers:nanoparticle:superballShapes} shows the surface of a superball for varied $p$.
    For $p\eq 1$ the superball is equivalent to the definition of a sphere and for $p \rightarrow \infty$ the superball converges to a cube with edge length $a \eq 2R$.

    The full description and derivation of the superball properties and formfactor is described in \refapp{ch:appendix:numericalMethods:superballFormfactor}.
    The result here is that the form factor amplitude of an oriented superball is given by the integral
    \begin{align}
      \begin{split}
        p_\mathrm{orient}(\vec{q}) &\eq \frac{2R^3}{q_z R V_p} \int_{-1}^{1} \dint x \int_{-\gamma}^{\gamma} \dint y \cos(R q_x x + R q_y y)  \sin (q_z R \zeta),\\
        &\mathrm{with}\\
        \gamma \eq& \sqrt[2p]{1-x^{2p}}, \\
        \zeta \eq& \sqrt[2p]{1-x^{2p} -y^{2p}}.
      \end{split}
    \end{align}

    For the data fit, an average over the size distribution $\Lambda(R)$ and an orientation average over one octant is performed as described in \refapp{ch:appendix:numericalMethods:superballFormfactor}, and the form factor is multiplied with a parameter for the particle density and contrast to the solvent
    \begin{align}
      \label{eq:superballFormfactorIntensity}
      I_\mathrm{Superball}(q) = \frac{2 I_0 \Delta \rho^2}{\pi} \int_0^{\pi/2} \dint \varphi \int_0^{\pi/2} \dint \theta \sin (\theta)  \int_0^\infty \dint R \Lambda(R) V_p^2 |p_\mathrm{orient.}(\vec{q}; R)|^2,
    \end{align}

    To account for an oleic acid surfactant of the nanoparticles, the superball is assumed to have a core-shell structure.
    This is done by replacing the amplitude in \refeq{eq:superballFormfactorIntensity} by
    \begin{align}
      \begin{split}
        &\Delta \rho p_\mathrm{orient.}(\vec{q}; R) \rightarrow\\
        &\hspace{2cm}(\rho_\mathrm{shell} - \rho_\mathrm{solvent}) p_\mathrm{orient.}(\vec{q}; R+D) + (\rho_\mathrm{core} - \rho_\mathrm{shell}) p_\mathrm{orient.}(\vec{q}; R),
      \end{split}
    \end{align}
    where $D$ is the thickness of the shell.
\end{document}