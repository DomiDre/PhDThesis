\providecommand{\main}{../../../..}
\documentclass[\main/dresen_thesis.tex]{subfiles}

\begin{document}
  \paragraphNewLine{Vibrating Sample Magnetometry}
    The monolayers are measured field- and temperature-dependent using a PPMS Evercool II (\refsec{ch:instruments:laboratoryInstruments:vsm}).
    An approximately $5 \times 5 \unit{mm^2}$ piece of ML-Ac-CoFe-C is cut out of the sample for the measurement using a diamond cutter.
    To avoid destroying the sample ML-Ac-CoFe-C-2, the equivalent sample ML-Ac-CoFe-C-2* that is also shown in the cross-sectional SEM micrographs is used in the magnetization measurements and cut to a similar size.

    Both samples are measured field-dependent in a range of $\pm 9 \unit{T}$ at $300 \unit{K}$ and $5 \unit{K}$ with a sweeping rate of $5 \unit{mT \, s^{-1}}$.
    The sample magnetization is furthermore measured temperature-dependent after zero-field cooling and field cooling at $10 \unit{mT}$ from $10 \unit{K}$ to $350 \unit{K}$ while warming the sample with a rate of $1.5 \unit{K \, s^{-1}}$.

  \paragraphNewLine{Polarized Neutron Reflectometry}
    Additional to the nuclear structure, the magnetic structure of ML-Ac-CoFe-C is discussed by measuring its reflectivity with polarized neutron at the D17 instrument.
    Additional to the measurement at guide field after zero-field cooling, described in the previous paragraph, the sample was measured afterwards at a saturating field of $6 \unit{T}$ and a negative field of $-100 \unit{mT}$.
    The reflectivity is measured using two incident angles of $0.50^\circ ,\, 1.8^\circ$ to cover a $q$-range of $0 - 1 \unit{nm^{-1}}$.
    And the data reduction is performed in analogue to the initial guide field measurement using the COSMOS software.

    The data is fit using the same model for the magnetic scattering length density as for the nuclear structure determination in the non-magnetized state before only manipulating the explicit magnetic SLD values of the specific materials.
    All materials except the nanocubes are set to have magnetic SLD of zero and thus leaving only a single parameter to be refined.


  \paragraphNewLine{Polarized Grazing-Incidence Small-Angle Neutron Scattering}
    To search for a super antiferromagnetic signal, the monolayer ML-Ac-CoFe-C-2 has been measured at the D33 instrument at the ILL in grazing-incidence small-angle geometry with polarized neutrons.
    The wavelength of the neutrons is set to $6 \unit{\angstrom}$ for the measurement and the sample-to-detector distance was chosen at $5 \unit{m}$ to measure a $q$-range of $\pm 0.66 \unit{nm^{-1}}$.
    The sample is cooled in zero-field to $5 \unit{K}$, measured initially at a guide field of $5 \unit{mT}$, after application and removal of a $4 \unit{T}$ and at a negative field of $-200 \unit{mT}$.
    Each measurement is performed without and with the polarization flipper turned on and for $2 \unit{h}$ per channel each.
    The sample incident angle for the measurements is set to $0.35 ^\circ$ by tilting the sample and cryomagnet as a whole.
    Furthermore the sample is rotated by $90 ^\circ$ with the same measurements performed again in unpolarized mode.

    Using the structure of the sample determined by GISAXS and the magnetic properties of the nanocubes from SANSPOL, simulations of the expected polGISANS signal are performed with the BornAgain software \cite{Burle_2018_borna} for different limiting cases of magnetic configurations on a square lattice.
    For the simulation, the beam and detector parameters have to be exchanged for those of D33, where a rectangular detector with $128 \times 256$ pixels and $640 \times 640 \unit{mm}^2$ is given.
    The detector is, also given a Gaussian resolution function of $5 \times 5 \unit{mm^2}$ spot size, to reproduce the finite resolution of the detector.
    The results are compared to the observed scattering to discuss the observed sample state.
\end{document}