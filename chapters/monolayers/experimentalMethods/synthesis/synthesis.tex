\providecommand{\main}{../../../..}
\documentclass[\main/dresen_thesis.tex]{subfiles}

\begin{document}
  \label{sec:monolayers:nanoparticle:synthesisOleatesAcAc}
  In the following, the steps that have been done for the preparation of cobalt ferrite nanocubes following the oleate and acetylacetonate route is described.

  \paragraphNewLine{Preparation of Cobalt Ferrite Oleate}
    In the first presented protocol to synthesize cobalt ferrite nanoparticles, a cobalt and iron oleate mixture is prepared as first step.
    For this purpose a clear solution of sodium oleate is prepared by dissolving $96 \unit{mmol}$ of \ch{NaOH} in $20 \unit{mL}$ of each \ch{H2O} and \ch{EtOH} and subsequently adding drop wise $96 \unit{mmol}$ of oleic acid under constant stirring.
    Then $12 \unit{mmol}$ of \ch{CoCl2 * 6 H2O} and $24 \unit{mmol}$ of \ch{FeCl3 * 6 H2O} are dissolved in $5 \unit{mL}$ \ch{H2O} and $15 \unit{mL}$ \ch{EtOH} each and added to the solution.
    After addition of $80 \unit{mL}$ \ch{H2O} and \ch{EtOH} each, as well as $160 \unit{mL}$ n-hexane, the mixture is held at reflux ($60 \unit{^\circ C}$) for $4 \unit{h}$ under constant strong magnetic stirring.
    Once the mixture is cooled back to room temperature, it is washed three times in a separatory funnel with $30 \unit{mL}$ \ch{H2O} each to remove \ch{NaCl}.
    The remaining n-hexane, ethanol and water is removed using a rotary evaporator.
    In the end, approximately $32 \unit{g}$ of a dark red and highly viscous metal oleate complex is obtained, which is then ready to be used for the nanoparticle synthesis.

  \paragraphNewLine{Preparation of \ch{CoFe2O4} Nanocubes from Oleate}
    To prepare nanoparticles, $10 \unit{mmol}$ of the cobalt ferrite oleate is dissolved in $50 \unit{mL}$ 1-octadecene within a $250 \unit{mL}$ three-neck round-bottom flask.
    To obtain cubically shaped nanoparticles, sodium oleate is prepared separately by dissolving $2.5 \unit{mmol}$ \ch{NaOH} in $10$ drops of \ch{H2O} and \ch{EtOH} and adding $2.5 \unit{mmol}$ of oleic acid drop wise while ultra sonificating the mixture.
    The sodium oleate is added to the dissolved oleate together with additional $2.5 \unit{mmol}$ oleic acid.
    The mixture is heated and held at $150 \unit{^\circ C}$ for one hour under constant magnetic stirring until all water and ethanol is evaporated.
    A fractionating column is put on the round-bottom flask and nitrogen is gently bubbled into the mixture.
    Using a temperature controller, the mixture is heated to reflux at approximately $315 \unit{^\circ C}$ with a gradient of $2.5 \unit{^\circ C min^{-1}}$, where it is held for $30 \unit{min}$.
    After cooling the reaction naturally to room temperature, the particles are precipitated with \ch{EtOAc} and \ch{EtOH}, centrifuged at $8000 \unit{rpm}$ and redispersed in n-hexane until the supernatant is clear.
    In the last step the mixture is centrifuged without adding \ch{EtOAc}/\ch{EtOH} and the supernatant fluid is taken as dispersion, where as the precipitate is thrown away as being unstable.
    This synthesis yields approximately $500 \unit{mg}$ nanocubes (yield $\approx 20 \%$) and is referred to in the following as Ol-CoFe-C.


  \paragraphNewLine{Preparation of \ch{CoFe2O4} Nanocubes from Acetylacetonates}
    To synthesize nanocubes from acetylacetonates, $0.52 \unit{mmol}$ of \ch{Co(acac)2}, $0.8 \unit{mmol}$ of \ch{Fe(acac)3}, $3 \unit{mmol}$ of freshly prepared sodium oleate and $3 \unit{mmol}$ of oleic acid are diluted in $10 \unit{mL}$ of dibenzyl ether in a $50 \unit{mL}$ three-neck round-bottom flask.
    The mixture is heated to $120 \unit{^\circ C}$ and held here for $1 \unit{h}$.
    After putting a fractionating column on the round-bottom flask, a temperature controller is used to heat the mixture to reflux at about $290 \unit{^\circ C}$ with a heating rate of $5 \unit{^\circ C min^{-1}}$.
    During the synthesis, nitrogen is blown gently over the solution all the time to form an inert blanket and the solution is magnetically stirred.
    After cool down, the product is transferred with n-hexane to centrifugal tubes and precipitated with \ch{EtOH}.
    After centrifugation at $8500 \unit{rpm}$ the supernatant is discarded and the precipitate is redispersed with n-hexane.
    This procedure is repeated three times, and in the last step the precipitate is dispersed in n-hexane without precipitating it again.
    After a final centrifugation at $8500 \unit{rpm}$ the supernatant is kept as dispersion.
    The synthesis yields approximately $50 \unit{mg}$ nanocubes (yield $\approx 45 \%$) and is referred to in the following as Ac-CoFe-C.
\end{document}