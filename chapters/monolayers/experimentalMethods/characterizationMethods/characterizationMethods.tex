\providecommand{\main}{../../../..}
\documentclass[\main/dresen_thesis.tex]{subfiles}

\begin{document}
  \label{sec:monolayers:nanoparticle:structuralCharacterization}

  \paragraphNewLine{Transmission Electron Microscopy}
    Transmission electron microscopy is used to visually characterize a sample of the prepared dispersion and validate the structural quality of the batch.
    For the measurement a drop of a dispersion with a concentration of $c \approx 0.1 \unit{mg/mL}$ is transferred to a copper grid and measured on a Zeiss Leo 902, which is described in \refsec{ch:instruments:laboratoryInstruments:tem}.
    To evaluate the particle size and size distribution, the edge length of over 150 nanocubes is measured for both Ol-CoFe-C and Al-CoFe-C and evaluated by fitting a log-normal distribution as described in \refch{ch:methods:em}.

  \paragraphNewLine{Small-Angle Scattering}
    For the measurement of SAXS of Ol-CoFe-C and Al-CoFe-C in $\textit{n}$-hexane are filled in borosilicate capillaries obtained from the company Hilgenberg with $1.5 \unit{mm}$ diameter and a $0.01 \unit{mm}$ wall thickness, which are sealed using a plastic stopper and glue gun.
    The samples are measured on two distances using the GALAXI instrument (\refch{ch:lss:galaxi}) at the \textsc{Forschungszentrum J\"ulich}. 
    Two measurements are performed for each sample, one at a large sample-to-detector distance of $3.53 \unit{m}$ and one at a short sample-to-detector distance of $0.83 \unit{m}$.
    Additionally, a capillary filled with the solvent and an empty capillary is measured under the same conditions for subtraction.

    To measure SANS and SANSPOL, the dispersions are dried at ambient conditions and redispersed in toluene-$\mathrm{d_8}$ to reduce the incoherent scattering coming from hydrogen atoms.
    The small-angle neutron scattering data of Ac-CoFe-C was measured at D22 (\refch{ch:lss:d22}) and Ol-CoFe-C on the D33 instrument (\refch{ch:lss:d33}) at the Institut Laue-Langevin.

  \paragraphNewLine{Vibrating Sample Magnetometry}
    To obtain magnetometry data of non-interacting particles varied approaches were considered.
    Measuring nanoparticles in a diluted dispersion has the advantage that a large amount of particles can be measured with a large relative distance to one another.
    However, a moving liquid sample is a difficult system in the framework of vibrating sample magnetometry as often due to air gaps the liquid is still able to move within its container during the vibration.
    The movement as a whole can lead to an out of phase vibration of the particles or shift of the sample center from calibration and thus to a systematic error in the measured signal \cite{Boekelheide_2016_Artif}.
    Freezing the liquid solves this problem to most parts, but for the case of cobalt ferrite, which typically has a blocking temperature around $300 \unit{K}$, this does not allow to measure the superparamagnetic state and therefore to directly access the single particle magnetic moment.

    Alternatively, dried nanoparticles can be measured on a substrate to have them fixed and non-interacting during the measurement.
    It was chosen to use samples where a diluted dispersion was dried quickly from n-hexane and without any directional control.
    Even though nearest neighbour interaction might still occur in these samples, no structural long-range order is observable and therefore it is expected that dipolar interaction is negligible for the magnetic properties.

    To obtain the magnetization scaled to the total particle volume in units of $\unit{kAm^{-1}}$, the data from magnetometery, which is given in units of $\unit{emu}$ is rescaled using the single particle volume from the superball model fit in small angle X-ray scattering and the magnetic moment determined from a Langevin curve fit that includes size distribution effects
    \begin{align}
      M \eq M_s
      \frac
      {\int_{0}^{\infty} p(\mu; \bar{\mu}, \sigma_\mu) \mu \biggl( \coth\Bigl(\frac{\mu B}{k_B T} \Bigr) - \frac{k_B T}{\mu B} \biggr) \dint \mu}
      {E[\mu]},
    \end{align}
    where $\sigma_\mu \eq 3 \sigma_a$ is set to be three times the particle size distribution obtained from SAXS, assuming the magnetic moment scales proportionally with the volume, $p(\mu; \bar{\mu}, \sigma_u)$ is a normalized lognormal distribution and $E[\mu]$ is the expectation value of $\mu$.
    With the hereby determined magnetic moment $\bar{\mu}$, the saturation magnetization is scaled to be
    \begin{align}
      M_s \eq \frac{\bar{\mu}}{V_p}.
    \end{align}

  \paragraphNewLine{Temperature-Dependant VSM}
    To study the samples further, the nanoparticles are measured in VSM at varied temperatures down to $10 \unit{K}$, where the N\'eel relaxation of the superspin moment is suppressed and the blocked single-domain particle is observed.
    The magnetometry data is rescaled using the same factors that were determined at room temperature with the SAXS superball volume and Langevin fit.
    Both hysteresis are given in \reffig{fig:monolaye rs:nanoparticle:vsm10K} and show a large coercivity, where the coercive field for Ol-CoFe-C is at $1.2 \unit{T}$ and for Ac-CoFe-C at $1.9 \unit{T}$, which is the typical range that is also in literature for cobalt ferrite nanoparticles.
    From the temperature dependent magnetization measurements, the blocking temperature of the two particle batches can be estimated, which is $218 \unit{K}$ for Ol-CoFe-C and $314 \unit{K}$ for Ac-CoFe-C.
    % Zitate Nanoparticle Koerzitivfelder
    Additionally, Ol-CoFe-C shows an exchange bias effect, when the hysteresis is measured after cooling in a strong magnetic field such as shown by the red curve, which was obtained after cooling the sample in a $1 \unit{T}$ field.
    %Discuss Exchange Bias Effect

  \paragraphNewLine{X-Ray Diffraction}
    XRD of Ac-CoFe-C and Ol-CoFe-C was measured in cooperation with the group of Daniel Nižňanský from the Department of Inorganic Chemistry at the Charles University in Prague on an PANanalytical X'Pert PRO, which is described in \refch{ch:instruments:laboratoryInstruments:xrd}.
    For the measurement, dried samples of the nanoparticles were send to the laboratory of the collaboration, where they were redispersed and dried on glass substrates for the measurement.
    The data is analyzed using the FullProf suite \cite{Rodriguez_1993_Recen} as described in \refch{ch:methods:xrd}.
    In both cases, the expected inverse spinell structure of cobalt ferrite (space group Fd$\bar{3}$m, No. 227) is fitted to the XRD data.
\end{document}