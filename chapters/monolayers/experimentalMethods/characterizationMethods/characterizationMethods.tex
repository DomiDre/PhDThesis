\providecommand{\main}{../../../..}
\documentclass[\main/dresen_thesis.tex]{subfiles}

\begin{document}
  \label{sec:monolayers:nanoparticle:structuralCharacterization}

  \paragraphNewLine{Transmission Electron Microscopy}
    Transmission electron microscopy is used to visually characterize a sample of the prepared dispersion and validate the structural quality of the batch.
    For the measurement in each case a drop of a dispersion with a concentration of $c \approx 0.1 \unit{mg/mL}$ is transferred to a copper grid and measured on a Zeiss Leo 902 (\refsec{ch:instruments:laboratoryInstruments:tem}).
    To evaluate the particle size and size distribution, the edge length of over 150 nanocubes is measured and evaluated by fitting a log-normal distribution as described in \refch{ch:methods:em}.

  \paragraphNewLine{X-Ray Diffraction}
    XRD of Ac-CoFe-C and Ol-CoFe-C was measured in cooperation with the group of Daniel Nižňanský from the Department of Inorganic Chemistry at the Charles University in Prague on an PANanalytical X'Pert PRO, which is described in \refch{ch:instruments:laboratoryInstruments:xrd}.
    For the measurement, dried samples of the nanoparticles were sent to the laboratory of the collaboration, where they were redispersed and dried on glass substrates for the measurement.
    The diffractometer operated with a Cu-K$\alpha$ source ($\lambda \eq 1.54 \angstrom$) and a $2 \theta \eq 5^\circ \ldots 80^\circ$ has been measured.
    The data is analyzed using the FullProf suite \cite{Rodriguez_1993_Recen} as described in \refch{ch:methods:xrd}.
    In both cases a LeBail refinement has been performed comparing the data to the phase of the inverse spinell structure of cobalt ferrite (space group $Fd\bar{3}m$, No. 227).

  \paragraphNewLine{Energy-Dispersive X-Ray Spectroscopy}
    Using the scanning electron microscope Neon Zeiss 40 (\refch{ch:instruments:laboratoryInstruments:sem}) EDX measurements of the nanoparticles were performed.
    The electron beam was accelerated with $20 \unit{kV}$ and in each case the X-ray spectra was measured five times at different positions of the sample over a time of $5 \unit{min}$.
    The relative ratio of cobalt and iron in the sample was obtained by performing Gaussian fits for the relative intensity of the K$\alpha$ energies using the literature values.

  \paragraphNewLine{Small-Angle Scattering}
    For the measurement of SAXS of Ol-CoFe-C and Al-CoFe-C in $\textit{n}$-hexane are filled in borosilicate capillaries obtained from the company Hilgenberg with $1.5 \unit{mm}$ diameter and a $0.01 \unit{mm}$ wall thickness, which are sealed using a plastic stopper and glue gun.
    The samples are measured on two distances using the GALAXI instrument (\refch{ch:lss:galaxi}) at the \textsc{Forschungszentrum J\"ulich}. 
    Two measurements are performed for each sample, one at a large sample-to-detector distance of $3.53 \unit{m}$ and one at a short sample-to-detector distance of $0.83 \unit{m}$.
    Additionally, a capillary filled with the solvent and an empty capillary is measured under the same conditions for subtraction.

    To measure SANS and SANSPOL, the dispersions are dried at ambient conditions and redispersed in toluene-$\mathrm{d_8}$ to reduce the incoherent scattering coming from hydrogen atoms.
    The small-angle neutron scattering data of Ac-CoFe-C was measured at D22 (\refch{ch:lss:d22}) and Ol-CoFe-C on the D33 instrument (\refch{ch:lss:d33}) at the Institut Laue-Langevin.

  \paragraphNewLine{Vibrating Sample Magnetometry}
    The macroscopic magnetization of the nanoparticles was measured using the PPMS Evercool II described in \refch{ch:instruments:laboratoryInstruments:vsm}.
    To obtain the magnetization in a non-interacting state at room temperature, two approaches have been considered.
    In the first, $40 \unit{\musf}L$ of the nanoparticle dispersion are sealed in a vial as described in \refch{ch:instruments:laboratoryInstruments:vsm}.
    The vial is subsequently fixed within a drinking straw, using additional folded straws to render the vial immobile.
    A moving liquid sample is a difficult system in the framework of vibrating sample magnetometry as due to air gaps the liquid is still able to move within its container during the vibration, which can lead to a systematic error in the measured signal \cite{Boekelheide_2016_Artif}.
    Therefore, in a second approach, the nanoparticle dispersion is quickly drop-casted in a diluted concentration ($c \eq 0.1 \unit{mg/mL}$) on a silicon substrate using $\textit{n}$-hexane as solvent to measure the nanoparticles in a dry state for comparison.

    For both approaches hysteresis curves are measured at $10 \unit{K}$ and $300 \unit{K}$, as well as on five additional temperatures in between ($20,\,50,\,100,\,150,\,200\unit{K}$).
    For the dry particles on a substrate, zero-field-, field-cooled warming curves are measured at a field of $10 \unit{mT}$ for a temperature range from $10 \ldots 350 \unit{K}$ with a heating rate of $1.5 \unit{K min^{-1}}$.

    The scaling of the data to the volume of the sample is performed as described in \refch{ch:methods:vsm}:
    By fitting the room-temperature data to a Langevin behaviour plus a term for an excess susceptibility, the magnetization data is scaled such that the saturation magnetization is given by the ratio of the single-particle moment and volume.
    \begin{align}
      M_s \eq \frac{\bar{\mu}}{V_p}.
    \end{align}
    The average particle volume and particle size distribution is hereby taken from the previously obtained SAXS data evaluation and the standard deviation of the magnetic moment is set to be three times the particle size distribution $\sigma_\mu \eq 3 \sigma_a$, assuming the magnetic moment scales proportionally with the volume.

  % \paragraphNewLine{Temperature-Dependant VSM}
  %   To study the samples further, the nanoparticles are measured in VSM at varied temperatures down to $10 \unit{K}$, where the N\'eel relaxation of the superspin moment is suppressed and the blocked single-domain particle is observed.
  %   The magnetometry data is rescaled using the same factors that were determined at room temperature with the SAXS superball volume and Langevin fit.
  %   Both hysteresis are given in \reffig{fig:monolaye rs:nanoparticle:vsm10K} and show a large coercivity, where the coercive field for Ol-CoFe-C is at $1.2 \unit{T}$ and for Ac-CoFe-C at $1.9 \unit{T}$, which is the typical range that is also in literature for cobalt ferrite nanoparticles.
  %   From the temperature dependent magnetization measurements, the blocking temperature of the two particle batches can be estimated, which is $218 \unit{K}$ for Ol-CoFe-C and $314 \unit{K}$ for Ac-CoFe-C.
  %   % Zitate Nanoparticle Koerzitivfelder
  %   Additionally, Ol-CoFe-C shows an exchange bias effect, when the hysteresis is measured after cooling in a strong magnetic field such as shown by the red curve, which was obtained after cooling the sample in a $1 \unit{T}$ field.
  %   %Discuss Exchange Bias Effect
\end{document}