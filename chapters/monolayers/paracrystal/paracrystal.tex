\providecommand{\main}{../../..}
\documentclass[\main/dresen_thesis.tex]{subfiles}
  \renewcommand{\thisPath}{\main/chapters/monolayers/paracrystal}

\begin{document}
  The long-range order of nanoparticle arrays over large areas can be quantitatively studied using x-ray and neutron scattering experiments.
  Whereas SEM is a method to obtain a local image of the structure, it leaves the possibility that the actual sample has actually great variations across the substrate, which might be overlooked.
  Also microscopy only gives an idea of the surface of the structure, while to study the vertical structure it becomes necessary to break the sample to image the cross-section.
  From grazing-incidence scattering and reflectometry experiments on the other hand the full average three dimensional information is obtained non-invasive from a large area of the sample.

  In the following the used models to describe the square array structure of a selected magnetic sample from Ac-CoFe-C are discussed.
  The results of the nuclear structure of the sample will then be a prerequisite to study the magnetic structure of the samples in the following step.

  \subsection{Square Array Paracrystal}
    \subfile{\thisPath/squareArrayParacrystal/squareArrayParacrystal}

\end{document}