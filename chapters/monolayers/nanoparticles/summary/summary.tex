\providecommand{\main}{../../../..}
\documentclass[\main/dresen_thesis.tex]{subfiles}

\begin{document}
  \label{sec:monolayers:nanoparticle:discussion:summary}
  Two literature known heating up syntheses of cobalt ferrite nanocubes have been performed and the prepared nanocubes have been characterized structurally and magnetically using X-ray and neutron scattering methods, as well as electron microscopy and vibrating sample magnetometry.
  \\

  The nanocubes Ol-CoFe-C prepared according to the synthesis route from cobalt- and iron oleates, show a homogeneous cubical shape and narrow size distribution below $10 \%$, but a core-shell phase structure and weak magnetic properties.
  The core is identified as w\"ustite phase with a \ch{Fe_{0.48}Co_{0.52}O} composition and the shell as inverse spinell with \ch{Co_{0.82}Fe_{2.18}O_4}, using a combined evaluation of EDX, XRD and SAS data.
  A weak spontaneous magnetization is observed by both VSM ($100 \unit{kA \, m^{-1}}$) and SANSPOL ($105 \unit{kA \, m^{-1}}$) at $1.2 \unit{T}$, where SANSPOL confirms that the magnetization is primarily determined by the nanocube shell.
  From temperature-dependent magnetization measurements a blocking temperature of $210 \unit{K}$ is determined for the nanocubes, and a hysteretic behaviour is observed at $10 \unit{K}$ with a coercive field in the order of $1 \unit{T}$.
  \\

  Pure phased nanocubes are obtained by performing a heating up synthesis starting from cobalt and iron acetylacetonates.
  Three batches of nanocubes are discussed: Ac-CoFe-C, Ac-CoFe-C-2 and Ac-CoFe-C-3, which have an average nanocube edge length of $10 - 12.5 \unit{nm}$ and a size distribution in the range of $10 - 14 \%$.
  The phase purity is shown for two nanoparticle batches by XRD, for which the cobalt to iron ratio is thus fully determined by EDX to \ch{Co_{0.66} Fe_{2.22} O_4} (Ac-CoFe-C) and \ch{Co_{0.80} Fe_{2.13} O_4} (Ac-CoFe-C-2).
  Using the determined phases, the small-angle scattering data of all batches is evaluated with a self-developed superball form factor that is a shape that can continuously vary by a single parameter between a sphere and a cube.
  The results are compared to the best result obtained by a sphere and cube form factor, where in both cases the superball form factor shows the closest agreement.
  For the two pure-phased nanocube batches, a characterization of the magnetic properties is furthermore performed by SANSPOL and VSM.
  The SANSPOL experiment results in a spontaneous magnetization of $334(5) \unit{kA \, m^{-1}}$ and $444(3) \unit{kA \, m^{-1}}$ for Ac-CoFe-C and Ac-CoFe-C-2 respectively, and a VSM analysis in $298(3) \unit{kA \, m^{-1}}$ and $413(12) \unit{kA \, m^{-1}}$.
  The complimentary experiments thus agree in the magnitude within $10 \%$ and both show the strong magnetic properties of the nanocubes prepared by the acetylacetonates method.
  Low-temperature measurements of the nanocubes in dispersion show a wide hysteresis with a coercive field in the order of $2 \unit{T}$ and from temperature dependent magnetization measurements of the nanocubes Ac-CoFe-C in a dry state, a blocking temperature above room temperature at $315 \unit{K}$ is determined.
  The magnetization of Ac-CoFe-C in dispersion and in a dry state are compared at high and low temperatures and the significant differences are discussed in terms of the magnetocrystalline anisotropy and interparticle interaction.
  \\

  Concluding, while the cubes from the oleate synthesis are homogeneous in shape and have a small size-distribution, a core-shell phase with weak magnetic properties is obtained.
  The cubes from the acetylacetonates synthesis have a pure phase with strong magnetic properties but a broader size distribution is the trade-off.
  In the following both type of nanoparticles will be used to study the properties of self-assembled long range ordered monolayers, where Ol-CoFe-C is used to study the parameters for preparing high quality monolayers, the nanocubes from acetylacetonates Ac-CoFe-C, Ac-CoFe-C-2 and Ac-CoFe-C-3 are used to study the collective magnetism of long range ordered monolayers.
\end{document}