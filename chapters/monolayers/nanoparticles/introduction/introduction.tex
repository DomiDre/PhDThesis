\providecommand{\main}{../../../..}
\documentclass[\main/dresen_thesis.tex]{subfiles}

\begin{document}
  Cobalt ferrite nanoparticles are nowadays routinely synthesized following methods such as co-precipitation \cite{Fried_2001_Order}, sol-gel \cite{Niederberger_2009_Metal}, micro emulsions \cite{Pillai_1996_Synth}, or thermal decomposition.
  For thermal decomposition, one can further differentiate between hot-injection \cite{Hyeon_2003_Chemi} and heating up methods \cite{Embden_2015_TheHe}.
  Where the first promises monodisperse nanoparticles by keeping the nucleation time period of the synthesis short, the latter is especially promising in being scaleable and highly controllable \cite{Park_2004_Ultra}.
  The heating up method is used extensively in this work to synthesize nanocubes and nanospheres of cobalt ferrite and iron oxide and is further elaborated in the following.

  A popular heating up route to synthesize nanoparticles is to first prepare a metal oleate precursor from metal salts, which is subsequently slowly heated above it's decomposition temperature in a high-boiling solvent, where it is aged in the presence of oleic acid and additional reagents to direct the growth and shape.
  For example, the shape of cobalt ferrite nanoparticles can be tuned by adding sodium oleate as precursor before the heating up process.
  The sodium oleate attaches to the (100) facets of forming nanocrystals and fosters the growth along the [111] direction of the crystal \cite{Bao_2009_Forma}.
  By tuning the ratio of sodium oleate to the oleate precursor and adjusting the aging time, this process allows to go from nanospheres over nanocubes to star-shaped nanoparticles.
  Wetterskog \etal extensively studied the formation of maghemite nanospheres and nanocubes, following the same route without cobalt oleate in the synthesis \cite{Wetterskog_2014_Preci, Wetterskog_2013_Anoma}, and found that due to the reducing environment a w\"ustite core is first formed and only by post-synthesis oxidation a maghemite shell is obtained.
  The same can be observed for the synthesis of cobalt ferrite nanoparticles from metal oleates \cite{Bodnarchuk_2009_Excha}, which explains in both cases the low degree of magnetism that is often observed in literature from such synthesis as w\"ustite is paramagnetic at room temperature and antiferromagnetic below $203 \unit{K}$.
  Through forced oxidation after the synthesis the magnetic properties can be enhanced especially for iron oxide nanoparticles \cite{Wetterskog_2013_Anoma}.
  However, cobalt ferrite provides a high oxygen diffusion barrier \cite{Chen_2015_Synth} and it is technically harder to oxidize cobalt ferrite nanoparticles completely without destroying the surfactant shell irreversibly.

  An alternative heating up synthesis that results in strongly magnetic nanoparticles with pure phase directly is achieved by the thermal decomposition of metal acetylacetonates in a high boiling solvent such as dibenzyl ether with the presence of oleic acid or oleylamine \cite{Sun_2002_SizeC, Wu_2014_Monol}.
  Again the addition of sodium oleate leads to the formation of nanocubes, such as in the oleate synthesis route, when the amount is tuned to the oleic acid content and aging time.
  However, this synthesis is in general difficult to control and scale partly due to the production of acetone during the synthesis that leads to small but violent explosions at elevated temperatures within the solution.
  Often the shape of the nanoparticles from this synthesis route is less uniform in comparison to the oleate synthesis route and a lot of fine-tuning and care has to be taken to obtain a homogeneous batch of nanoparticles that can be used for self-assembly experiments.
  
\end{document}