\providecommand{\main}{../../../..}
\documentclass[\main/dresen_thesis.tex]{subfiles}

\begin{document}
  \label{sec:monolayers:nanoparticle:edx}

  \begin{table}[ht]
    \centering
    \caption{\label{tab:monolayers:nanoparticles:edx}Ratio of cobalt and iron in the nanoparticles as determined by EDX. Additionally the formula unit deduced for the acetylacetonate particles from assuming total occupation of the inverse spinell (\ch{Co_x Fe_{3-x} O4}) and from the assumption of only \ch{Fe^{3+}} in the lattice (\ch{Co_x Fe_y O4}) are given. For Ol-CoFe-C the ratio is used to approximate the composition of the w\"ustite phase.}
    \begin{tabular}{ l | r | r | r }
      \textbf{EDX}                                & \textbf{Ol-CoFe-C} & \textbf{Ac-CoFe-C}          & \textbf{Ac-CoFe-C-2}\\
      \hline
      \rule{0pt}{2ex} $N_{\ch{Fe}} / N_{\ch{Co}}$ & 2.49(27)           & 3.35(10)                    & 2.67(6)\\
      \hline
      \hline
      \rule{0pt}{2ex}  \ch{Co_x Fe_{3-x} O4}    &  & \ch{Co_{0.69} Fe_{2.31} O4} & \ch{Co_{0.82} Fe_{2.18} O4}\\
      \rule{0pt}{2ex}  \ch{Co_x Fe_y O4}        &  & \ch{Co_{0.66} Fe_{2.22} O4} & \ch{Co_{0.80} Fe_{2.13} O4}\\
      \hline
    \end{tabular}
  \end{table}

  The evaluation of the spectra obtained by EDX on large agglomerations of the nanoparticle batches yields the relative compositon in iron and cobalt for the nanoparticle batches as tabulated in \reftab{tab:monolayers:nanoparticles:edx}.
  Ol-CoFe-C has a composition ratio $2.49(27)$, for a feed ratio of $2$.
  Ac-CoFe-C shows a ratio of iron to cobalt of $3.35(10)$, whereas Ac-Co-Fe-C-2 is given by $2.67(6)$.
  This can be compared to the feed ratio of $1.54$ used in Ac-CoFe-C and $1.74$ in Ac-CoFe-C-2.
  This is in contrast to the linear dependence of feed ratio and resulting composition ratio that is observed in literature \cite{Sathya_2016_Cofeo}.

  To explain this discrepancy, the differently chosen heating rates in both synthesis need to be considered.
  The reactants in the acetylacetonates synthesis decompose not exactly equally as can be seen by a thermogravimetric-analysis performed on the chemicals as shown in \reffig{fig:monolayers:nanoparticle:edx:TGARefluxAcAc}.
  The observed mass loss of iron acetylacetonate upon decomposition at elevated temperature is higher than the mass loss of cobalt acetylacetonate.
  Therefore, when given more time by a slow heating as was the case in the synthesis of Ac-CoFe-C ($5 \unit{K \, min^{-1}}$) in comparison to Ac-CoFe-C-2  ($15 \unit{K \, min^{-1}}$), the ratio of iron to cobalt monomers is in favor of iron at the growth stage.
  \begin{figure}[tb]
    \centering
    % \includegraphics{monolayers_EDX_TGA_OleateTGA}
    \includegraphics{monolayers_EDX_TGA_AcAcTGA}
    \caption{\label{fig:monolayers:nanoparticle:edx:TGARefluxAcAc}Decomposition  \ch{Fe(acac)_3} and \ch{Co(acac)_2} monitored by thermo-gravimetric analysis.}
    % of \ch{Fe(oleate)} and \ch{Co(oleate)} (left) and
  \end{figure}

  The ratio can be used to deduce the formula unit of the nanoparticles from acetylacetonate.
  The result for both assumptions, a complete occupation of the inverse spinell structure, as typically assumed in literature, and a partially vacant crystal structure with only \ch{Fe^{3+}} are listed in \reftab{tab:monolayers:nanoparticles:edx}.
  For the studied nanocubes, both assumptions result in a similar formula unit.
  For the continuing analysis of the nanoparticles, \ch{Co_x Fe_y O4} is considered as it is assumed that no \ch{Fe^{2+}} is present.
  Thus Ac-CoFe-C has the average formula unit \ch{Co_{0.66} Fe_{2.22} O4} and Ac-CoFe-C-2 \ch{Co_{0.80} Fe_{2.13} O4}.

  For Ol-CoFe-C, the information of the ratio of cobalt to iron in the nanoparticle is not sufficient to deduce the formula unit. Here, the core-shell nanoparticles have two phases that may contain cobalt and iron, namely \ch{Co_x Fe_{3-x} O4} and \ch{Fe_y Co_{1-y} O}
  The density of the acetylacetonate based particles is estimated using the determined composition and the lattice constant from XRD \refsec{sec:monolayers:nanoparticle:xrd} and given in \reftab{tab:monolayers:nanoparticles:density}.

  \begin{table}[ht]
    \centering
    \caption{\label{tab:monolayers:nanoparticles:density}Mass density $\rho$ of the two acetylacetonate based nanoparticle batches determined by the results from EDX and XRD.}
    \begin{tabular}{ l | r | r }
       Density                & \textbf{Ac-CoFe-C} & \textbf{Ac-CoFe-C-2}\\
      \hline
      \rule{0pt}{2ex} $\rho \, / \unit{g \, mL^{-1}}$  & 5.09               & 5.17\\
      \hline
    \end{tabular}
  \end{table}
  % & \textbf{Ol-CoFe-C}
  % & 6.42
\end{document}