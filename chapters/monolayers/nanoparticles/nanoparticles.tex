\providecommand{\main}{../../..}
\documentclass[\main/dresen_thesis.tex]{subfiles}
  \renewcommand{\thisPath}{\main/chapters/monolayers/nanoparticles/}
\begin{document}
  \subfile{\thisPath/introduction/introduction}

  \subsection{Structural \& Magnetic Characterization of the Nanocubes}
    \subfile{\thisPath/discussion/discussion}
      \FloatBarrier

  In summary, the two heating up syntheses of cobalt ferrite nanocubes that are presented here provide a trade-off in homogeneous shape vs. phase purity and strong magnetism.
  Both syntheses yield cubic particles in the size of approximately $10 \unit{nm}$.
  While the cubes from the oleate synthesis are more homogeneous in shape but has a core-shell phase with weak magnetic properties, the cubes from the acetylacetonates synthesis have a pure phase with stronger magnetic properties but a broader size distribution.
  This has been shown by using complementary experimental techniques that give a mutual and reliable picture of the global size properties.
  In the following both batches will be used to study the properties of self-assembled long range ordered monolayers, where Ol-CoFe-C is used to study the parameters for preparing high quality monolayers, Ac-CoFe-C is used to study the collective magnetism of long range ordered monolayers.
\end{document}