\providecommand{\main}{../../..}
\documentclass[\main/dresen_thesis.tex]{subfiles}

\begin{document}
  Maghemite ($\gamma-$\ch{Fe2O3}) and cobalt ferrite (\ch{CoFe2O4}) nanoparticles are synthesized following methods such as co-precipitation \cite{Fried_2001_Order}, sol-gel \cite{Niederberger_2009_Metal}, micro emulsions \cite{Pillai_1996_Synth}, or thermal decomposition.
  For thermal decomposition, one can further differentiate between hot-injection \cite{Hyeon_2003_Chemi} and heating up methods \cite{Embden_2015_TheHe}.
  The latter is especially promising in being scalable, highly controllable and yielding small particle size distributions.
  It is used extensively in this work to synthesize nanocubes and nanospheres of cobalt ferrite and iron oxide and is further elaborated in the following.

  A popular heating up route to synthesize nanoparticles is to first prepare a metal oleate precursor from the metal salts, which is subsequently slowly heated above it's decomposition temperature in a high-boiling solvent such as 1-octadecene and aged here for about half an hour in the presence of oleic acid.
  The shape of the nanoparticle is tuned by adding sodium oleate as precursor before the heating up process, which attaches to the (100) facets of forming nanocrystals and fosters the growth along the [111] direction of the crystal \cite{Bao_2009_Forma}.
  By tuning the ratio of sodium oleate to the oleate precursor and adjusting the aging time, this process allows to go from nanospheres over nanocubes to star-shaped nanoparticles.
  Wetterskog \etal extensively studied the formation of maghemite nanospheres and nanocubes during this process \cite{Wetterskog_2014_Preci, Wetterskog_2013_Anoma} and found that due to the reducing environment an antiferromagnetic wustite core is first formed and only by oxidation a maghemite shell is obtained.
  The same can be observed for the synthesis of cobalt ferrite nanoparticles from metal oleates \cite{Bodnarchuk_2009_Excha}, which explains in both cases the low degree of magnetism that is often observed in literature from such synthesis.
  However, cobalt ferrite provides a better oxygen diffusion barrier \cite{Chen_2015_Synth} and it is in generally harder to oxidize cobalt ferrite nanoparticles completely.

  An alternative heating up synthesis that results strongly magnetic nanoparticles with pure phase is by thermal decomposition of metal acetylacetonates in a high boiling solvent such as dibenzyl ether in the presence of oleic acid or oleylamine \cite{Sun_2002_SizeC, Wu_2014_Monol}.
  Again the addition of sodium oleate leads to the formation of nanocubes such as in the oleate synthesis route.
  However, this synthesis is in general harder to control due to the production of acetone during the synthesis that leads to small but violent explosions at elevated temperatures in the reaction.
  Therefore, the shape of the nanoparticles is less uniform in comparison to the oleate synthesis route and a lot of fine-tuning and care has to be taken to obtain a good batch of nanoparticles.

  In the following the preparation of cobalt ferrite nanoparticles following both routes is presented and the obtained nanoparticles are characterized similar to \refch{ch:looselyPackedNS}. The synthesis and characterization of iron oxide nanocubes is presented in \refch{ch:colloidalCrystals}.

\end{document}