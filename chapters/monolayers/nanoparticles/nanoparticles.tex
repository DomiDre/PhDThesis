\providecommand{\main}{../../..}
\documentclass[\main/dresen_thesis.tex]{subfiles}

\begin{document}
  Cobalt ferrite nanoparticles are nowadays routinely synthesized following methods such as co-precipitation \cite{Fried_2001_Order}, sol-gel \cite{Niederberger_2009_Metal}, micro emulsions \cite{Pillai_1996_Synth}, or thermal decomposition.
  For thermal decomposition, one can further differentiate between hot-injection \cite{Hyeon_2003_Chemi} and heating up methods \cite{Embden_2015_TheHe}.
  Where the first promises monodisperse nanoparticles by keeping the nucleation time period of the synthesis short, the latter is especially promising in being scaleable and highly controllable \cite{Park_2004_Ultra}.
  The heating up method is used extensively in this work to synthesize nanocubes and nanospheres of cobalt ferrite and iron oxide and is further elaborated in the following.

  A popular heating up route to synthesize nanoparticles is to first prepare a metal oleate precursor from metal salts, which is subsequently slowly heated above it's decomposition temperature in a high-boiling solvent, where it is aged in the presence of oleic acid and possibly additional reagents to direct the growth and shape.
  For example, the shape of cobalt ferrite nanoparticles can be tuned by adding sodium oleate as precursor before the heating up process.
  The sodium oleate attaches to the (100) facets of forming nanocrystals and fosters the growth along the [111] direction of the crystal \cite{Bao_2009_Forma}.
  By tuning the ratio of sodium oleate to the oleate precursor and adjusting the aging time, this process allows to go from nanospheres over nanocubes to star-shaped nanoparticles.
  Wetterskog \etal extensively studied the formation of maghemite nanospheres and nanocubes, following the same route without cobalt oleate in the synthesis \cite{Wetterskog_2014_Preci, Wetterskog_2013_Anoma}, and found that due to the reducing environment a w\"ustite core is first formed and only by post-synthesis oxidation a maghemite shell is obtained.
  The same can be observed for the synthesis of cobalt ferrite nanoparticles from metal oleates \cite{Bodnarchuk_2009_Excha}, which explains in both cases the low degree of magnetism that is often observed in literature from such synthesis as w\"ustite is paramagnetic at room temperature.
  Through forced oxidation after the synthesis the magnetic properties can be enhanced especially for iron oxide nanoparticles \cite{Wetterskog_2013_Anoma}.
  However, cobalt ferrite provides a high oxygen diffusion barrier \cite{Chen_2015_Synth} and it is technically harder to oxidize cobalt ferrite nanoparticles completely.

  An alternative heating up synthesis that results in strongly magnetic nanoparticles with pure phase is achieved by the thermal decomposition of metal acetylacetonates in a high boiling solvent such as dibenzyl ether with the presence of oleic acid or oleylamine \cite{Sun_2002_SizeC, Wu_2014_Monol}.
  Again the addition of sodium oleate leads to the formation of nanocubes such as in the oleate synthesis route.
  However, this synthesis is in general difficult to control and scale due to the production of acetone during the synthesis that leads to small but violent explosions at elevated temperatures within the reaction.
  Often the shape of the nanoparticles from this synthesis route is less uniform in comparison to the oleate synthesis route and a lot of fine-tuning and care has to be taken to obtain a good batch of nanoparticles.

  In the following, the preparation of cobalt ferrite nanocubes following both routes is presented and the obtained nanoparticles are characterized similar to \refch{ch:looselyPackedNS}.
  The synthesis and characterization of iron oxide nanocubes following the oleate route is presented in \refch{ch:colloidalCrystals}.

  \subsection{Synthesis from Metal Oleates and Acetylacetonates}\label{sec:monolayers:nanoparticle:synthesisOleatesAcAc}
    \subsubsection{Preparation of Cobalt Ferrite Oleate}
      In the first presented protocol to synthesize cobalt ferrite nanoparticles, a cobalt and iron oleate mixture is prepared as first step.
      For this purpose a clear solution of sodium oleate is prepared by dissolving $96 \unit{mmol}$ of \ch{NaOH} in $20 \unit{mL}$ of each \ch{H2O} and \ch{EtOH} and subsequently adding drop wise $96 \unit{mmol}$ of oleic acid under constant stirring.
      Then $12 \unit{mmol}$ of \ch{CoCl2 * 6 H2O} and $24 \unit{mmol}$ of \ch{FeCl3 * 6 H2O} are dissolved in $5 \unit{mL}$ \ch{H2O} and $15 \unit{mL}$ \ch{EtOH} each and added to the solution.
      After addition of $80 \unit{mL}$ \ch{H2O} and \ch{EtOH} each, as well as $160 \unit{mL}$ n-hexane, the mixture is held at reflux ($60 \unit{^\circ C}$) for $4 \unit{h}$ under constant strong magnetic stirring.
      Once the mixture is cooled back to room temperature, it is washed three times in a separatory funnel with $30 \unit{mL}$ \ch{H2O} each to remove superfluous \ch{NaCl}.
      The remaining n-hexane, ethanol and water is removed using a rotary evaporator.
      In the end, approximately $32 \unit{g}$ of a dark red and highly viscous metal oleate complex is obtained, which is then ready to be used for the nanoparticle synthesis.

    \subsubsection{Preparation of Nanocubes from Oleate}
      To prepare nanoparticles, $10 \unit{mmol}$ of the cobalt ferrite oleate is dissolved in $50 \unit{mL}$ 1-octadecene within a $250 \unit{mL}$ three-neck round-bottom flask.
      To obtain cubically shaped nanoparticles, sodium oleate is prepared separately by dissolving $2.5 \unit{mmol}$ \ch{NaOH} in $10$ drops of \ch{H2O} and \ch{EtOH} and adding $2.5 \unit{mmol}$ of oleic acid drop wise while ultra sonificating the mixture.
      The sodium oleate is added to the dissolved oleate together with additional $2.5 \unit{mmol}$ oleic acid.
      The mixture is heated and held at $150 \unit{^\circ C}$ for one hour under constant magnetic stirring until all water and ethanol is evaporated.
      A fractionating column is put on the round-bottom flask and nitrogen is gently bubbled into the mixture.
      Using a temperature controller, the mixture is heated to reflux at approximately $315 \unit{^\circ C}$ with a gradient of $2.5 \unit{^\circ C min^{-1}}$, where it is held for $30 \unit{min}$.
      After cooling the reaction naturally to room temperature, the particles are precipitated with \ch{EtOAc} and \ch{EtOH}, centrifuged at $8000 \unit{rpm}$ and redispersed in n-hexane until the supernatant is clear.
      In the last step the mixture is centrifuged without adding \ch{EtOAc}/\ch{EtOH} and the supernatant fluid is taken as dispersion, where as the precipitate is thrown away as being unstable.
      This synthesis yields approximately $500 \unit{mg}$ nanocubes (yield $\approx 20 \%$) and is referred to in the following as Ol-CoFe-C.


    \subsubsection{Preparation of Nanocubes from Acetylacetonates}
      To synthesize nanocubes from acetylacetonates, $0.52 \unit{mmol}$ of \ch{Co(acac)2}, $0.8 \unit{mmol}$ of \ch{Fe(acac)3}, $3 \unit{mmol}$ of freshly prepared sodium oleate and $3 \unit{mmol}$ of oleic acid are diluted in $10 \unit{mL}$ of dibenzyl ether in a $50 \unit{mL}$ three-neck round-bottom flask.
      The mixture is heated to $120 \unit{^\circ C}$ and held here for $1 \unit{h}$.
      After putting a fractionating column on the round-bottom flask, a temperature controller is used to heat the mixture to reflux at about $290 \unit{^\circ C}$ with a heating rate of $5 \unit{^\circ C min^{-1}}$.
      During the synthesis, nitrogen is blown gently over the solution all the time to form an inert blanket and the solution is magnetically stirred.
      After cool down, the product is transferred with n-hexane to centrifugal tubes and precipitated with \ch{EtOH}.
      After centrifugation at $8500 \unit{rpm}$ the supernatant is thrown away and the precipitate is redispersed with n-hexane.
      This procedure is repeated three times and in the last step the precipitate is dispersed in n-hexane without precipitating it again.
      After a final centrifugation at $8500 \unit{rpm}$ the supernatant is kept as dispersion and the rest thrown away.
      The synthesis yields approximately $50 \unit{mg}$ nanocubes (yield $\approx 45 \%$) and is referred to in the following as Ac-CoFe-C.

  \subsection{Structural Characterization of the Nanocubes}
  \label{sec:monolayers:nanoparticle:structuralCharacterization}
    After synthesis, transmission electron microscopy (TEM) is used to characterize a sample of the prepared dispersion and validate the structural quality of the batch.
    The exemplary images shown in \reffig{fig:monolayers:nanoparticle:tem} reveal that both procedures result in nanocubes.
    The average particle size and size distribution of the particles is estimated from TEM by fitting the log-normal distribution described in \refeq{eq:looselyPackedNP:nanoparticle:lognormalDist}.
    The particles from the metal oleate (Ol-CoFe-C) are qualitatively more homogeneous in shape than the particles from acetylacetonates (Ac-CoFe-C) and have an average edge length of $10.90(4) \unit{nm}$ with size distribution $8.8(3) \unit{\%}$.
    Within Ac-CoFe-C, the size distribution appears to be larger and non-cubic particles can be found in between the cubes, their average edge length is estimated to $10.1(1) \unit{nm}$ with a size distribution of $13.9(9) \unit{\%}$.

    \begin{figure}[tb]
      \centering
      \hspace{0.3 cm}
      \includegraphics{monolayers_TEM_Ol_CoFe_C}
      \hspace{0.3 cm}
      \includegraphics{monolayers_TEM_Ac_CoFe_C}
      \includegraphics{monolayers_TEM_Ol_CoFe_C_sizeDist}
      \includegraphics{monolayers_TEM_Ac_CoFe_C_sizeDist}
      \caption{\label{fig:monolayers:nanoparticle:tem}Transmission electron microscopy images of cobalt ferrite nanocubes synthesized from oleates (upper left) and acetylacetonates (upper right) and the edge length size distribution evaluated from the images below respectively.}
    \end{figure}

    Using small-angle x-ray and neutron scattering, the nuclear and magnetic structure of diluted nanoparticles in dispersion is quantitatively evaluated over a macroscopic area of the sample and shown in \reffig{fig:monolayers:nanoparticle:sas:AcOlCoFeC}.
    For the SAXS measurement, the samples were diluted in n-hexane and measured at the GALAXI instrument in the \textsc{Forschungszentrum J\"ulich}, which is described in \refapp{ch:appendix:lss:galaxi}.
    The small-angle neutron scattering data was measured using the D22 and D33 instrument at the Institut Laue-Langevin, which are described in \refapp{ch:appendix:lss:d22} and \refapp{ch:appendix:lss:d33} respectively.
    For the neutron scattering experiment, the sample was dispersed in deuterated toluene (Toluene-d8) to reduce the incoherent scattering coming from hydrogen atoms.
    The comparison between the Ol-CoFe-C and Ac-CoFe-C small-angle scattering data shows on a qualitative level that the nanocubes synthesized following the oleate route have a smaller size distribution and higher homogeneous sample quality as more oscillations are visible with more pronounced minima, confirming the result observed from TEM.
    However, it also shows that Ol-CoFe-C is only weakly magnetic whereas the particles in Ac-CoFe-C have a stronger magnetic moment, which is visible in the greater splitting of $I(+)$ and $I(-)$ in SANSPOL.

    \begin{figure}[tb]
      \centering
      \includegraphics{monolayers_SAS_Ol_CoFe_C_SASFit}
      \includegraphics{monolayers_SAS_Ol_CoFe_C_SANSPOLFit}
      \includegraphics{monolayers_SAS_Ac_CoFe_C_SASFit}
      \includegraphics{monolayers_SAS_Ac_CoFe_C_SANSPOLFit}
      \caption{\label{fig:monolayers:nanoparticle:sas:AcOlCoFeC}SAXS and SANS measurement of Ol-CoFe-C (upper left) and Ac-CoFe-C (lower left), as well as SANSPOL at $1.2 \unit{T}$ respectively (right). The data is fit to a superball model.}
    \end{figure}

    To evaluate the data quantitatively, the data of both samples is used to fit the parameters of a superball model.
    Briefly, a superball is a mathematical shape that can be used to describe rounded cubes and it's volume is defined by
    \begin{align}
      x^{2p} + y^{2p} + z^{2p} < R^{2p},
    \end{align}
    where $R$ is the radius of the superball and $p$ describes whether the body is closer to a sphere or a cube.
    For $p\eq 1$ the superball is equivalent to the definition of a sphere and for $p \rightarrow \infty$ the superball converges to a cube with edge length $a \eq 2R$.
    The full description and derivation of the superball formfactor is described in \refapp{ch:appendix:numericalMethods:superballFormfactor}.
    To account for the oleic acid surfactant of the nanoparticles, the superball volume is assumed to have a core-shell structure, where the shell is variable in scattering length density and thickness.
    The fitted model is shown in \reffig{fig:monolayers:nanoparticle:sas:AcOlCoFeC} and the important parameter of interest are given in \ref{tab:monolayers:nanoparticle:sas} for the following discussion, whereas the full set of model parameters is listed in \refapp{ch:appendix:modelparameters:monolayers:sas_olac_cofe_c}.
    \begin{table}[ht]
      \centering
      \caption{\label{tab:monolayers:nanoparticle:sas}Relevant parameters of the superball fit of the small-angle scattering data of the presented nanocubes, the complete set used to describe the models are found in \refapp{ch:appendix:modelparameters:monolayers:sas_olac_cofe_c}.}
      \begin{tabular}{ c | l | l }
          & Ol-CoFe-C & Ac-CoFe-C \\
        \hline
        $R$
          & $5.62(2) \unit{nm}$
          & $4.69(1) \unit{nm}$\\
        $\sigma_R$
          & $9.27(8) \,\%$
          & $11.9(1) \,\%$\\
        $D$
          & $1.633(8) \unit{nm}$
          & $1.16(4) \unit{nm}$\\
        $p$
          & $1.54(3) \unit{nm}$
          & $2.66(9) \unit{nm}$\\
        $\rho_\mathrm{mag}^\mathrm{sans}$
          & $0.268(8) \cdot 10^{-6} \angstrom^{-2}$
          & $0.423(7) \cdot 10^{-6} \angstrom^{-2}$\\
        \hline
        $V_p$
          & $1030(10) \unit{nm^3}$
          & $726(5) \unit{nm}$\\
        $M^\mathrm{sans}$
          & $92(3) \unit{kAm^{-1}}$
          & $145(2) \unit{kAm^{-1}}$\\
        \hline
      \end{tabular}
    \end{table}
    The superball model result confirms the qualitative observation that the Ol-CoFe-C nanoparticles have a smaller size distribution than the Ac-CoFe-C cubes.
    Where the specimen size used to determine size and size distributions in transmission electron microscopy leave the option for a biased result due to possibly missing parts of the dispersion by only measuring a small selected amount of nanoparticles, the large number of particles scanned in the small-angle scattering experiments reveals without bias the average particle size and variation.
    Furthermore, the model gives that on average Ol-CoFe-C has a higher degree of roundness on the corners, visible from the smaller $p$ parameter in comparison to Ac-CoFe-C.
    To see this in imaging experiments, high resolution transmission electron microscopy (HR-TEM) experiments are necessary.
    A study comparing the effectiveness of the superball model in comparison to HR-TEM can be found in \refapp{app:structureCoFe2O4Nanocubes} for three particle batches synthesized similar to Ac-CoFe-C (with a lower heating rate and varied cobalt content, but otherwise same synthesis parameters).

    \begin{figure}[tb]
      \centering
      \includegraphics{monolayer_VSM_Ol_CoFe_C}
      \includegraphics{monolayer_VSM_Ac_CoFe_C}
      \caption{\label{fig:monolayers:nanoparticle:vsm}Room-temperature hysteresis measurement of Ol-CoFe-C (upper left) and Ac-CoFe-C (lower left) taken on a vibrating sample magnetometer fitted with a Langevin curve.}
    \end{figure}
    The surfactant shell thickness are in the range of $1 - 2 \unit{nm}$, which is expected for oleic acid chains.
    As the chains mix with the solvent on the way out, it is expected that the average scattering length density of the shell is not exactly that of bulk oleic acid ($7.8 \cdot 10^{-8} \angstrom^{-2}$), but a mixture with the solvent SLD, which correlates with the model thickness and might explain the reduced value in Ac-CoFe-C.
    The exact thickness, density and shape of the shell is not of detailed interest for the study of the magnetic properties of the nanocubes and therefore in the scope of this work this rough estimate of the shell is good enough for the discussion.

    The magnetic scattering length density is determined from the SANSPOL data, which is used to determine the magnetization of the nanoparticle using \refeq{eq:looselyPackedNP:nanoparticles:SLDtoMagnetization} to $92(3) \unit{kAm^{-1}}$ for Ol-CoFe-C and $145(2) \unit{kAm^{-1}}$ for Ac-CoFe-C.
    This can be compared to the saturation magnetization obtained from measuring a liquid dispersions of nanoparticles on a vibrating sample magnetometer (VSM) as shown in \reffig{fig:monolayers:nanoparticle:vsm}, which results in magnetizations of $70 \unit{kAm^{-1}}$ for Ol-CoFe-C and $275 \unit{kAm^{-1}}$ for Ac-CoFe-C at the same magnetic field of $1.2 \unit{T}$.
    The result obtained for Ol-CoFe-C appears reasonable and in agreement for both experiments considering that measuring liquid samples in vibrating sample magnetometry can have larger systematic errors due to the movement of the liquid.
    For Ac-CoFe-C, the results are off, however, by nearly a factor of two.
    But importantly, on a qualitative level, both experiments agree in the result that the nanoparticles from the oleate route are weakly magnetic, while the particles from the acetylacetonates are stronger magnetic.

    \begin{figure}[tb]
      \centering
      \includegraphics{monolayers_SAS_Ac_CoFe_C_SASSphereModelFit}
      \includegraphics{monolayers_SAS_Ac_CoFe_C_SANSPOLSphereModelFit}
      \includegraphics{monolayers_SAS_Ac_CoFe_C_SASCubeModelFit}
      \includegraphics{monolayers_SAS_Ac_CoFe_C_SANSPOLCubeModelFit}
      \caption{\label{fig:monolayers:nanoparticle:sas:SphereCubeFit}Best sphere and cube model fit to the same SAS data of Ac-CoFe-C fitted in \reffig{fig:monolayers:nanoparticle:sas:AcOlCoFeC}.}
    \end{figure}

    To understand this quantitative discrepancy, it's worthwhile to study the impact of the chosen shape model of the nanoparticle on the result of the magnetic scattering density.
    Where the  superball model appears seems to be reasonable from a physical point of view, the simpler models of a perfect sphere and cube are also fit to the date of both samples for comparison.
    These model provide a good reference point for the parameter values and as the models have one less parameter, they are additionally less prone to systematic errors in the fitting routine due to parameter correlations.
    The result of the cubic and spherical fit are shown in \reffig{fig:monolayers:nanoparticle:sas:SphereCubeFit} with the relevant parameters listed in \reftab{tab:monolayers:nanoparticle:sasSphereCubeFit}.

    \begin{table}[ht]
      \centering
      \caption{\label{tab:monolayers:nanoparticle:sasSphereCubeFit}Relevant parameters of the sphere and cube fit to the small-angle scattering data of Ac-CoFe-C, the complete set of parameters is found in \refapp{ch:appendix:modelparameters:monolayers:sas_olac_cofe_c}.}
      \begin{tabular}{ c | l | l }
          & Sphere & Cube \\
        \hline
        $R, \, a$
          & $5.56(2) \unit{nm}$
          & $9.01(2) \unit{nm}$\\
        $\sigma_R, \, \sigma_a$
          & $13.0(3) \,\%$
          & $10.7(2) \,\%$\\
        $D$
          & $1.46(4) \unit{nm}$
          & $1.03(4) \unit{nm}$\\
        $\rho_\mathrm{mag}^\mathrm{sans}$
          & $0.596(8) \cdot 10^{-6} \angstrom^{-2}$
          & $0.429(7) \cdot 10^{-6} \angstrom^{-2}$\\
        \hline
        $V_p$
          & $720(4) \unit{nm^{3}}$
          & $731(3) \unit{nm^{3}}$\\
        $M^\mathrm{sans}$
          & $205(3) \unit{kAm^{-1}}$
          & $147(2) \unit{kAm^{-1}}$\\
        \hline
      \end{tabular}
    \end{table}

    It's clearly visible by looking at the SAXS data that the spherical model underestimates the intensity of the first order peak, while the cube model overestimates it, whereas the superball was able to adjust in between.
    The determined particle volume on the other hand varies only weakly from the specific choice of the different model.
    From the SANSPOL data, it is visible that parameters such as the shell thickness, particle magnetization, as well as the particle concentration and the estimated nuclear scattering length densities (listed in Appendix), are strongly affected by the choice of model in the fitting process.
    This becomes especially visible in that the cube model results in a smaller shell thickness and magnetization, whereas the sphere model suggests values that are in the order of $30 - 50 \unit{\%}$ higher.

    Conclusively, this comparison helps to estimate the parameter range and variation that has to be accepted without restricting specific parameters or forcing certain parameters by including results from complementary experiments.

    \begin{figure}[tb]
      \centering
      \includegraphics{monolayer_VSM_10K_Ac_CoFe_C}
      \includegraphics{monolayer_VSM_10K_Ac_CoFe_C}
      \caption{\label{fig:monolaye rs:nanoparticle:vsm10K}Low temperature hysteresis measurement of frozen Ol-CoFe-C (upper left) and Ac-CoFe-C (lower left) using the same samples as in \reffig{fig:monolayers:nanoparticle:vsm}.}
    \end{figure}

    To study the samples further, the dispersions are measured in VSM at a low temperature of $10 \unit{K}$, where they are frozen and the nanoparticles can be considered as disordered and additionally both the Brown and N\'eel relaxation of the superspin moment are suppressed.
    At this temperature the blocked single-domain particle is thus observed, assuming that the nanoparticles are non-interacting.

    % include PPMS DD67 & discussion


    \begin{figure}[tb]
      \centering
      \includegraphics{monolayer_XRD_Ol_CoFe_C}
      \includegraphics{monolayer_XRD_CoFe2O4WustiteFit_Ol_CoFe_C}
      \includegraphics{monolayer_XRD_Ac_CoFe_C}
      \caption{\label{fig:monolayers:nanoparticle:xrd}X-ray diffraction of Ol-CoFe-C (upper left) and Ac-CoFe-C (lower left) with a Rietveld refinement assuming an inverse spinell structure. Additionally, Ol-CoFe-C was refined assuming a combination of an inverse spinell phase and a w\"ustite phase (upper right).}
    \end{figure}

    An explanation for the especially weak magnetic properties of the oleate synthesized particles is found in literature and confirmed when looking at X-ray diffraction (XRD) data shown in \reffig{fig:monolayers:nanoparticle:xrd}.
    The XRD data was measured in cooperation with the group of Daniel Nižňanský from the Department of Inorganic Chemistry at the Charles University in Prague, where the experimental details are found in \refapp{app:additionalExperimentalTechniques:xrd}.
    For both diffraction data sets a Rietveld analysis was performed, where the expected inverse spinell structure of cobalt ferrite (space group Fd-3m) was fitted.
    The order of adding parameters to the refinement was in all cases the same, where first global parameters as the scale factor and background are estimated, and then the lattice parameter, peak shape and temperature displacement parameters.
    The detailed list of all parameters can be found in \refapp{ch:appendix:modelparameters:monolayers:xrd_olac_cofe_c} and the most relevant for the discussion in the following in \reftab{tab:monolayers:nanoparticle:xrd}.
    The particles from acetylacetonates are reasonably well fitted by the structure model of \ch{CoFe2O4} with a $\chi^2 \eq 2.0$.
    On the other hand, Ol-CoFe-C can not be fitted well by assuming a single phase model only as visible in the upper left figure of \reffig{fig:monolayers:nanoparticle:xrd}.
    The relative intensity of the peaks do not match well and deviate especially around $q \eq 3 \unit{\angstrom^{-1}}$.
    Earlier works on nanoparticles synthesized from oleates \cite{Bodnarchuk_2009_Excha, Wetterskog_2013_Anoma} report already that instead of pure phased cobalt ferrite particles, core-shell particles with an w\"ustite core are obtained during the oleate synthesis due to the reductive environment of the synthesis.
    And indeed, including a w\"ustite phase in the Rietveld analysis greatly improves the model as shown in the upper right image of \reffig{fig:monolayers:nanoparticle:xrd} and reduce the $\chi^2 \eq 13.1$ from only the single phase fit to $\chi^2 \eq 3.6$.
    The shown structure models neglect the cubic shape of the particles and possibly preferred orientation in the sample.
    Also some temperature displacement parameters have not been fitted in the analysis but kept at $0$, as they tend to diverge to nonphysical negative values, when they are included in the analysis, which suggests that even higher quality diffraction data is necessary to resolve them properly.

    Additionally to providing information about the phases of the nanoparticles, the Rietveld analysis provides information about the average size of the coherently scattering crystallites in the sample via the Scherrer equation
    \begin{equation}
      \tau \eq \frac{K \lambda}{\beta \cos(\theta)},
    \end{equation}
    with 

    \begin{table}[h]
      \centering
      \caption{\label{tab:monolayers:nanoparticle:xrd}Most relevant parameters of the XRD Rietveld analysis of Ol-CoFe-C and Ac-CoFe-C.}
      \begin{tabular}{ c | l | l }
          & Ol-CoFe-C & Ac-CoFe-C \\
        \hline
        $R$
          & $5.62(2) \unit{nm}$
          & $4.29(4) \unit{nm}$\\
        $\sigma_R$
          & $9.27(8) \,\%$
          & $11.9(1) \,\%$\\
        $D$
          & $1.633(8) \unit{nm}$
          & $1.16(4) \unit{nm}$\\
        $p$
          & $1.54(3) \unit{nm}$
          & $2.66(9) \unit{nm}$\\
        $\rho_\mathrm{mag}^\mathrm{sans}$
          & $0.268(8) \cdot 10^{-6} \angstrom^{-2}$
          & $0.423(7) \cdot 10^{-6} \angstrom^{-2}$\\
        \hline
        $n^\mathrm{saxs}$
          & $0.02637(6) \cdot 10^{-8} \angstrom^{-3}$
          & $0.0259(3) \cdot 10^{-8} \angstrom^{-3}$\\
        $n^\mathrm{sans}$
          & $0.198(1) \cdot 10^{-8} \angstrom^{-3}$
          & $0.20(1) \cdot 10^{-8} \angstrom^{-3}$\\
        $\rho_\mathrm{core}^\mathrm{sans}$
          & $6.397(1) \cdot 10^{-6} \angstrom^{-2}$
          & $6.631(1) \cdot 10^{-6} \angstrom^{-2}$\\
        $\Delta \theta_\mathrm{s. a.}$
          & $0.00090(8)$
          & $0.00315(9)$\\
        $\Delta \theta_\mathrm{l. a.}$
          & $0.00142(3)$
          & $0.0038(1)$\\
        $bg^\mathrm{saxs}$
          & $0.00114 \unit{cm}^{-1}$
          & $0.0 \unit{cm}^{-1}$\\
        $bg^\mathrm{sans}$
          & $0.0158(2) \unit{cm}^{-1}$
          & $0.0041(3) \unit{cm}^{-1}$\\
        $\rho_\mathrm{shell}^\mathrm{sans}$
          & $0.47(1) \cdot 10^{-6} \angstrom^{-2}$
          & $1.39(1) \cdot 10^{-6} \angstrom^{-2}$\\
        \hline
        $\rho_\mathrm{core}^\mathrm{saxs}$
          & \multicolumn{2}{c}{$41.749 \cdot 10^{-6} \angstrom^{-2}$}\\
        $\rho_\mathrm{solvent}^\mathrm{saxs}$
          & \multicolumn{2}{c}{$6.461 \cdot 10^{-6} \angstrom^{-2}$}\\
        $\rho_\mathrm{solvent}^\mathrm{sans}$
          & \multicolumn{2}{c}{$5.664 \cdot 10^{-6} \angstrom^{-2}$}\\
        $\lambda^\mathrm{sans}$
          & \multicolumn{2}{c}{$5.9984 \unit{\angstrom}$}\\
        $\Delta \lambda / \lambda ^\mathrm{sans}$
          & \multicolumn{2}{c}{$4.247 \, \%$}\\
        \hline
      \end{tabular}
    \end{table}
    %weiter XRD diskutieren

    %zusammenfassend Kapitel abrunden
\end{document}