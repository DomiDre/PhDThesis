\providecommand{\main}{../../..}
\documentclass[\main/dresen_thesis.tex]{subfiles}
  \renewcommand{\thisPath}{\main/chapters/monolayers/intro}

\begin{document}
  A primary goal of this work is the study of magnetic dipolar interactions between nanoparticles.
  Due to the highly directional nature of dipolar magnetism, random orientation of the nanoparticles and random packing, as described in the previous chapter, cloud the observation of macroscopic effects arising from the interaction.
  Therefore an intermediate goal towards the observation of dipolar interaction is the successful preparation of highly ordered nanostructures.
  Simple arrangements that have been discussed extensively in theory, are the square lattice and hexagonal lattice in the plane.
  Here, dipolar coupling leads to an emergent macroscopic states for the nanostructure, namely a super antiferromagnetic state for the square lattice and a super ferromagnetic state for the hexagonal lattice \cite{Politi_2002_Dipol, Russier_2001_Calcu, Varon_2013_Dipol}.
  In this work, a procedure to fabricate such long range ordered nanostructures from nanocubes and nanospheres has been developed.

  Multiple methods are known in literature to prepare monolayers on a substrate starting from a nanoparticle dispersion.
  Among the most prominent techniques are the Langmuir-Blodgett/Schaefer method \cite{Ukleev_2017_Selfa, Pauly_2011_Monol, Fried_2001_Order}, spin-coating \cite{Mishra_2012_Selfa}, doctor blade casting \cite{Bodnarchuk_2010_Excha}, dip coating \cite{Kim_2002_Multi} and drop casting \cite{Bigioni_2006_Kinet}.
  Depending on the applied technique, the resulting order of the nanoparticles can greatly vary.
  While methods as spin-coating and dip-coating are very quick to perform, the in-plane order of the particles is comparably poor to results obtained from evaporation-driven processes that allow for a slow self-assembly process of the particles.

  The Langmuir-Blodgett/Schaefer technique is a method that was designed for the formation of monolayers of organic molecules, but has been transferred to the study of nanoparticle monolayer formation \cite{Heitsch_2010_Gisax, Vorobiev_2015_Subst}.
  Here, the dispersion is allowed to evaporate slowly from an organic solvent on a water surface, where it then is compressed by movable beams on the surface to monolayer density.
  The Langmuir-Blodgett method then proceeds to transfer the particles from the water surface on a wafer by lifting a previously inserted substrate vertically out of the water.
  Whereas the Langmuir-Schaefer method use a wafer with hydrophobic coating as a stamp on the particles to transfer the layer.
  In-situ XRR and GISAXS studies on the water surface show that a high degree of in-plane order on the surface is possible for $10 \unit{nm}$ sized nanoparticles \cite{Vorobiev_2015_Subst}.
  The final experimental challenge is then to transfer the ordered particles from the surface to the wafer without destroying the order.
  A study of Wen \etal \cite{Wen_2011_Ultral} shows that by modification of the Langmuir through, where additionally the solvent evaporation is controlled, micrometer sized ordered arrays can be obtained by the Langmuir-Schaefer method.

  In the drop-casting method, the nanoparticles are transferred to a substrate by spreading a drop of dispersion on a flat surface and allowing the solvent to evaporate slowly, while the nanoparticles self-assemble.
  It is therefore an instrumentally cheap and very material efficient method, where the exact amount of nanoparticles needed for a monolayer on a given area can be set.
  Additionally, once a monolayer has successfully been formed, the drop casting method is easily reusable to prepare multi layered samples, by repeating the process sequentially.
  The challenge of the drop casting method is to control the self-assembly process by manipulating the environmental conditions during the evaporation process and choosing the correct solvent for the nanoparticles.
  Bigioni \etal \cite{Bigioni_2006_Kinet} showed for dodecanethiol-ligated gold nanoparticles that a combination of toluene/dodecanethiol as solvent/co-solvent mixture leads to a large long-range order on the micrometer scale.
  However, no source in literature has been found for this thesis on how to prepare long-range ordered monolayers from ferrite nanoparticles.

  In the following, the preparation of such self-assembled lattices with long-range order is presented in detail on the case of the square lattice from cobalt ferrite nanocubes.
  For this purpose, the synthesis and characterization of magnetic nanocubes is described first, and subsequently the production of monolayer samples from same nanoparticles.
  Multiple parameters that have to be considered are discussed and rule of thumbs to experimentally obtain their optimal value by small variations around the estimates are explained.
  The transfer to other iron spinel nanocubes and nanospheres based on this procedure is also presented, as well as possible variations in the procedure to improve and go beyond the analyzed samples.
  Finally, for selected nanoparticles and monolayers, the quantitative evaluation of the structures are shown, as well as the characterization of their magnetic properties. Emergent magnetic properties that are not observed for the individual nanoparticles are described and compared to model calculations.
\end{document}