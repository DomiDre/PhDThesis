\providecommand{\main}{../../..}
\documentclass[\main/dresen_thesis.tex]{subfiles}
  \renewcommand{\thisPath}{\main/chapters/monolayers/intro}

\begin{document}
  A primary goal of this work is the study of magnetic dipolar interactions between nanoparticles.
  Due to the highly directional nature of dipolar magnetism, random orientation of the nanoparticles and random packing, as described in the previous chapter, cloud the observation of macroscopic effects arising from the interaction.
  Therefore an intermediate goal towards the observation of dipolar interaction is the successful preparation of highly ordered nanostructures.
  Simple arrangements that have been discussed extensively in theory, are the square lattice and hexagonal lattice in the plane.
  Here, dipolar coupling leads to an emergent macroscopic states for the nanostructure, namely a super antiferromagnetic state for the square lattice and a super ferromagnetic state for the hexagonal lattice \cite{Politi_2002_Dipol, Russier_2001_Calcu, Varon_2013_Dipol}.
  In this work, a procedure to fabricate such long range ordered nanostructures from nanocubes and nanospheres has been developed.
  The preparation of such self-assembled lattices with long-range order is presented in detail on the case of the square lattice from cobalt ferrite nanocubes.
  For this purpose this chapter separates into multiple parts, the synthesis following protocols from literature and characterization of magnetic cobalt ferrite nanocubes is described first, and subsequently the production of monolayer samples from same nanoparticles is described.
  For selected nanoparticles and monolayers, the quantitative evaluation of the structures are shown, as well as the characterization of their magnetic properties. Emergent magnetic properties that are not observed for the individual nanoparticles are described and compared to model calculations.
  \\

  Cobalt ferrite nanoparticles are nowadays routinely synthesized following methods such as co-precipitation \cite{Fried_2001_Order}, sol-gel \cite{Niederberger_2009_Metal}, micro emulsions \cite{Pillai_1996_Synth}, or thermal decomposition.
  For thermal decomposition, one can further differentiate between hot-injection \cite{Hyeon_2003_Chemi} and heating up methods \cite{Embden_2015_TheHe}.
  Where the first promises monodisperse nanoparticles by keeping the nucleation time period of the synthesis short, the latter is especially promising in being scaleable and highly controllable \cite{Park_2004_Ultra}.
  The heating up method is used extensively in this work to synthesize nanocubes and nanospheres of cobalt ferrite and iron oxide and is further elaborated in the following.

  A popular heating up route to synthesize nanoparticles is to first prepare a metal oleate precursor from metal salts, which is subsequently slowly heated above it's decomposition temperature in a high-boiling solvent, where it is aged in the presence of oleic acid and additional reagents to direct the growth and shape \cite{Park_2004_Ultra, Wetterskog_2014_Preci}.
  For example, the shape of cobalt ferrite nanoparticles can be tuned by adding sodium oleate as precursor before the heating up process \cite{Bodnarchuk_2009_Excha}.
  The sodium oleate attaches to the (100) facets of forming nanocrystals and fosters the growth along the [111] direction of the crystal \cite{Bodnarchuk_2009_Excha}.
  By tuning the ratio of sodium oleate to the oleate precursor and adjusting the aging time, this process allows to prepare either nanospheres, nanocubes, polyhedral nanoparticles or even star-like nanoparticles \cite{Bodnarchuk_2009_Excha, Bao_2009_Forma, Wetterskog_2014_Preci}.

  Wetterskog \etal extensively studied the formation of maghemite nanospheres and nanocubes, following the same route for iron oleate in the synthesis \cite{Wetterskog_2014_Preci, Wetterskog_2013_Anoma}, and found multiple deviations from the expected pure phase for the nanoparticles.
  On the one hand, during the oleate synthesis \ch{CO} is constantly being produced from the solvents leading to an reaction environment where \ch{Fe^{3+}} is reduced to \ch{Fe^{2+}} \cite{Hai_2010_Sizec}.
  Therefore, in the synthesis a w\"ustite core is formed and only by post-synthesis oxidation a maghemite shell is obtained \cite{Wetterskog_2013_Anoma}.
  On the other hand, a defect structure can be observed in the nanoparticles.
  Even though the nanoparticle magnetization can be enhanced by forced oxidation at elevated temperatures of $150^\circ$ after the synthesis, anti-phase boundaries remain after complete oxidation of the particles \cite{Wetterskog_2013_Anoma} and the defective structures across the nanoparticle volume correlates with a lower magnetization in comparison to bulk material.
  Additionally it is observed, if the particles are exposed to oxygen and temperatures of $150 \unit{^\circ C}$ for over $2 \unit{h}$, the oleic acid shell degrades and the nanoparticles become unstable in dispersion \cite{Wetterskog_2013_Anoma}.
  The same observations of a core-shell structure are also made in literature for the synthesis of cobalt ferrite nanoparticles from metal oleates \cite{Bodnarchuk_2009_Excha}.
  Here however, cobalt ferrite provides a higher oxygen diffusion barrier and it is technically harder to oxidize cobalt ferrite nanoparticles completely \cite{Chen_2015_Synth}.
  Chen \etal \cite{Chen_2015_Synth} showed that after $120 \unit{days}$ only a $2 \unit{nm}$ oxidized shell is observed for $12.5 \unit{nm}$ and $19 \unit{nm}$ sized nanospheres and only after adding dehydrated trimethylamine N-oxide (TMNO) to the as-prepared cobalt ferrite nanoparticles with subsequent oxidation at $160 \unit{^\circ C}$ of the particles for over $20 \unit{h}$, a shell of $4 \unit{nm}$ can be oxidized until the oxygen barrier from cobalt ferrite stops any further oxidation.

  An alternative popular heating up synthesis is achieved by the thermal decomposition of cobalt and iron acetylacetonates in a high boiling solvent such as dibenzyl ether with the presence of oleic acid or oleylamine \cite{Sun_2002_SizeC, Wu_2014_Monol}.
  Again, the addition of sodium oleate leads to the formation of nanocubes, such as in the oleate synthesis route, when the amount is tuned to the oleic acid content and aging time \cite{Wu_2014_Monol}.
  During the acetylacetonate route, no reducing carbon monoxide is formed, however it has to be noted that from the decomposition of the acetylacetonates, acetone and carbon dioxide is constantly being produced \cite{Lu_2015_Synth}
  \begin{align}
    \ch{Co(acac)_2} + \ch{Fe(acac)_3} \rightarrow \ch{CoFe2O4} + \ch{CH3COCH3} + \ch{CO2}.
  \end{align}
  And indeed, x-ray diffraction of particles by acetylacetonates shows an agreement with the crystal structure of \ch{CoFe2O4} \cite{Wu_2014_Monol, Sathya_2016_Cofeo} and elemental mapping further confirms the homogeneous distribution of cobalt and iron inside the particle \cite{Sathya_2016_Cofeo}.
  The ratio of cobalt to iron in the nanoparticle is observed to be lower than the feed ratio of the synthesis meaning that not all of the cobalt acetylacetonate is converted to the nanoparticles and part remains in solution, leading to a composition of \ch{Co_x Fe_y O_4}, however the ratio can tuned towards \ch{CoFe2O4} by starting from a higher feed ratio  \cite{Sathya_2016_Cofeo, Yu_2013_Cobal, Wu_2014_Monol}.
  Using M\"ossbauer spectroscopy, the observation of two sub-spectra with an isomer shift below $\mathrm{IS} < 0.4 \unit{mm/s}$ \cite{Pianciola_2014_Sizea}, for cobalt ferrite particles which are synthesized from acetylacetonates, shows that no \ch{Fe^{2+}} is present but that all iron is in a \ch{Fe^{3+}} state, distributed on the A and B sites of the inverse spinell \cite{Angotzi_2017_Spine, Figuera_2015_Moess}.

  Particles in the size range of $10 - 20 \unit{nm}$ obtained by the acetylacetonate route, show strong magnetic properties such as a coercivity in the order of $2 \unit{T}$ at $10 \unit{K}$ \cite{Sun_2002_SizeC, Sathya_2016_Cofeo} and a saturation magnetization in the order of $230 - 300 \unit{kA m^{-1}}$ \cite{Wu_2014_Monol, Sathya_2016_Cofeo}.
  The homogeneous composition and strong magnetic properties therefore make these particles good candidates for the case study of magnetic interparticle interactions.
  \\

  Multiple methods are known in literature to prepare monolayers on a substrate starting from a nanoparticle dispersion.
  Among the most prominent techniques are the Langmuir-Blodgett/Schaefer method \cite{Ukleev_2017_Selfa, Pauly_2011_Monol, Fried_2001_Order}, spin-coating \cite{Mishra_2012_Selfa}, doctor blade casting \cite{Bodnarchuk_2010_Large}, dip coating \cite{Kim_2002_Multi} and drop casting \cite{Bigioni_2006_Kinet}.
  Depending on the applied technique, the resulting order of the nanoparticles can vary greatly.
  While methods as spin-coating and dip-coating are quick to perform, the in-plane order of the particles is comparably poor to results obtained from evaporation-driven processes that allow for a slow self-assembly process of the particles.

  The Langmuir-Blodgett/Schaefer technique is a method that was designed for the formation of monolayers of organic molecules, but has been transferred to the study of nanoparticle monolayer formation \cite{Heitsch_2010_Gisax, Vorobiev_2015_Subst}.
  Here, the dispersion is allowed to evaporate slowly from an organic solvent on a water surface, where it then is compressed by movable barriers on the surface to monolayer density.
  In-situ XRR and GISAXS studies on the water surface show that a high degree of in-plane order on the surface is possible for $10 \unit{nm}$ sized nanoparticles \cite{Vorobiev_2015_Subst}.
  The final experimental challenge is then to transfer the ordered particles from the surface to the wafer without destroying the order.
  A study of Wen \etal \cite{Wen_2011_Ultral} shows that by modification of the Langmuir through, where additionally the solvent evaporation is controlled, micrometer sized ordered arrays can be obtained by the Langmuir-Schaefer method.


  \begin{figure}[tb]
    \centering
    \includegraphics{monolayers_preparation_dryingStages}
    \caption{\label{fig:monolayers:preparation:dryingConditions:dryingStages} Depiction of the evolution during the first drying stage according to \cite{Bigioni_2006_Kinet}. As the droplet shrinks it fixes the nanoparticles on the droplet surface, where they can still move in two dimensions to form a long range ordered lattice. Finally the ordered structure remains with the remaining slow evaporating solvent.}
  \end{figure}

  In the drop casting method, the nanoparticles are transferred to a substrate by spreading a drop of dispersion on a flat surface and allowing the solvent to evaporate slowly, while the nanoparticles self-assemble \cite{Han_2012_Learn}.
  It is an instrumentally cheap and a material efficient method, where the exact amount of nanoparticles needed for a monolayer on a given area is set.
  Additionally, once a method for the preparation of a monolayer is established, the drop casting method is easily reusable to prepare multilayered samples, by sequential repetition of the process.
  The challenge of the drop casting method is to control the self-assembly process by manipulating the environmental conditions during the evaporation process and choosing the correct solvent for the nanoparticles.
  A phenomenon that is often observed from the drop-casting method is an inhomogeneous particle distribution, where the particles tend to agglomerate at the edge of the drying droplet, known as the 'coffee-ring' effect \cite{Deegan_1997_Capil, Anyfantakis_2015_Modul, Dugyala_2014_Contr}.
  This effect comes from an evaporation-driven capillary flow of the particles induced by the inhomogeneous evaporation profile of a pinned drop with a finite contact angle \cite{Deegan_1997_Capil}.
  The addition of surfactants in the dispersion is known to be able to alter the drying pattern, as they can induce counter-directed Marangoni flows \cite{Kajiya_2009_Contr}, or influence the particle-particle, particle-substrate or particle-free interface interaction \cite{Anyfantakis_2015_Modul}.

  When a solvent for even distribution of the particles on the substrate after drop casting is found, the other challenge is to control the self-assembly process towards long-range order.
  Bigioni \etal \cite{Bigioni_2006_Kinet} showed for dodecanethiol-ligated gold nanoparticles that a combination of toluene/dodecanethiol as solvent/co-solvent mixture leads to a large long-range order on the micrometer scale.
  In the kinetic theory depicted in \reffig{fig:monolayers:preparation:dryingConditions:dryingStages}, the crucial ingredients for long-range ordered monolayers from drop casting are two parameters: the flux of particles to the droplet surface and the diffusion length on the liquid-air interface.
  The particle diffusion to the surface is controlled by the evaporation rate, where the shrinking droplet height catches particles from the dispersion.
  And on the other hand a mechanism needs to be present in the evaporating droplet, which pins the particles on the interface to increase the interfacial diffusion length.
  Furthermore, Bigioni \etal \cite{Bigioni_2006_Kinet} observe that the amount of excess dodecanethiol in the dispersion is critical as to whether long-range ordered monolayers are obtained or not.
  The diffusion length can be tuned by the particle size, surface tension or osmotic pressure and thereby methods can be conceived with which monolayers for arbitrary materials can be prepared.
  In terms of monolayers of oleic acid ligated nanoparticles prepared by drop-casting no study is found in current literature as to which solvent and conditions need to be used to obtain a single homogeneous layer.

  \begin{figure}[tb]
    \centering
    \includegraphics{monolayers_MagneticStructure_fm_field}
    \includegraphics{monolayers_MagneticStructure_afm_field}
    \caption{\label{fig:monolayers:intro:magneticStateSquareArray}Visualization of possible magnetic ground states of dipolar coupled spins on a square lattice. Shown are a super ferromagnetic state (left) and a super antiferromagnetic state (right).}
  \end{figure}
  Magnetic nanoparticle monolayers are interesting as a model system for the study of collective magnetism, especially when they are ordered on a regular lattice.
  A square array of primarily dipolar interacting nanoparticle macrospins is known to have a super antiferromagnetic ground state \cite{Russier_2001_Calcu}.
  This is in contrast to the ground state that would be expected from an hexagonal lattice, where a super ferromagnetic ground state is expected \cite{Russier_2001_Calcu}.
  A super ferromagnetic ground state is characterized by an equal orientation along one direction for neighbouring macrospins in a domain, whereas the super antiferromagnetic ground state is given by rows of equally oriented spins that alternate in direction.
  Both configurations and the resulting magnetic field lines are depicted for the square array in \reffig{fig:monolayers:intro:magneticStateSquareArray}.
  Electron holography studies exist that observe such a super antiferromagnetic state in nanoparticle systems \cite{Varon_2013_Dipol}, but on close inspection of the TEM micrographs the samples do not present a long range ordered square array such as described in the theoretical derivation of the ground state, but are thin short-range ordered chains instead.
  \\

  In this work, a drop-casting method to prepare monolayers from oleic acid-ligated ferrite nanoparticles is presented.
  To obtain a homogeneous distribution of the nanoparticles on a substrate and to achieve long-range order, the influence of the primary solvent, as well as the addition of co-solvents is studied.
  Focus is set on ordering nanoparticles with a cubic morphology, as the closest packing of cubes is a square array.
  The prepared square arrays of magnetic nanocubes are then studied for their magnetic properties to determine whether signatures of either a super ferromagnetic or super antiferromagnetic ground state are present.
  A super antiferromagnetic state on a square array should have a vanishing macroscopic magnetization and would be visible in neutron scattering by the emergence of a magnetic reflection with the double period of the original lattice.
  To decide whether observed effects are from the magnetic interparticle interaction or an individual nanoparticle property, the used nanoparticle batches are pre-characterized by small-angle scattering and complimentary electron microscopy, magnetometry and diffraction experiments.
  Aim is then to obtain from macroscopic magnetization measurements, as well as PNR and polGISANS an unbiased observation of dipolar interparticle coupling in a long-range ordered square array of magnetic nanocubes.
\end{document}