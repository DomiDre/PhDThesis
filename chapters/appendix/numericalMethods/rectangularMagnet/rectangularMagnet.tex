\providecommand{\main}{../../../..}
\documentclass[\main/dresen_thesis.tex]{subfiles}
\renewcommand{\thisPath}{\main/chapters/appendix/numericalMethods/superball}

\begin{document}
\section{Rectangular Magnet}\label{ch:appendix:numericalMethods:rectangularMagnet}
For a hard magnetic material with a constant bulk magnetization of $\vec{M} \eq M_0 \vec{e}_z$ the produced magnetic field can be calculated by using the macroscopic Maxwell equations. For once we have
\begin{align}
  \vec{\nabla} \times \vec{H} \eq& 0\\
  \Rightarrow \vec{H} \eq& - \vec{\nabla} \phi_m,
\end{align}
where $\rho_{m}$ is a potential introduced to simplify the calculations. Furthermore, we have
\begin{align}
  \vec{\nabla}  \cdot \vec{B} \eq& 0
\end{align}
and the material equation
\begin{align}
  \vec{H} \eq& \mu_0 (\vec{B} + \vec{M}).
\end{align}
Combining the three equations we obtain
\begin{align}
  \Delta \phi_m \eq& - \mu_0 \nabla \cdot \vec{M},
\end{align}
The solution for this differential equation depends on the surface of the magnetized body and can be written according to Greens theorem as integral
\begin{align}
\phi_m \eq \frac{\mu_0}{4 \pi} \int_{\partial V} \frac{\vec{n}^\prime \cdot \vec{M}(\vec{r}^\prime)}{\left| \vec{r} - \vec{r}^\prime \right|} \mathrm{d}A - \frac{\mu_0}{4 \pi} \int_V \frac{\vec{\nabla}^\prime \cdot \vec{M}(\vec{r}^\prime)}{\left| \vec{r} - \vec{r}^\prime \right|} \mathrm{d}V^\prime,
\end{align}
where the second addend vanishes for a homogeneously magnetized body.

For a rectangular magnet, the integral can be solved and thus by taking the magnetic field inside and outside of the magnet determined. At a point $(x,y,z)$ relative to a rectangular magnet with width $a$, length $b$, height $c$, the corner set at $(0,0,0)$ and a magnetization $M_0$ along its $z$-axis it's determined by
\begin{align}
  \begin{split}
    B_x \eq {}& -\frac{\mu_0 M_0}{8 \pi} \bigg( \\
              & \Big(\Gamma(a-x,y,z,c) +\Gamma(a-x,b-y,z,c) -\Gamma(x,y,z,c) -\Gamma(x,b-y,z,c)\Big) -\\
              & \Big(\Gamma(a-x,y,z,0) +\Gamma(a-x,b-y,z,0) -\Gamma(x,y,z,0) -\Gamma(x,b-y,z,0)\Big) \bigg)
  \end{split} \\
  \begin{split}
    B_y \eq {}& -\frac{\mu_0 M_0}{8 \pi} \bigg( \\
              & \Big(\Gamma(b-y,x,z,c) +\Gamma(b-y,a-x,z,c) -\Gamma(y,x,z,c) -\Gamma(y,a-x,z,c)\Big) -\\
              & \Big(\Gamma(b-y,x,z,0) +\Gamma(b-y,a-x,z,0) -\Gamma(y,x,z,0) -\Gamma(y,a-x,z,0)\Big) \bigg)
  \end{split} \\
  \begin{split}
    B_z \eq {}& -\frac{\mu_0 M_0}{4 \pi} \bigg( \\
              & \Big(\Phi(y,a-x,z,h) + \Phi(b-y,a-x,z,h) + \Phi(x,b-y,z,h) + \Phi(a-x,b-y,z,h) +\\
              &\Phi(b-y,x,z,h) + \Phi(y,x,z,h) + \Phi(a-x,y,z,h) + \Phi(x,y,z,h)\Big) -\\
              & \Big(\Phi(y,a-x,z,0) + \Phi(b-y,a-x,z,0) + \Phi(x,b-y,z,0) + \Phi(a-x,b-y,z,0) +\\
              &\Phi(b-y,x,z,0) + \Phi(y,x,z,0) + \Phi(a-x,y,z,0) + \Phi(x,y,z,0)\Big) \bigg)
  \end{split}
\end{align}
with
\begin{align}
\Gamma(\gamma_1, \gamma_2, \gamma_3, z_0) \eq \log \Bigg( \frac{\sqrt{\gamma_1^2 + \gamma_2^2 + (\gamma_3 - z_0)^2} - \gamma_2}{\sqrt{\gamma_1^2 + \gamma_2^2 + (\gamma_3 - z_0)^2} + \gamma_2} \Bigg) \\
\Phi(\phi_1, \phi_2, \phi_3, z_0) \eq \arctan \Bigg( \frac{\phi_1}{\phi_2} \frac{\phi_3 - z_0}{\sqrt{\phi_1^2 + \phi_2^2 + (\phi_3-z_0)^2}} \Bigg)
\end{align}

For $(a=b)$. The solution along the z-axis $(x,y)=(a/2, a/2)$ simplifies to 
\begin{align}
  \begin{split}
    B_x \eq {}& 0
  \end{split} \\
  \begin{split}
    B_y \eq {}& 0
  \end{split} \\
  \begin{split}
    B_z \eq {}& \frac{2 \mu_0 M_0}{\pi} \Bigg( 
              \arctan \bigg( \frac{z}{\sqrt{a^2/2 + z^2}} \bigg) -
              \arctan \bigg( \frac{z - h}{\sqrt{a^2/2 + (z-h)^2}} \bigg) \Bigg)
  \end{split}
\end{align}

And the the magnetic field at the center of the surface is
\begin{align}
  B_z \eq {}& \frac{2 \mu_0 M_0}{\pi} 
              \arctan \bigg( \frac{1}{\sqrt{1 + a^2/2h^2}} \bigg)
\end{align}

% \textbf{References}\\
% [1] Gou et al. Appl. Math. Mech. Engl. Ed., 25, 3 (2004)
\end{document}