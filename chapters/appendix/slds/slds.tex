\providecommand{\main}{../../..}
\documentclass[\main/dresen_thesis.tex]{subfiles}
\begin{document}
  \chapter{Scattering Lenght Densities}
  \label{appendix:slds}
  A list of atomic form factors and nuclear coherent scattering lengths of elements studied in this thesis by X-ray and neutrons is given in \reftab{tab:appendix:affNuc}.
  Scattering length densities used in small-angle scattering for solvents and the oleic acid surfactant shell are given in \reftab{tab:appendix:slds}.

  \begin{table}[ht]
    \centering
    \caption{\label{tab:appendix:affNuc}Atomic form factor $f$ (at $\lambda \eq 1.3414 \unit{\angstrom}$) and the nuclear coherent scattering length $b$ of elements used in this thesis \cite{Sears_1992_Neutr, BerkeleyLab_1993_asf}.}
    \begin{tabular}{ c | l | c }
                & $f$       & $b \, / \unit{fm}$ \\
      \hline
      $\ch{O}$  & 8.04077   & 5.803   \\
      $\ch{Co}$ & 26.3717   & 2.49  \\
      $\ch{Fe}$ & 25.7468   & 9.45  \\
      \hline
    \end{tabular}
  \end{table}

  \begin{table}[!htbp]
    \centering
    \caption{\label{tab:appendix:slds}Scattering length densities used in this thesis for materials studied by SAS, GISAS and reflectometry. Here, $\rho$ is the mass density of the material, $\rho_\mathrm{el.}$ the electron scattering length density at $1.3414 \unit{\angstrom}$ for X-ray experiments and $\rho_\mathrm{nuc.}$ the nuclear scattering length density for neutron experiments.}
    \begin{tabular}{ l | l | l | l }
      \textbf{Material}  & $\rho / \unit{g \, mL^{-1}}$ & $\rho_\mathrm{el.} \, / \unit{10^{-6} \angstrom^{-2}}$ & $\rho_\mathrm{nuc.} \, / \unit{10^{-6} \angstrom^{-2}}$\\
      \hline
      oleic acid               & $0.895$        & $8.52$       & $0.078$\\
      \textit{c}-hexane        & $0.779$        & $7.55$       & $-0.279$\\
      \textit{n}-hexane        & $0.655$        & $6.46$       & $-0.571$\\
      toluene                  & $0.867$        & $8.01$       & $0.941$\\
      toluene-\textit{d8}      & $0.943$        & $8.00$       & $5.664$\\
      \hline
    \end{tabular}
  \end{table}

\end{document}
