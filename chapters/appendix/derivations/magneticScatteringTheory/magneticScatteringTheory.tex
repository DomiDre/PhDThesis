\providecommand{\main}{../../../..}
\documentclass[\main/dresen_thesis.tex]{subfiles}

\begin{document}
  \section{Magnetic Scattering Of Polarized Neutrons}
    \label{ch:appendix:calculations:magneticScatteringTheory}
    The interaction of a polarized neutron with magnetic moment $\vec{\mu}_n \eq \mu_n \hat{\sigma}$ with a magnetic field can be modelled by a Zeeman potential
    \begin{align}
      V_m (\vec{r}) \eq - \vec{\mu}_n \cdot \vec{B},
    \end{align}
    where the components of the magnetic moment are given by the Pauli matrices,
    \begin{align}
      \mu_n^i \eq \mu_n \hat{\sigma}_i,
    \end{align}
    \begin{align}
      \hat{\sigma}_x \eq \begin{pmatrix} 0 & 1 \\1 & 0 \end{pmatrix},\,\,
      \hat{\sigma}_y \eq \begin{pmatrix} 0 & -i \\i & 0 \end{pmatrix},\,\,
      \hat{\sigma}_z \eq \begin{pmatrix} 1 & 0 \\0 & -1 \end{pmatrix},
    \end{align}
    A magnetic sample generates a field for once by the orbital motion of bound electrons $\vec{B}_L$, which can be modelled by the Biot-Savart law, and on the other hand by the electron spins $\vec{B}_S$ which can be modelled for a single electron at position $\vec{r}_i$ by a dipole field
    \begin{align}
      \vec{B}_L \eq& -\frac{\mu_0 e}{4 \pi} \vec{v} \times \vec{\nabla} \frac{ 1}{|\vec{r} - \vec{r}_i|}\\
      \vec{B}_S \eq& \frac{\mu_0}{4\pi} \vec{\nabla} \times \biggl(\vec{\mu}_e \times \vec{\nabla} \frac{ 1}{|\vec{r} - \vec{r}_i|} \biggr).
    \end{align}
    The contribution of the orbital movement is relatively complex to evaluate as it depends on the specific shell and atom the electrons state is in.
    In most cases it is considered as a small contribution relatively to the spin magnetization and is therefore neglected in this work \cite{Zhu_2005_Moder}.
    Focusing on the spin-dependant interaction one can use the Fourier identity
    \begin{align}
      \frac{1}{|\vec{r} - \vec{r}_i|} \eq \frac{1}{2 \pi^2} \int \dint \vec{q} \frac{e^{i \vec{q} \cdot (\vec{r} - \vec{r}_i)}}{q^2},
    \end{align}
    and replace the magnetic moment of a single electron $\vec{\mu}_e$ at position $\vec{r}_i$ by the integral over a spin-density
    \begin{align}
      \vec{\mu}_e \rightarrow \mu_B \int \dint \vec{r}^\prime \vec{s}_e(\vec{r}^\prime),
    \end{align}
    with the Bohr magneton $\mu_B \eq e\hbar / 2 m_e \approx 9.274 \cdot 10^{-24} \unit{JT^{-1}}$ and write the Fourier integral over the potential for the differential cross-section in \refeq{eq:theoreticalBackground:scattering:scatteringTheory:differentialCrossSectionIntegralOverPotential} as

    \begin{align}
      \begin{split}
        f_M (\vec{q})
          &\eq - \frac{m_n}{2\pi \hbar^2} \int \dint \vec{r} e^{-i\vec{q} \cdot \vec{r}}\vec{\mu}_n \cdot \vec{B}_S (\vec{r})\\
          &\eq - \frac{\mu_0 \mu_B m_n}{(2\pi)^4 \hbar^2}  \int \dint \vec{r} e^{-i\vec{q} \cdot \vec{r}}\vec{\mu}_n \cdot \biggl[ \vec{\nabla} \times \biggl(\int \dint \vec{r}^\prime \vec{s}_e(\vec{r}^\prime) \times \vec{\nabla} \int \dint \vec{q}^\prime \frac{e^{i \vec{q}^\prime \cdot (\vec{r} - \vec{r}^\prime)}}{q^{\prime 2}} \biggr) \biggr].
      \end{split}
    \end{align}
    Using
    \begin{align}
      \vec{\nabla}e^{i \vec{q}^\prime \cdot (\vec{r} - \vec{r}^\prime)} \eq& i \vec{q} e^{i \vec{q}^\prime \cdot (\vec{r} - \vec{r}^\prime)},\\
      \int \dint \vec{r} e^{i(\vec{q}^\prime - \vec{q}) \cdot \vec{r}} \eq& (2 \pi)^3 \delta(\vec{q}^\prime - \vec{q}),\\
      \hat{q} \eq \frac{\vec{q}}{q}
    \end{align}
    the integral reads
    \begin{align}
      f_M (\vec{q}) \eq & \frac{\mu_0 \mu_B m_n}{2 \pi \hbar^2} \vec{\mu}_n \cdot \biggl[ \hat{q} \times \biggl( \int \dint \vec{r}^\prime e^{- i \vec{q} \cdot \vec{r}^\prime} \vec{s}_e(\vec{r}^\prime) \times \hat{q}\biggr) \biggr].
    \end{align}
    Considering the case of a constant spin direction $\vec{s}_e (\vec{r}^\prime) \eq s_e (\vec{r}^\prime) \hat{s}$, the direction and magnitude separate into
    \begin{align}
      f_M (\vec{q}) \eq & \hat{\mu}_n \cdot \biggl[ \underbrace{ \hat{q} \times \biggl( \hat{s} \times \hat{q}\biggr) }_{\eq \hat{s}_\perp}\biggr] \frac{m_n}{\hbar^2}  \frac{\mu_0}{2\pi} \mu_B \mu_n   \int \dint \vec{r}^\prime e^{- i \vec{q} \cdot \vec{r}^\prime} s_e(\vec{r}^\prime).
    \end{align}
    The two cross products reduce $\hat{s}$ only to the contribution that is perpendicular to $\vec{q}$ as is evident by applying the Grassman identity $\vec{a} \times (\vec{b} \times \vec{c}) \eq (\vec{a} \cdot \vec{c}) \vec{b} - (\vec{a} \cdot \vec{b}) \vec{c}$.
    This result means that only when the spins of a material are aligned perpendicularly to the scattering vector, one measures a magnetic scattering contribution.
    The prefactor consist of only natural constants and evaluates to
    \begin{align}
      \frac{m_n}{\hbar^2}  \frac{\mu_0}{2\pi} \mu_B \mu_n \approx 2.699\,\cdot 10^{-5} \unit{\angstrom}.
    \end{align}
    By defining the magnetic scattering length density by
    \begin{align}
      \rho_\mathrm{mag} (\vec{r}) \eq \frac{m_n}{\hbar^2}  \frac{\mu_0}{2\pi} \mu_B \mu_n s_e (\vec{r}),
    \end{align}
    one obtains a similar formula for the magnetic scattering amplitude contribution to the differential cross section as for the nuclear contribution
    \begin{align}
      f_M (\vec{q}) \eq & \hat{\mu}_n \cdot \hat{s}_\perp \int \dint \vec{r} e^{- i \vec{q} \cdot \vec{r}} \rho_\mathrm{mag}(\vec{r}).
    \end{align}
    In an experiment with polarized neutrons, the polarization of the incoming beam is defined by a polarizer and stabilized by a weak magnetic guide field, which defines the $z$ direction in terms of the spin operator.
    When $\hat{s}_\perp$ is parallel (antiparallel) to $z$, $f_M$ contributes with a positive (negative) sign to the differential cross section and the direction of the spin is conserved during the scattering.
    On the other hand, when $\hat{s}_\perp$ is perpendicular to $z$, either $\hat{\sigma}_x$ or $\hat{\sigma}_y$ act and the neutron spin flips therefore during the scattering.
    In this experiment situation one would measure the magnetic scattering contributions therefore in the spin-flip channel.
\end{document}