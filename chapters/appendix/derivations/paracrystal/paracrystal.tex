\providecommand{\main}{../../../..}
\documentclass[\main/dresen_thesis.tex]{subfiles}

\begin{document}
\section{Paracrystals and Disorder of the 2nd Kind}\label{ch:appendix:calculations:paracrystal}
The structure factor from $N$ scattering centers is, as described in \refeq{eq:theoreticalBackground:scattering:scatteringTheory:structureFactor}, given by
\begin{align}
  S(\vec{q}) \eq \frac{1}{N} \sum_{j, k}^{\infty} e^{i\vec{q} \cdot (\vec{x}_j - \vec{x}_k)}.
\end{align}
For a regularly ordered structure, this results in sharp peaks when the scattering vector is such that it corresponds to a regular lattice spacing via $\vec{q} \cdot \Delta \vec{x} \eq 2 \pi n$ where the width of the peaks is determined by instrumental resolution and finite size of the lattice.

In atomic crystals, the atoms can fluctuate from their mean position due to thermal excitation.
This is termed as disorder of the first kind as it preserves the long range order and it can be accounted for by a Debye-Waller factor
\begin{align}
  I \rightarrow I e^{-q^2 \braket{u^2}/3},
\end{align}
where u is the atomic displacement.
Disorder of the first kind reduces the peak height but does not further broaden the peak width.

On the other hand, especially in the case of nanostructured materials, the position of the nearest neighbour of a scatterer can also be systematically shifted from the mean position.
From the point of view of a single scattering center, this systematic offset of the nearest neighbour propagates to the next-nearest neighbour, the next-to-next nearest neighbour and so forth, increasing the uncertainty of the position ever larger the further away.
This systematic displacement is termed disorder of the second kind or paracrystalline order and it effectively leads to an additional broadening of the peak.

The paracrystal is typically derived in one dimension and then extended to higher dimensions by convolution.
The nearest neighbour probability distribution is noted as $p_1(x) \eq p(x-a)$, where $a$ is the lattice constant of the regular chain and $p$ the probability distribution for the displacement from the ideal structure for the nearest neighbour.
The probability for the $n$-th lattice site offset is then calculated by the convoluting the probability distribution of the nearest neighbour offset $n$ times
\begin{align}
  p_n (x) \eq \circledast^{|n|} p(x-a).
\end{align}
Implementing the probability distributions in the formula for the structure factor of a one dimensional regular lattice, it reads
\begin{align}
  S(q) \eq \sum_{n\eq -N/2}^{N/2} \int_{- \infty}^\infty p_n(x) e^{i q x} \dint x.
\end{align}
The ideal lattice is returned at this point, when as probability distribution the Delta distribution $p(x) \eq \delta(x)$ is used.

The integral that appears in the structure factor, is just the Fourier transform of $p_n(x)$.
Using the convolution theorem and the property of the Fourier transform for a variable shift, it can be given as
\begin{align}
  \mathcal{F} [p_n](q) \eq e^{-inqa} P^{|n|}(q).
\end{align}
where $P(q)$ is the Fourier transform of the nearest neighbour offset probability distribution $P(q) \eq \mathcal{F} [p](q)$.
The structure factor is then
\begin{align}\begin{split}
  S(q) \eq& \sum_{n\eq -N/2}^{N/2} e^{-inqa} P^{|n|}(q)\\
  \eq& -1 + 2\Re e\sum_{n\eq 0}^{N/2} \Bigl(e^{iqa} P(q) \Bigr)^n\\
  \underset{N \rightarrow \infty}{\Rightarrow}& \frac{1 - P^2(q)}{1 - 2 P(q) \cos(qa) + P^2(q)}.\label{eq:appendix:calculations:paracrystal:StructureFactor}
\end{split}\end{align}
Where it was used that the sum of a complex number with it's complex conjugate is just two times it's real part, the geometric sum is then performed and the lattice is assumed to be infinite.
It is however enough to assume that the lattice is larger than the coherence length, which is discussed in the following for the case of the Gaussian probability distribution.

The Gaussian probability distribution is given by
\begin{align}
  p(x) \eq \frac{1}{\sqrt{2 \pi} \sigma} \exp \biggl( - \frac{x^2}{2 \sigma^2}\biggr),
\end{align}
where $\sigma$ is the uncertainty in the nearest neighbour position.
As the convolution of two Gaussians with uncertainty $\sigma_1$ and $\sigma_2$ is again a Gaussian with uncertainty $\sigma^2 \eq \sigma_1^2 + \sigma_2^2$, the uncertainty in position of the $n$-th lattice site is given by a Gaussian with uncertainty $\sigma_n \eq \sqrt{n} \sigma$.
The coherence length of the paracrystal is then defined as the length scale $L_\mathrm{coh} \eq n a$, where the uncertainty is equal to the lattice spacing $\sigma_n \eq a$, which reads
\begin{align}
  L_\mathrm{coh} \eq \frac{a^3}{\sigma^2}.\label{eq:appendix:calculations:paracrystal:CoherenceLength}
\end{align}
Beyond this length scale, the lattice appears disordered as the uncertainty in position becomes too large.

To discuss the structure factor for the Gaussian probability distribution in the framework of the paracrystal model, the Fourier transform
\begin{align}
  P(q) \eq e^{-\frac{q^2 \sigma^2}{2}},
\end{align}
is inserted in \refeq{eq:appendix:calculations:paracrystal:StructureFactor} and after some reorganization it is given by
\begin{align}
  S(q) \eq \frac{\coth(q^2 \sigma^2/4)}{1 + \frac{1 - \cos(qa)}{1 - \cosh(q^2 \sigma^2/2)}}.
\end{align}
At the peak positions $q_{p, n} \eq 2 \pi n/a$, the cosine is one and the structure factor has the value
\begin{align}
  S(q_{p, n}) \eq \coth \Bigl( \frac{n^2 \pi^2 \sigma^2}{a^2} \Bigr) \underset{\sigma \ll a}{\approx} \frac{a^2}{n^2 \pi^2 \sigma^2},
\end{align}
where it is assumed that the uncertainty in the nearest neighbour position is small in comparison to the lattice constant.
In the small-q region where the peaks are well defined $q \sigma \ll 1$, the structure factor can be approximated around the peak position by a Lorentz function, which becomes apparent when the cosine is expanded around $q_{p, n}$ by $\cos(qa) \approx 1 - \Delta q^2 a^2 /2$ (with $\Delta q \eq q - q_{p, n}$) and the cosine hyperbolicus around zero $\cosh(x) \approx 1 + x^2/2$ is used
\begin{align}
  S(q) \approx \frac{S(q_{p, n})}{1 + \biggl(2\frac{\Delta q}{q_{p,n}^2 \sigma^2/a} \biggr)^2}.
\end{align}
From this, the FWHM of the Lorentzian $\gamma_n$ for the $n$-th peak reads as
\begin{align}\begin{split}
  \gamma_n \eq& \frac{q_{p,n}^2 \sigma^2}{a}\\
  \eq& \frac{4 \pi^2 n^2 \sigma^2}{a^3},
\end{split}\end{align}
from which it becomes visible that the peak width broadens with the square of increasing order of the peak.
This is characteristic for disorder of the second kind and can be used to distinguish it from broadening due to finite size effects.

Using \refeq{eq:appendix:calculations:paracrystal:CoherenceLength}, the coherence length can be related to the inverse of the FWHM by
\begin{align}
  L_\mathrm{coh} \eq \frac{4 \pi^2 n^2}{\gamma_n},\label{eq:appendix:calculations:paracrystal:CoherenceFWHMEquation}
\end{align}
which is used to extract from the line profile a measure of the coherence length for a paracrystal.
\end{document}