\providecommand{\main}{../../../..}
\documentclass[\main/dresen_thesis.tex]{subfiles}

\begin{document}
\section{Semiclassical Magnetization of a Spin}\label{ch:appendix:calculations:magnetizationClassicalSpin}
In this work, nanoparticles that comprise of a ferromagnetic domain of $\gg 1000$ collinear spins are commonly discussed.
When many spins couple to form a super spin, the quantum mechanic results converge with the results from classical calculations for large spin quantum numbers.
Therefore, in the following the magnetization behaviour of a super spin is discussed in the context of classical physics.
This approach to describe the magnetization is also valid if a single electron spin is treated semi classically and thus to align freely in all directions.

As discussed in \refch{ch:theoreticalBackground:magnetism:paramagnetism}, a single spin exposed to an external magnetic field $B$ feels a torque to minimize the Zeeman energy
\begin{align}
  E_Z = - \vec{\mu}_e \cdot \vec{B}.
\end{align}
To determine the equilibrium position, if the spin is in a system at a temperature $T$, the canonical partition function $Z$ has to be evaluated to determine the equilibrium state that maximizes the entropy for the given temperature.
The partition function is calculated by integrating the Boltzmann factor over all possible orientations of $\vec{\mu}$
\begin{align}
  Z \eq& \int_0^{2\pi} \dint \phi \int_0^\pi \dint \theta \sin(\theta) \exp \biggl(\frac{\mu B}{k_B T} \cos(\theta) \biggr)\\
  \eq& 4 \pi \frac{k_B T}{ \mu B} \sinh \biggl( \frac{\mu B}{k_B T} \biggr).
\end{align}
From which the free energy is given by
\begin{align}
  F \eq& -k_B T \log(Z)\\
  \eq& -k_B T \log \biggl( 4 \pi \frac{k_B T}{ \mu B}  \sinh \biggl( \frac{\mu B}{k_B T} \biggr) \biggr),
\end{align}
and the expectation value of the magnetization deduces to
\begin{align}
  \tilde{M} \eq& - \frac{\partial F}{\partial B}\\
  \eq& k_B T \frac{\partial}{\partial B} \log \biggl( 4 \pi \frac{k_B T}{ \mu B}  \sinh \biggl( \frac{\mu B}{k_B T} \biggr) \biggr)\\
  \eq& k_B T \Biggl(
    \underbrace{\frac{\partial}{\partial B} \log \biggl( 4 \pi \frac{k_B T}{ \mu } \biggr)}_{\eq 0}
  + \underbrace{\frac{\partial}{\partial B} \log \biggl( \frac{1}{B} \biggr)}_{\eq - \frac{1}{B}}
  + \underbrace{\frac{\partial}{\partial B}\log  \biggl(  \sinh \biggl( \frac{\mu B}{k_B T} \biggr) \biggr) \Biggr)}_{\eq \frac{\mu }{k_B T} \coth \biggl( \frac{\mu B}{k_B T} \biggr) \biggr)}\\
  \eq& \mu \biggl(\coth \biggl( \frac{\mu B}{k_B T} \biggr) - \frac{k_B T}{\mu B} \biggr).
\end{align}
The resulting function for the magnetization is named after Paul Langevin as Langevin function.
For a collection of $N$ particles the magnetization can be summed up if they can be assumed to be decoupled.

For a superparamagnetic material, the magnetization is normalized to the volume of magnetic material $V$
\begin{align}
  M \eq \frac{N \mu}{V} \biggl(\coth \biggl( \frac{\mu B}{k_B T} \biggr) - \frac{k_B T}{\mu B} \biggr).
\end{align}
The prefactor is then a material constant, known as the saturation magnetization $M_s$.
When the volume is written as collection of the single particle volumes $V \eq N V_p$, the relation between $mu$, $M_s$ and $V_p$ is
\begin{align}
  M_s \eq \frac{\mu}{V_p}.
\end{align}

In the case of relatively weak magnetic fields or high temperatures $\mu B \ll k_B T$, the Langevin function can be approximated with its first Taylor series component as
\begin{align}
  M \eq \frac{N \mu}{V} \frac{\mu}{3 k_B T} B.
\end{align}
\end{document}