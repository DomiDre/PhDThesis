\providecommand{\main}{../../../..}
\documentclass[\main/dresen_thesis.tex]{subfiles}

\begin{document}
\section{Scattering Theory For Electromagnetic Waves}\label{ch:appendix:calculations:scatteringTheoryElectromagneticWaves}
In \refsec{sec:theoreticalBackground:scattering:scatteringTheory} the scattering theory was derived in the frame work of quantum mechanics which is valid for non-interacting and non-relativistic particles such as neutrons.
This work also relies heavily on x-ray scattering to study the structure of nanoparticles and their assemblies.
The correct description of scattering for photons is given by quantum electrodynamics.
However, the propagation of x-ray photons is well described by classical electrodynamics for all cases that are relevant in this work.
In this appendix it is shown that classical electrodynamics leads to the same type of differential equation for the description of the propagation of x-ray photons, as the Schr\"odinger equation which is used for neutrons.
To describe the general propagation of photons, one starts from Maxwell's macroscopic equations, which read \cite{Jackson_1999_Class}.
\begin{align}
  \vec{\nabla} \times \vec{E} + \partial_t \vec{B} \eq& 0,\\
  \vec{\nabla} \cdot \vec{B} \eq& 0,\\
  \vec{\nabla} \cdot \vec{D}  \eq& \rho,\\
  \vec{\nabla} \times \vec{H} - \partial_t \vec{D} \eq& \vec{j}
\end{align}
with the material equations
\begin{align}
  \vec{D} \eq& \epsilon_0 \vec{E} + \vec{P},\\
  \vec{B} \eq& \mu_0 (\vec{H} + \vec{M}).
\end{align}
Here, $\mu_0 = 4 \pi \cdot 10^{-7} \unit{N A^{-2}}$ and relates with $\epsilon_0$ to the speed of light via
\begin{equation}
  c \eq \frac{1}{\sqrt{\mu_0 \epsilon_0}}.
\end{equation}
$\rho$ and $\vec{j}$ free charge and current densities, $\vec{E}$ is the electrical field, $\vec{B}$ the magnetic field, $\vec{D}$ the electric displacement field and $\vec{H}$ the magnetizing field. 

The polarization $\vec{P}$ and the magnetization $\vec{M}$ describe macroscopically all the microscopic dipoles and ring currents within a material.
They are generally expressed in relation to $\vec{E}$ and $\vec{H}$
\begin{align}
  \vec{P} \eq& \epsilon_0 \chi_e \vec{E},\\
  \vec{M} \eq& \chi_m \vec{H},
\end{align}
where the electric and magnetic susceptibilities $\chi_e$, $\chi_m$ are in general tensors of 2nd-order and can be complicated functions that depend on the history of a sample (for example in the case of hysteresis).
For this discussion, we assume that polarization and magnetization of the materials respond linearly and isotropic to the field.
With
\begin{align}
  \epsilon_r \eq& 1 + \chi_e,\\
  \mu_r \eq& 1 + \chi_m,
\end{align}
the refractive index is defined as
\begin{equation}
  n \eq \sqrt{\epsilon_r \mu_r},
\end{equation}
and the phase velocity of light inside a material by
\begin{equation}
  c_p \eq \frac{1}{\sqrt{\epsilon_0 \mu_0 \epsilon_r \mu_r}} \eq \frac{c}{n}.
\end{equation}
Plugging these definitions into Maxwell's equation
\begin{align}
  \vec{\nabla} \times \vec{E} + \partial_t \vec{B} \eq& 0,\\
  \vec{\nabla} \cdot \vec{B} \eq& 0,\\
  \epsilon_0 \epsilon_r \vec{\nabla} \cdot \vec{E}  \eq& \rho,\\
  \vec{\nabla} \times \vec{B} - \frac{1}{c_p^2} \partial_t \vec{E} \eq& \mu_0 \mu_r \vec{j},
\end{align}
taking the curl on both sides of the first equation, and inserting the other equations, one arrives at
\begin{align}
  \vec{\nabla} \times \vec{\nabla} \times \vec{E} + \frac{1}{c_p^2}\partial_t^2  \vec{E} \eq& - \mu_0 \mu_r \partial_t \vec{j}.
\end{align}
which finally transform with the general vector identity
\begin{align}
  \vec{\nabla} \times \vec{\nabla} \times \vec{A} = \vec{\nabla} (\vec{\nabla} \cdot \vec{A}) - \Delta \vec{A}
\end{align}
to the general wave equation for $\vec{E}$
\begin{align}
  \Delta \vec{E} - \frac{1}{c_p^2}\partial_t^2  \vec{E} \eq& \mu_0 \mu_r \partial_t \vec{j} + \frac{1}{\epsilon_0 \epsilon_r}\vec{\nabla} \rho.
\end{align}
This differential equation, derived from Maxwell's equation describes in general the propagation of electromagnetic waves.
Assuming that there is no explicit time dependence in the material that is studied, the time dependence of $\vec{E}$ can be factorized to a simple phase factor
\begin{align}
  \vec{E}(\vec{r}, t) \eq \vec{E} (\vec{r}) e^{-i\omega t},
\end{align}
and with the dispersion relation
\begin{align}
  \omega \eq c k,
\end{align}
the wave equation reads
\begin{align}
  (\Delta + n^2 k^2 )\vec{E} \eq& \frac{1}{\epsilon_0 \epsilon_r} \biggl(  \frac{n^2}{c^2} \partial_t \vec{j} + \vec{\nabla} \rho \biggr),
\end{align}
where again the definition of the phase velocity was used on the right hand side. This is an inhomogeneous Helmholtz-equation and it's interesting to note here the close resemblance to the time-independent Schr\"odinger equation, which reads with $k^2 \eq 2mE/\hbar^2$ in positional space as
\begin{align}
  (\Delta + k^2 )\psi \eq& \frac{2m}{\hbar^2} V \psi.
\end{align}
Therefore one can nicely observe at this point how the two theories of classical electrodynamics and quantum mechanics result in a similar description of different problems.

For a further discussion about the x-ray scattering process, it is necessary to describe how the x-ray photons interacts with matter.
It's illustrative to discuss the electric field generated by a single electron cloud oscillating in phase with an incoming field
\begin{align}
  \vec{E}_i (\vec{r}, t) \eq \hat{e}_i E^0 e^{i (\vec{k} \cdot \vec{r} - \omega t)},
\end{align}
where the unit vector $\hat{e}_i$ defines the direction of polarization of the electric field and is perpendicular to $\vec{k}$.
There is no free static charge in this model but only a single moving cloud with density distribution $\rho_e(\vec{r})$ that oscillates with $v(t)$
\begin{align}
  \rho (\vec{r}, t) \eq& 0,\\
  \vec{j} (\vec{r}, t) \eq& -e \rho_e(\vec{r}) \vec{v} (t).
\end{align}
One can use Newton's law for a simplified model of the acceleration of an electron as reaction to the electric field via
\begin{align}
  \partial_t \vec{v} \eq \vec{a} \eq -\frac{e}{m} \hat{e}_i E^0 e^{i(\vec{k}\cdot \vec{r} - \omega t)}.
\end{align}
Then the Helmholtz equation for the electric field reads
\begin{align}
  (\Delta + n^2 k^2 )\vec{E} \eq& \frac{n^2 e^2}{\epsilon_0 \epsilon_r m c^2} \rho_e \hat{e}_i E^0 e^{i\vec{k}\cdot \vec{r}}.
\end{align}
For x-rays $n$ is close to 1 (deviating in the order $10^{-5}$), therefore we can approximate for this discussion $n^2 \approx 1$.
To solve this differential equation, it is helpful again to use the Green's function like in \refsec{ch:appendix:calculations:greenFunctionFreeHamiltonian}, which fulfils the relation
\begin{align}
  (\Delta + k^2) G(\vec{r}, \vec{r}^\prime) \eq& \delta(\vec{r} - \vec{r}^\prime).
\end{align}
It is given by \cite{Jackson_1999_Class}
\begin{align}
  G^{\pm}(\vec{r}, \vec{r}^\prime) \eq& - \frac{1}{4\pi} \frac{e^{\pm i k |\vec{r} - \vec{r}^\prime|}}{|\vec{r} - \vec{r}^\prime|},
\end{align}
for an outgoing (+) and inward going (-) spherical wave with the center at $\vec{r}^\prime$. Again, for scattering only the outgoing solution is interesting to describe radiation.

With the Green's function the inhomogeneous Helmholtz equation is solved straight forward as superposition of the homogeneous solution and the outgoing solution
\begin{align}
  \vec{E} (\vec{r}, t) \eq& \vec{E}_i(\vec{r}, t) + \hat{e}_i \frac{e^2}{4\pi \epsilon_0 m c^2} E^0 \int \dint V^\prime \frac{e^{i k |\vec{r} - \vec{r}^\prime|}}{|\vec{r} - \vec{r}^\prime|} \rho_e (\vec{r}^\prime) e^{i\vec{k}\cdot \vec{r}}.
\end{align}
Using the same approximations for $|\vec{r} - \vec{r}^\prime|$ and $|\vec{r} - \vec{r}^\prime|^{-1}$ as in \refsec{sec:theoreticalBackground:scattering:scatteringTheory}, and the definition of the classical electron radius
\begin{align}
  r_e \eq \frac{e^2}{4\pi \epsilon_0 m c^2} \approx 2.8 \, \unit{fm}
\end{align}
the electric field reads
\begin{align}
  \vec{E} (\vec{r}, t) \eq& \vec{E}_i(\vec{r}, t) + \hat{e}_i E^0   \frac{e^{ik r}}{r} r_e \underbrace{\int \dint V^\prime e^{-i \vec{q} \cdot  \vec{r}^\prime}  \rho_e (\vec{r}^\prime)}_{\eq f_a(\vec{q})},
\end{align}
where the Fourier transform of the electron density is identified as the atomic form factor.
Equivalent to the case of neutron scattering, we identify the scattering solution as an integral of spherical waves generated at every point in space proportional to the electron density and proportional to the incoming wave.

The differential cross-section that is defined for the intensity, which is $I \propto |\vec{E}|^2$, is then finally given by
\begin{align}
  \frac{\dint \sigma}{\dint \Omega} \eq |\hat{e}_i \cdot \hat{e}_f|^2 r_e^2 |f_a(\vec{q})|^2,
\end{align}
which is known as Thomson scattering and includes an additional polarization factor $|\hat{e}_i \cdot \hat{e}_f|^2$, where $\hat{e}_f$ is the direction of detection.
The polarization factor depends on the geometry of the experiment and the source that is used \cite{AlsNielsen_2011_Eleme}
\begin{align}
  |\hat{e}_i \cdot \hat{e}_f|^2 \eq \begin{cases}
    1 & \textsf{synchrotron: vertical scattering plane} \\
    \cos^2(\psi) & \textsf{synchrotron: horizontal scattering plane}\\
    \frac{1}{2} (1+\cos^2(\psi)) & \textsf{unpolarized source}
  \end{cases}
\end{align}
\end{document}