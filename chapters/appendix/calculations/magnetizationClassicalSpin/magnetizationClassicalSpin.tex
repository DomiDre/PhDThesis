\providecommand{\main}{../../../..}
\documentclass[\main/dresen_thesis.tex]{subfiles}

\begin{document}
\section{Magnetization of a Superparamagnetic Spin}\label{ch:appendix:calculations:magnetizationClassicalSpin}
In this work, nanoparticles that comprise of a ferromagnetic domain of $\gg 1000$ collinear spins are commonly discussed.
When many spins couple to form a super spin, the quantum mechanic results converge with the results from classical calculations for large spin quantum numbers.
Therefore, in the following the magnetization behaviour of a super spin is discussed in the context of classical physics.

First, the behaviour of a paramagnetic spin with magnetic moment $\vec{\mu}$ is discussed.
The spin is under the influence of an external magnetic field $\vec{H}$ and the temperature $T$.
The magnetic field tries to align the magnetic moment to minimize the Zeeman potential
\begin{align}
  V = - \mu_0 \vec{\mu} \cdot \vec{H}.
\end{align}

As the temperature $T$ is fixed, the canonical partition function has to be evaluated to determine the equilibrium state that maximizes the entropy for the given temperature.
It is given by integrating over all possible orientations of $\vec{\mu}$.
When $\theta$ is set as the angle between $\vec{\mu}$ and $\vec{H}$ this is evaluated by
\begin{align}
  Z \eq& \int_0^{2\pi} \dint \phi \int_0^\pi \dint \theta \sin(\theta) \exp \biggl(\frac{\mu \mu_0 H}{k_B T} \cos(\theta) \biggr)\\
  \eq& 4 \pi \frac{k_B T}{ \mu \mu_0 H} \sinh \biggl( \frac{\mu \mu_0 H}{k_B T} \biggr).
\end{align}
Then the free energy is given by
\begin{align}
  F \eq& -k_B T \log(Z)\\
  \eq& -k_B T \log \biggl( 4 \pi \frac{k_B T}{ \mu \mu_0 H}  \sinh \biggl( \frac{\mu \mu_0 H}{k_B T} \biggr) \biggr),
\end{align}
and the expectation value of the magnetization
\begin{align}
  \tilde{M} \eq& - \frac{1}{\mu_0 } \frac{\partial F}{\partial H}\\
  \eq& \frac{k_B T}{\mu_0 } \frac{\partial}{\partial H} \log \biggl( 4 \pi \frac{k_B T}{ \mu \mu_0 H}  \sinh \biggl( \frac{\mu \mu_0 H}{k_B T} \biggr) \biggr)\\
  \eq& \frac{k_B T}{\mu_0 } \Biggl(
    \underbrace{\frac{\partial}{\partial H} \log \biggl( 4 \pi \frac{k_B T}{ \mu \mu_0} \biggr)}_{\eq 0}
  + \underbrace{\frac{\partial}{\partial H} \log \biggl( \frac{1}{H} \biggr)}_{\eq - \frac{1}{H}}
  + \underbrace{\frac{\partial}{\partial H}\log  \biggl(  \sinh \biggl( \frac{\mu \mu_0 H}{k_B T} \biggr) \biggr) \Biggr)}_{\eq \frac{\mu \mu_0}{k_B T} \coth \biggl( \frac{\mu \mu_0 H}{k_B T} \biggr) \biggr)}\\
  \eq& \mu \biggl(\coth \biggl( \frac{\mu \mu_0 H}{k_B T} \biggr) - \frac{k_B T}{\mu \mu_0 H} \biggr).
\end{align}
This function is known as the Langevin function and describes the dependence of the magnetization for a single magnetic moment with respect to temperature and magnetic field.
For a collection of $N$ independent particles the magnetization can be summed up.
Typically, the magnetization is normalized to the volume of magnetic material $V$
\begin{align}
  M \eq \frac{N \mu}{V} \biggl(\coth \biggl( \frac{\mu \mu_0 H}{k_B T} \biggr) - \frac{k_B T}{\mu \mu_0 H} \biggr).
\end{align}
The prefactor is a material constant, known as the saturation magnetization $M_s$.
When the volume is written as collection of the single particle volumes $V \eq N V_p$, the relation between $mu$, $M_s$ and $V_p$ becomes clear as
\begin{align}
  M_s \eq \frac{\mu}{V_p}.
\end{align}
\end{document}