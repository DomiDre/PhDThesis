\providecommand{\main}{../../../..}
\documentclass[\main/dresen_thesis.tex]{subfiles}
\renewcommand{\thisPath}{\main/chapters/appendix/formfactors/cube}
\begin{document}
\section{Cube Model}\label{ch:appendix:formfactors:cube}
In this work, nanocubes are regularly used to study self-assembled nanoparticles on a square lattices.
The form factor amplitude of oriented cubes with edge length $a$ are obtained quickly in Cartesian coordinates by evaluating
\begin{align}
  p^\mathrm{orientated}_\mathrm{cube} (\vec{q})
  \eq & \int_{-a/2}^{a/2} \dint x e^{-i q_x x} \int_{-a/2}^{a/2} \dint y e^{-i q_y y} \int_{-a/2}^{a/2} \dint z e^{-i q_z z} \\
  \eq & \sinc \biggl(\frac{q_x a}{2} \biggr)\sinc\biggl(\frac{q_y a}{2} \biggr)\sinc\biggl(\frac{q_z a}{2} \biggr)
\end{align}
In a dispersion the cubes are however randomly orientated. Therefore, an additional integration over all orientations of the form factor has to be performed. Notice that this is performed on the form factor and not the amplitude, as the particles are considered to scatter independently from one another. For this purpose $\vec{q}$ is written in spherical coordinates by
\begin{align}
  q_x \eq& q \cos(\phi) sin(\theta),\\
  q_y \eq& q \sin(\phi) sin(\theta),\\
  q_z \eq& q \cos(\theta),\\
\end{align}
and the integration over all orientations is subsequently performed
\begin{align}
  P_\mathrm{cube} (q) \eq \frac{1}{4 \pi} \int_0^{2\pi} \dint \phi \int_0^{\pi} \dint \theta \sin(\theta) \biggl| p^\mathrm{orientated}_\mathrm{cube} (\vec{q}) \biggr|^2.
\end{align}
There is no analytical solution to this integral, therefore it has to be evaluated numerically. This can efficiently be done by applying a Gauss-Legendre quadrature rule. Here the integrals are replaced by sums 
\begin{align}
  \int_0^{2\pi} f(x) \dint x \approx \pi \sum_{i=1}^n w_i f(\pi (x_i + 1)),\\
  \int_0^{\pi} f(x) \dint x \approx \frac{\pi}{2} \sum_{i=1}^n w_i f\biggl(\frac{\pi}{2} (x_i + 1) \biggr),
\end{align}
where $x_i$ and $w_i$ depend on the order of the quadrature rule and can be quickly looked up from tables to orders up to $n = 100$, where an order of $n = 10$ is sufficient for convergence in every relevant case.
\end{document}