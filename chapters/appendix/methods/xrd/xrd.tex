\providecommand{\main}{../../../..}
\documentclass[\main/dresen_thesis.tex]{subfiles}

\begin{document}
  \section{X-Ray Diffraction (XRD)}
    \label{app:methods:xrd}
    X-ray diffraction is a standard laboratory technique to study the crystal structure of a sample on the atomic length scale.
    The presented XRD patterns in this thesis have been performed in collaboration with the group of Daniel Nižňanský from the Department of Inorganic Chemistry at the Charles University in Prague.
    A PANalytical X'Pert PRO diffractometer was used equipped with a secondary monochromater and a PIXcel detector and the data is acquired over a range $2 \theta \eq 5^\circ \ldots 80^\circ$ with a step size of $0.03^\circ$ using Cu K$\alpha$ radiation with a wavelength of $lambda \eq 1.54 \unit{\angstrom}$.
    This corresponds to a scattering vector range of $q \eq 1 \ldots 5 \unit{\angstrom^{-1}}$.

    To evaluate the obtained XRD profile, a Rietveld analysis is performed using the FullProf Suite \cite{Rodriguez_1993_Recen}, where an expected crystal phase can be modeled and its respective parameters refined.
    While performing the analysis, the order of adding parameters to the refinement is in all cases the same, where first the global parameters as the scale factor and background are estimated, and then the lattice parameter, peak shape and temperature displacement parameters are varied.
    The peak shape parameters are essentially determined by a calibration measurement of a LaB6 reference measurement, using a pseudo-Voigt profile function.
    As the finite crystallite size can lead to an additional broadening of the peaks, an additional Lorentzian broadening is allowed during the refinement.

    FullProf models the diffraction data by calculating for each given angle $\theta_i$
    \begin{align}
      y_{\mathrm{m}, \, i} \eq
        \sum_{\varphi} S_{\varphi}
        \sum_{\vec{h}}
          I_{\varphi, \, \vec{h}}
          \Omega ( \theta_{i} - \theta_{\varphi, \, \vec{h}})
        + b_i.
    \end{align}
    Here, the sums run over the phases of the model $\varphi$, which have a scale factor $S_{\varphi}$ each, and the respective Bragg reflections $\vec{h}$.
    $\Omega$ is the peak profile function and the background model is accounted for by $b_i$.
    The peak intensities $I_{\varphi, \, \vec{h}}$ include the structure factor $F^2$, the absorption correction $A$, the Lorentz, polarization and multiplicity factors $L$, possibly a preferred orientation function $P$ and other special corrections that may be added to the model $C$
    \begin{align}
      I_{\varphi, \, \vec{h}} \eq (F^2 A L P C)_{\varphi, \, \vec{h}}.
    \end{align}

    By including the Lorentzian broadening, the Rietveld analysis provides information about the average size of the crystallites $L$ in the sample.
    This can be extracted by the Scherrer equation
    \begin{equation}
      L \eq \frac{K \lambda}{\beta \cos(\theta)},
    \end{equation}
    where $\lambda$ is the X-ray wavelength, $\beta$ the breadth of a peak, $\theta$ the scattering angle of the peak and $K$ the shape factor, which depends on the crystal shape, the definition of what is taken as peak breadth and the reflex indices \cite{Langford_1978_Scher}.
    In FullProf, $K$ is set to 1 and $\beta$ is determined from the pseudo-Voigt line profile according to the formula given by De Keijser \cite{DeKeijser_1982_Useof}.

    To quantify the quality of the Rietveld refinement, multiple figure of merits are presented.
    During the refinement, FullProf minimizes the reduced $\chi^2$ value defined by
    \begin{align}
      \chi^2 \eq \frac{1}{n-p} \sum_{i \eq 1}^n \biggl(\frac{y_i - y_{\mathrm{m},\,i}}{\sigma_i}\biggr)^2,
    \end{align}
    where $y_i$ are the $n$ experimental values, $\sigma_i$ the respective standard deviations and $p$ the number of parameters

    Furthermore, FullProf provides more agreement factors which can be considered and are stated for completeness and considerations.
    It provides profile factor with varied weighting, to assess the agreement between data and model, namely the profile factor
    \begin{align}
      R_\mathrm{p} \eq 100 \frac{\sum_{i\eq1}^n |y_i - y_{\mathrm{m},\,i}|}{\sum_{i\eq1}^n y_i},
    \end{align}
    the weighted profile factor
    \begin{align}
      R_{\mathrm{wp}} \eq 100 \Biggl( \frac{\sum_{i \eq 1}^n \biggl(\frac{y_i - y_{\mathrm{m},\,i}}{\sigma_i}\biggr)^2}{\sum_{i\eq1}^n \frac{y_i^2}{\sigma_i^2}} \Biggr)^{1/2},
    \end{align}
    and the expected profile factor
    \begin{align}
      R_{\mathrm{exp}} \eq 100 \Biggl( \frac{n-p}{ \sum_{i\eq1}^n \frac{y_i^2}{\sigma_i^2} } \Biggr)^{1/2}.
    \end{align}
    As well as the crystallographic $R_F$ factor for a given phase, which is defined as a sum over all Bragg reflexes with
    \begin{align}
      R_F \eq 100
        \frac{\sum_{\vec{h}} |F_{\mathrm{obs},\, k} - F_{\mathrm{calc},\,\vec{h}}|}
             {\sum_{\vec{h}} F_{\mathrm{obs},\, \vec{h}}}.
    \end{align}
    where the $F_{\mathrm{obs},\, \vec{h}}$ and $F_{\mathrm{calc},\,\vec{h}}$ is evaluated in FullProf by
    \begin{align}
      F_{ \mathrm{obs}, \, \vec{h} } &\eq
      \sqrt{ \frac{I_{\vec{h}}}{L_{\vec{h}}} \sum_{i} \Omega(\theta_i - \theta_{\vec{h}}) \frac{y_{i} - b_i}{y_{\mathrm{m},\,i} - b_i} } \\
      F_{\mathrm{calc},\,\vec{h}} &\eq \sqrt{\frac{I_{\vec{h}}}{L_{\vec{h}}}}
    \end{align}
\end{document}