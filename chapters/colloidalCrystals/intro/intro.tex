\providecommand{\main}{../../..}
\documentclass[\main/dresen_thesis.tex]{subfiles}

\begin{document}
  Having discussed long-range ordered mono and double layers of nanocubes, the next natural step is to study long-range ordered three dimensional nanostructures.
  Vertical deposition is a preparation method developed in the last two decades that is known to produce three-dimensional long-ranged ordered colloidal crystals \cite{Jiang_1999_Singl, Ye_2000_Selfa}.
  Here, a glass or silicon substrate is placed vertically in a colloidal dispersion, which is evaporated.
  The method was inspired by earlier studies on colloidal photonic crystals prepared by gravity sedimentation of silica particles \cite{Lopez_1997_Photo, Miguez_1998_Contr} and by studies of monolayers prepared of polystyrene particles by a technique that relies on capillary forces \cite{Denkov_1992_Mecha}.
  The vertical deposition technique relies on both the capillary forces at the solvent-substrate-air interface and the convex flow induced by the solvent evaporation \cite{Kuai_2004_Highq}.
  Colloidal crystals prepared by this deposition technique can have a long-range order that extends over a centimeter for submicrometer-sized particles \cite{Ye_2000_Selfa}.
  The sample thickness can be controlled by the particle size and the particle concentration in dispersion \cite{Ye_2000_Selfa}.
  Over the last decade, this technique was transferred to various types of particles including hematite nanocubes in water \cite{Meijer_2012_Selfa} and core-shell w\"ustite-magnetite nanocubes in toluene \cite{Wichmann_2016_Synth}, where in each case long-range ordered three-dimensional colloidal crystals were obtained.
  \\

  The colloidal crystals from nanocubes showed in each case a face-to-face orientation in the ordered structures and thereby a nanoparticle shape-induced order in the crystal.
  In the case of nanometer-sized iron oxide nanocubes evaporated from toluene \cite{Wichmann_2016_Synth}, a body-centered tetragonal ($bct$) unit cell for the nanoparticle arrangement was determined from GISAXS, similar to the superstructure observed in mesocrystals of iron oxide nanocubes with a similar size \cite{Wetterskog_2016_Tunin}.
  Mesocrystals are an active field of research that are extensively discussed in literature for iron oxide nanocubes \cite{Agthe_2017_Follo, Josten_2017_Super, Disch_2011_Shape}.
  The study by Wetterskog \etal \cite{Wetterskog_2016_Tunin} shows that depending of the nanoparticle size and morphology two growth orientations of the $bct$ structure in iron oxide mesocrystals can be identified, one with $a\eq b < c$ and a $[001]$ growth direction parallel to the substrate normal, and one with a $[101]$ growth direction.
  Nanocubes in the size of around $9.6 \unit{nm}$ with sharper edges preferred the $[001]$ direction, whereas $12.6 \unit{nm}$ and rounder edges preferred the growth along $[101]$ \cite{Wetterskog_2016_Tunin}.
  It is observed for the $bct$ structures that the dimension of $a$ and $c$ are such that $\sqrt{c^2 + a^2} \eq 2c \rightarrow c \eq \sqrt{3} a$, which allows for coherent grain boundaries between the two growth modes.
  Furthermore, for even larger nanocubes of $13.6 \unit{nm}$, the GISAXS reveals a transition of the $[001]$ mode from a $bct$ structure to a simple cubic unit cell.

  The extension of the studied mesocrystals is typically in the order of $0.5 - 1 \unit{\musf m}$ in height and $1 - 3 \unit{\musf m}$ in width \cite{Wetterskog_2016_Tunin}.
  A problem that arises from the island structure of the samples is, however, that they are inaccessible to reflectometry studies due to the high surface roughness.
  Therefore the magnetic profile of such systems can not be easily studied by PNR.
  \\

  The vertical deposition technique provides extended colloidal crystals with a long-range order comparable to mesocrystalline islands as observed using GISAXS.
  These colloidal crystals of iron oxide nanoparticles represent a suitable model system for PNR studies aiming at accessing the depth-resolved magnetic profile.
  On this basis, the magnetic ground state within the colloidal crystal can be discussed and thereby effects emerging from the combination of dipolar nanoparticle interaction and the three-dimensional long-range order can be studied.
\end{document}