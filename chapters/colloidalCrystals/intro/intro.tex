\providecommand{\main}{../../..}
\documentclass[\main/dresen_thesis.tex]{subfiles}

\begin{document}
  Having discussed long-range ordered mono and double layers of nanocubes, the next natural step is to study long-range ordered three dimensional nanostructures.
  The assembly of ordered nanostructures from nanocubes is extensively discussed in recent literature in the form of mesocrystals that are distributed as islands on a substrate and prepared by drop casting \cite{Wetterskog_2016_Tunin, Agthe_2017_Follo, Josten_2017_Super, Disch_2011_Shape}.
  A problem that arises from the island structure of the samples is, however, that they are inaccessible to depth-sensitive probing methods such as reflectometry due to the high surface roughness.
  Therefore no polarized neutron reflectometry can be performed and the magnetic profile of such systems can not be easily studied but has to be deduced from the macroscopic magnetization.
  \\

  A way to prepare homogeneous layers of long-range ordered nanocubes is found by slowly evaporating a nanoparticle dispersion, with a substrate standing upright in the liquid during the evaporation \cite{Wichmann_2016_Synth}.
  Structural characterizations of such prepared layers \cite{Wichmann_2016_Synth}, show that an equivalent degree of long-range order is visible by GISAXS as it is seen for the literature known mesocrystals.
  As such prepared samples, which will be called colloidal crystals in the following, additionally do not show a separated island structure but a connected thin layer with a homogeneous surface, it can be hoped that a reflectometry study reveals signs of a complex magnetic structure within the long-range ordered sample emerging from dipolar interparticle interaction.
\end{document}