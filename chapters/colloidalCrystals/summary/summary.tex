\providecommand{\main}{../../..}
\documentclass[\main/dresen_thesis.tex]{subfiles}

\begin{document}
  \label{sec:colloidalCrystals:colloidalCrystals:summary}
  The final chapter of this thesis approached a method to prepare three-dimensional long-range ordered layers of nanocubes and discusses the structural and magnetic characterization of three specimen with varied thickness.

  For this purpose, iron oxide nanocubes are prepared following a literature known synthesis route from iron oleate and characterized structurally and magnetically.
  A mean edge length of $12.26(4) \unit{nm}$ and a size distribution of $7.2(1) \%$ are determined from fitting a superball form factor to both SAXS and SANS data.
  The resulting superball has an exponent of $2.15(7)$, which represents cubes with rounded edges.
  Measurements of the nanocubes to different points in time after the synthesis show a proceeding oxidation process within the first months after the synthesis.
  Whereas by SANS measured three months after the synthesis a core-shell morphology with a $3.3(2) \unit{nm}$ thick shell of magnetite around w\"ustite core is identified, a SAXS measurement performed eight months after the synthesis is characterized by a homogeneous phase.
  The LeBail analysis of the XRD data measured one year after the synthesis by a combination of an inverse spinell and w\"ustite phase, yields a w\"ustite lattice constant of $4.18 \unit{\angstrom}$ that strongly deviates from the literature value ($4.33 \unit{\angstrom}$), but is instead on the order of $\tfrac{1}{2} a_\mathrm{inv.\,spinell}$ (with $a_\mathrm{inv.\,spinell} \eq 8.38 \unit{\angstrom}$).
  This indicates also a complete oxidation of the nanocubes at this point, and that the two phases actually result from a discrepancy of the tetrahedral and octahedral sublattices in the inverse spinell phase due to anti-phase boundaries in the bulk of the nanoparticle from a topotaxial oxidation \cite{Wetterskog_2013_Anoma}.

  The on-going oxidation is also observed in the comparison of the magnetic characterization of VSM performed one week after synthesis and SANSPOL results obtained three months after synthesis.
  Where the VSM measurement results in an average particle magnetization of $155(1) \unit{kA\, m^{-1}}$, the SANSPOL shows that after time passage an average magnetization of $371(13) \unit{kA \, m^{-1}}$ is reached, showing that the nanocubes reach a stronger magnetization with time.
  This shows that the magnetic state of the nanocubes is initially not well defined as they are oxidation, which makes a direct comparison of single nanoparticle properties determined early-on from dispersion and of prepared nanostructures difficult.
  \\

  The nanocubes were used to prepare highly ordered colloidal crystals, where three colloidal crystals are discussed for their structural and magnetic properties.
  The samples are prepared by slowly evaporating dispersions with varied nanoparticle concentrations with a silicon substrate standing vertically inside.
  After complete evaporation of the solvent within $2 - 3 \unit{days}$ a thin film of nanocubes is obtained on the wafer surface.
  From SEM, the homogeneous layer surface and the high three-dimensional long-range order is observed.
  Three different thickness are observed, for the first the contrast between the nanoparticle layer and substrate is not good enough to estimate the thickness, the second spans over approximately $200 \unit{nm}$ and the last close to $1 \unit{\musf m}$.
  The best structural ordering is observed in the intermediate sample CC-Fe-0.37, where both the top-view and cross-sectional view show a clear ordering across the whole sample thickness.

  This intermediate thickness sample is studied by GISAXS to determine the three-dimensional mesocrystalline structure.
  Using a orthorhombic unit cell, the lattice constant are determined from the reflection positions to $a \eq 16.6(2) \unit{nm}$, $b\eq 33.9(3) \unit{nm}$ and $c\eq 51.3(5) \unit{nm}$.
  It is found that the ratios of $b/a \approx 2.0$ and $c/a \approx \sqrt{9.6}$ shows parallels to literature known GISAXS studies on mesocrystals that are prepared from drop casting of iron oxide nanocubes dispersions at high concentration on a substrate \cite{Wetterskog_2016_Tunin}.
  The nanoparticle superstructure in that study are partially described by a body-centered tetragonal unit cell with a [101]-orientation, which can be written as a orthorhombic unit cell with ratios of $b/a \eq 2$ and $c/a \eq \sqrt{12}$.
  The lower ratio of $c/a$ indicates that not a perfect [101]-orientation is observed but a slight deviation in the tilt angle may be present in the sample.

  Using XRR, a correlation peak is clearly visible in every reflectivity, which corresponds to a length scale of $L \eq 18.0(5) \unit{nm}$ and is associated with the face diagonal of the rotated nanocubes.
  The ratio to the $c$ lattice observed in GISAXS is $c/L \approx 2.85$, which is connected to the $bct$ unit cell when an interparticle corner-to-corner distance of $7.5 \unit{nm}$ along the vertical is given.
  As no Kiessig fringes are visible in any measured reflectivity, the sample thickness of the colloidal crystals can not be determined from XRR.
  The missing Kiessig fringes for the samples with an expected thickness $\le 200 \unit{nm}$ might be due to large thickness variation in the sample or a high surface roughness, which effectively blurs the fringes out.
  For the thicker sample in the order of $1 \unit{\mu m}$, the instrumental resolution should additionally be a limiting factor.
  \\

  The magnetic properties of the colloidal crystals are discussed by vibrating sample magnetometry and polarized neutron reflectometry.
  While the crystals show the same magnetic behaviour at $300 \unit{K}$, the temperature-dependent and low-temperature magnetization measurements show deviations.
  For the sample CC-Fe-0.37 the temperature dependent magnetization shows a shift of the peaks towards lower temperatures both for the zero-field and field cooled case in comparison to CC-Fe-0.25 and CC-Fe-0.5.
  This indicates that if the shift is connected to dipolar interparticle interaction, they are weaker in this sample as according to literature stronger dipolar interactions are generally associated with a shift towards higher blocking temperatures \cite{Morup_2010_Magne, Pauly_2012_Sized, Otero_2000_Influ}.
  Comparing the field-cooled and zero-field cooled hysteresis at $10 \unit{K}$ for CC-Fe-0.37 also shows a $5(1) \unit{mT}$ shift of the hysteresis towards negative fields and an increase of the remanent magnetization value by $25 \%$ for the field-cooled case.
  It is probable that this effect is a single-particle property and not an interaction effect, as a similar effect is also observed in literature for non-interacting nanocubes with anti-phase boundaries in the particle volume \cite{Wetterskog_2013_Anoma}.

  The polarized neutron reflectometry does not reveal any variations across the three samples.
  Correlations peaks with a period corresponding to a length scale of $16(1) \unit{nm}$ can be qualitatively observed and are in comparable order of magnitude as the length scale obtained from XRR.
  The reflectivities show a splitting in the saturated cases as well as in remanence.
  Comparing the zero-field and field cooled reflectivites, a stronger splitting between the two neutron channels are observed for the field cooled case in remanence.
  This is connected to the observed change of the hysteresis in vibrating sample magnetometry, where the field cooled and zero-field cooled hysteresis shows an increase of the magnetization by $25 \%$.
  \\

  Concludingly, from the obtained data an approach to a model of the structure of the colloidal crystal that is similar to that observed for mesocrystals from iron oxide nanocubes is presented.
  Even though deviations in the vertical lattice constant are clearly visible, a close resemblance can be observed and it is concluded that the real structure corresponds to a slightly more tilted version.
  The magnetism of the three colloidal crystals resembles each other closely in both VSM and PNR, where the only significant difference across the three samples is a shift of the blocking temperature for the intermediate sample.
  From the qualitative discussion and observed effects, it is not possible to directly deduce on interparticle effects.
  For a deeper evaluation in a future study, the characterization of the low temperature magnetic properties of the non-interacting nanoparticles is necessary, as well as the quantitative evaluation of the observed colloidal crystal magnetic properties in accordance with the determined nanoparticle superstructure.
\end{document}