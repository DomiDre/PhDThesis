\providecommand{\main}{../../..}
\documentclass[\main/dresen_thesis.tex]{subfiles}

\begin{document}
  The final chapter of this thesis approached a method to prepare three-dimensional long-range ordered layers of nanocubes and discusses the structural and magnetic characterization of three specimen with varied thickness.

  For this purpose, iron oxide nanocubes are prepared following a literature known synthesis route from iron oleate and characterized structurally and magnetically.
  A mean edge length of $12.26(4) \unit{nm}$ and a size distribution of $7.2(1) \%$ are determined from fitting a superball form factor to both SAXS and SANS data.
  The resulting superball has an exponent of $2.15(7)$, which represents cubes with rounded edges.
  Furthermore, a core-shell w\"ustite-magnetite structure is observed by both XRD and SAS.
  The thickness of the magnetite shell is strongly varying across the experiments, which were measured with longer time gaps in between and therefore reproduce the on-going oxidation of the nanocubes with time.
  This is then represented in the comparison of the magnetic characterization of VSM and SANSPOL results.
  Where the VSM right after the synthesis results in an average particle magnetization of $155(1) \unit{kA\, m^{-1}}$, the SANSPOL shows that after time passage an average magnetization of $387(13) \unit{kA \, m^{-1}}$ is reached, showing that the nanocubes reach a stronger magnetization with time.
  But this also shows that the magnetic state of the nanocubes is not well defined as they are changing uncontrolled in phase, which makes a direct comparison of single nanoparticle properties determined from dispersion and of prepared nanostructures difficult.
  \\

  The nanocubes were used to prepare highly ordered colloidal crystals, which range from a thin layer to a micrometer thick sample.
  The three-dimensional structure is determined by GISAXS on the intermediate sample, where a body-centered tetragonal unit cell, known in literature for mesocrystals produced from similar iron oxide nanocubes, is found to index the observed peak positions.
  The intermediate sample is chosen for GISAXS as it exhibits qualitatively the best order in SEM in the top-view and cross-sectional view.

  From low-temperature PNR after zero-field and field cooling, no substantial difference can be observed qualitatively between the three samples.
  In the magnetic characterization the sample that is assumed to have the best structural ordering also shows a shift in peak positions in the zero-field and field cooled temperature-dependent magnetization measurements with respect to the other two samples.
  It is therefore concluded that the higher structural order reduces the energy barrier to overcome the iron oxide nanocubes magnetocrystalline anisotropy and is therefore a sign of dipolar interparticle interaction.
  \\

  Concludingly iron oxide colloidal crystals are a sample, where interparticle interaction show measurable effects.
  The use of phase unstable nanocubes as building blocks makes a quantitative and direct evaluation, however, difficult and in a future study, great care should be taken to use nanocubes within a well defined magnetic state.
  Additionally a study including a direct comparison of the structural correlations of the three-dimensional structure and the varying magnetic properties can provide further insight into the nature of the emergent state from interparticle interactions.

\end{document}