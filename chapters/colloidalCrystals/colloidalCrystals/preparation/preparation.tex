\providecommand{\main}{../../../..}
\documentclass[\main/dresen_thesis.tex]{subfiles}
\begin{document}
  \label{sec:colloidalCrystals:layers:preparation}
  In vertical deposition, the variable parameters are the concentration of the dispersion, the choice of the solvent, the ambient temperature and pressure and the inclination angle of the silicon wafer within the dispersion.
  By multiple initial attempts, it was found that homogeneous colloidal crystals are formed when a long time scale for the solvent evaporation is ensured and any additional human interference during the process is avoided.

  Three colloidal crystals were prepared by the vertical deposition method: CC-Fe-0.25, CC-Fe-0.37 and CC-Fe-0.50.
  For the three samples, the concentration was varied to $0.25 mg \, mL^{-1}$, $0.37 mg \, mL^{-1}$ and $0.50 mg \, mL^{-1}$ respectively, whereas all remaining parameters are kept equal.
  The solvent is chosen to toluene and the beaker used for vertical deposition is tightly covered with aluminum foil, where only a small hole for evaporation is punctured, to tune the time scale for the vertical deposition to $1 - 2 \unit{weeks}$.
  A closed fume hood in a low-vibration room is used as place to evaporate the solvent to minimize human interference with the crystallization process and thereby obtain a homogeneous coverage.
  By this, the ambient conditions are set to room temperature ($\approx 22\unit{^\circ C}$) and to a slight under pressure given by the air circulation within the fume hood.
\end{document}