\providecommand{\main}{../../../..}
\documentclass[\main/dresen_thesis.tex]{subfiles}

\begin{document}
  \label{sec:colloidalCrystals:colloidalCrystals:summary}
  Three colloidal crystals prepared from Ol-Fe-C are discussed for their structural and magnetic properties.
  The samples are prepared by slowly evaporating dispersions with varied nanoparticle concentrations with a silicon substrate standing vertically inside.
  After complete evaporation a thin film of nanocubes is found on the wafer surface.
  From SEM the homogeneous layer surface and the high three-dimensional long-range order is observed.
  Three different thickness are observed, where one is close to a single layer, the second spans over $200 \unit{nm}$ and the last close to a micrometer.
  The best structural ordering is observed in the intermediate sample CC-Fe-0.37, where both the top-view and cross-sectional view show a clear ordering across the whole sample thickness.

  This intermediate thickness sample is studied by GISAXS to determine the three-dimensional mesocrystalline structure.
  It is found that the observed peaks can be indexed analogue to literature known mesocrystals that are prepared from drop casting iron oxide nanocubes dispersions at high concentration on a substrate \cite{Wetterskog_2016_Tunin}.
  The nanoparticle superstructure is thereby described by a body-centered tetragonal unit cell with a [101]-orientation.
  The lattice constant of an equivalent orthogonal unit cell is determined from the peaks to $a \eq 16.4 \unit{nm}$, $b\eq 2a$ and $c\eq \sqrt{12} a$.

  Using XRR, the correlation peak from the nanocube size is clearly visible in every reflectivity.
  As however no Kiessig fringes are visible in any measured reflectivity, it can be concluded that the samples exhibit a large surface roughness that is connected to the tilted orientation of the nanocubes.
  It can be observed that with increasing sample thickness, the critical edges bends downward and reduces to smaller $q_z$ values, which is connected for one to a stronger absorption of the samples with the increasing iron oxide material.
  For the other, the substrate SLD becomes less relevant for the critical edge with increasing sample thickness, but is determined by the average SLD of the colloidal crystal layer, which can be assumed to be lower than the silicon SLD from the made observation.
  The same holds true for the neutron reflectivities that were obtained, where the critical edge also reduces with increasing colloidal crystal thickness.
  For neutrons, the sample absorption by iron oxide is less dominant as for X-rays, therefore the second effect is here only to be held accountable.
  \\

  The magnetic properties of the colloidal crystals is discussed by vibrating sample magnetometry and polarized neutron reflectometry.
  While the crystals show the same magnetic behaviour at $300 \unit{K}$, the temperature-dependent and low-temperature magnetization measurements show deviations.
  For the sample CC-Fe-0.37 with the best structure, as determined by SEM, the temperature dependent magnetization shows a shift of the peaks towards lower temperatures, which suggests a lowering of the energy barrier to overcome the magnetocrystalline anisotropy due to the collective behaviour of the nanocubes.

  The polarized neutron reflectometry on the other hand shows no strong variations across the three samples.
  Due to the high surface roughness of the samples, only correlations peaks from the nanoparticle size can be made out.
  Comparing the zero-field and field cooled reflectivites, different magnitudes of the splitting between the two neutron channels are observed, which is associated with the exchange bias effect of the individual nanocubes.

\end{document}