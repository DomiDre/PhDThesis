\providecommand{\main}{../../../..}
\documentclass[\main/dresen_thesis.tex]{subfiles}

\begin{document}
  \paragraphNewLine{Transmission Electron Microscopy}
    The iron oxide nanocubes are investigated by transmission electron microscopy using a Zeiss Leo 902 (\refsec{ch:instruments:laboratoryInstruments:tem}).
    The micrographs were obtained at an acceleration voltage of $120 \unit{kV}$ with a \ch{LaB6} cathode in bright field mode.
    The nanocube edge length is determined from the micrographs by using the Fiji distribution \cite{Schindelin_2012_Fijia} to count 200 nanocube edges, which are binned in a histogram and evaluated by fitting a log-normal distribution as described in \refch{ch:methods:em}.

  \paragraphNewLine{X-Ray Diffraction}
    An X-ray diffraction experiment of Ol-Fe-C was performed in cooperation with the group of Daniel Nižňanský from the Department of Inorganic Chemistry at the Charles University in Prague on an PANanalytical X'Pert PRO, described in \refch{ch:instruments:laboratoryInstruments:xrd}.
    For the measurement the sample is transported in a dry state to the University in Prague, where it is redispersed and subsequently transferred on a glass substrate for measurement.
    The diffractometer is equipped with a Cu-K$\alpha$ source ($\lambda \eq 1.54 \angstrom$) and a $2 \theta \eq 5^\circ \ldots 80^\circ$ has been measured.
    The instrumental resolution is determined from a LaB6 reference measurement (SR 660b, NIST).

    The nanocubes are evaluated by a LeBail fit using the FullProf suite \cite{Rodriguez_1993_Recen} as described in \refch{ch:methods:xrd}.
    The background is estimated manually by selecting 20 background points, from which the background in the measured range is interpolated linearly.
    The data between $5 ^\circ \ldots 15 ^\circ$ is excluded from the refinement due to a strong background coming from the glass substrate in this range.
    As in the previous chapters that discuss nanoparticles synthesized from oleates, two phases are necessary to describe the XRD.
    The experimental data is therefore refined to a combination of an inverse spinell phase (space group $Fd\bar{3}m$, No. 227) and a w\"ustite phase (space group $Fm\bar{3}m$, No. 225) is refined.

  \paragraphNewLine{Small-Angle Scattering}
    The nanoparticle structure of Ol-Fe-C is studied in dispersion using both small-angle X-ray and neutron scattering.
    The SAXS measurement were performed at GALAXI (\refch{ch:lss:galaxi}), where the nanoparticles were dispersed in toluene and filled in borosilicate capillaries (Hilgenberg) with $1.5 \unit{mm}$ diameter and a $0.01 \unit{mm}$ wall thickness.
    A plastic stopper and a glue gun was used to seal the capillaries for the measurement in a vacuum.
    The samples have been measured on the largest ($3.53 \unit{m}$) and shortest ($0.83 \unit{m}$) sample-to-detector distances possible at GALAXI at the Ga-K$\alpha$ wavelength of $\lambda \eq 1.3414 \angstrom$.
    For the background subtraction, a capillary filled with toluene and an empty capillary is measured and the data is caled to absolute units following the protocol described in \refch{ch:methods:saxs}.

    For SANS and SANSPOL, the D33 instrument (\refch{ch:lss:d33}) has been utilized.
    The nanocubes were dried over night at ambient conditions, and then redispersed in toluene-$\mathit{d_8}$ by sonification to minimize the hydrogen content in the sample.
    For the measurement, the dispersions are filled in Hellma quartz cuvettes with a thickness of $2 \unit{mm}$.
    The velocity selector at D33 was set to a wavelength of $6 \unit{\angstrom}$, with a wavelength spread of $10 \%$ (FWHM).
    The sample was measured both at a sample-to-detector distance of $11 \unit{m}$ and $2 \unit{m}$ to cover a large $q$-range.
    The collimation-to-sample distance were $10.3 \unit{m}$ and $5.3 \unit{m}$ respectively
    The Ol-CoFe-C dispersion was measured at ambient conditions without the application of a magnetic field , as well as at magnetic fields of $1.234 \unit{T}$. % $153 \unit{mT}$, $308 \unit{mT}$, $600 \unit{mT}$, $882 \unit{mT}$, and
    The magnetic field is applied by an electromagnet perpendicular to the beam in the horizontal plane.
    The data is reduced using the GRASP software, where the solvent is subtracted and the data is corrected for the polarization efficiency of $99 \%$ and the flipping efficiency of $97 \%$.
    The azimuthal average of the magnetic scattering is obtained from a $20 ^\circ$ sector around the vertical $z$ direction, whereas the nuclear scattering is obtained from averaging the complete detector area from the measurement at zero field.

    For the evaluation of the form factors from the azimuthally averaged small-angle scattering data, the same procedure as described in \refsec{sec:looselyPackedNS:characterization:nanoparticles} for iron oxide nanospheres is performed using a superball core-shell form factor instead of a spherical core-shell form factor.
    The superball form factor is described in detail in \refsec{sec:monolayers:methods:superballFF}.
    For the nanoparticle core a composition of \ch{FeO} is assumed and for the shell an inverse spinell structure of \ch{Fe_{3-$\delta$} O4}, where the composition and lattice constants as estimated from X-ray diffraction are used to determine and fix the scattering length densities.
    The solvent scattering length density of toluene and toluene-\textit{d8}, as well as oleic acid for X-rays and neutrons are used as given in \refsec{sec:monolayers:nanoparticle:structuralCharacterization}.
    
    The total particle size and particle size distribution is determined by the fit of the SAXS data and fixed to the same value for the SANS fit.
    The core/shell fraction of the particle is, however freely fit in both fits as it can not be excluded that the nanocubes oxidized in between the experiments.
    The surfactant shell is ignored in the SAXS fit due to the poor contrast between oleic acid and toluene. For Ol-Fe-C it is set to $1.41 \unit{nm}$ to be in a comparable magnitude as was determined in the previous chapter \refch{ch:monolayers} for cobalt ferrite nanocubes.

  \paragraphNewLine{Vibrating Sample Magnetometry}
    To measure the room temperature magnetic properties of the nanocubes, a ADE EV 7 Vibrating Sample Magnetometer is used.
    For the measurement an empty Teflon pot is first measured to determine the background susceptibility and subsequently, $20 \unit{\musf L}$ of the stock solution Ol-Fe-C, which contain approximately $0.75 \unit{mg}$ of nanocubes is dried in the pot.
    The dry sample is then measured in a range of $\pm 2 \unit{T}$ at ambient conditions ($295 \unit{K}$).
    % Additionally, the nanocubes that were forcibly oxidized are measured in VSM by the same procedure.
    % Here, the nanocube sample has a mass of approximately $0.17 \unit{mg}$.
\end{document}