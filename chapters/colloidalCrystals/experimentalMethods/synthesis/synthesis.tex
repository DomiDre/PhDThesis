\providecommand{\main}{../../../..}
\documentclass[\main/dresen_thesis.tex]{subfiles}

\begin{document}
  \label{sec:colloidalCrystals:nanoparticle:synthesisOleatesAcAc}
  For the synthesis of iron oxide nanocubes, the synthesis route from iron oleate is chosen.
  In a first step iron oleate is prepared, which is subsequently used in a heating-up synthesis with the aim to obtain homogeneously shaped nanocubes with a small size distribution and strong magnetic properties.
  In the following, the synthesis steps for the oleate preparation and the heating-up synthesis are described.

  % REWRITE THIS!
  \paragraphNewLine{Preparation of Iron Oleate \cite{Hyeon_2003_Chemi, Wetterskog_2014_Preci}}
    A clear solution of sodium oleate is prepared by dissolving $72 \unit{mmol}$ ($2.88 \unit{g}$) of \ch{NaOH} in $20 \unit{mL}$ of each \ch{H2O} and \ch{EtOH} and by adding drop wise $72 \unit{mmol}$ ($25.25 \unit{mL}$) of oleic acid.
    Subsequently $24 \unit{mmol}$ ($6.48 \unit{g}$) of \ch{FeCl3 * 6 H2O} are dissolved in $5 \unit{mL}$ \ch{H2O} and $15 \unit{mL}$ \ch{EtOH} and added to the solution.
    After addition of $80 \unit{mL}$ \ch{H2O} and \ch{EtOH} each, as well as $160 \unit{mL}$ n-hexane, the mixture is held at reflux ($60 \unit{^\circ C}$) for $4 \unit{h}$ under constant strong magnetic stirring.
    After that the mixture is cooled to room temperature, and washed three times in a separatory funnel with $30 \unit{mL}$ \ch{H2O} each.
    Using a rotary evaporator, the remaining n-hexane, ethanol and water is removed yielding approximately $30 \unit{g}$ of iron oleate as a viscous liquid.

  \paragraphNewLine{Preparation of \ch{FeO}/\ch{Fe_{3-$\delta$}O4} Nanocubes from Oleate \cite{Wetterskog_2014_Preci}}
    $10 \unit{mmol}$ ($9.036 \unit{mg}$) of the iron oleate is dissolved in $50 \unit{mL}$ 1-octadecene in a $250 \unit{mL}$ three-neck round-bottom flask.
    Sodium oleate is prepared separately by solving $2.5 \unit{mmol}$ ($100 \unit{mg}$) \ch{NaOH} in $1 \unit{mL}$ of \ch{H2O} and \ch{EtOH}, and adding $2.5 \unit{mmol}$ ($1.127 \unit{mL}$) of oleic acid drop wise.
    The sodium oleate is added to the solution with the oleate and additional $2.5 \unit{mmol}$ ($1.127 \unit{mL}$) oleic acid are added.
    The mixture is heated and held at $120 \unit{^\circ C}$ for one hour under constant magnetic stirring until no water/ethanol condenses on the three-neck round bottom flask.
    A fractionating column is placed on the round-bottom flask and nitrogen is gently bubbled into the mixture.
    Using a temperature gradient of $2.5 \unit{^\circ C min^{-1}}$, the mixture is heated to reflux at approximately $315 \unit{^\circ C}$, where it is held for $30 \unit{min}$.
    After cooling the reaction to room temperature, the particles are precipitated with \ch{EtOAc} and \ch{EtOH}, centrifuged at $8000 \unit{rpm}$ and redispersed in n-hexane until the supernatant is clear.
    The final product is dispersed in n-hexane and stored as highly concentrated dispersion of $37 \unit{mg \, mL^{-1}}$.
    The concentration of the dispersion is determined gravimetrically by measuring the weight of $100 \unit{\musf L}$ dispersion after evaporation.
    This synthesis yields approximately $500 \unit{mg}$ of nanocubes and is referred to as Ol-Fe-C.
  

  % \paragraphNewLine{Oxidation of Nanocubes}
  %   To determine the sensitivity of the nanocubes to oxidation, $3 \unit{mL}$ of the stock solution ($\approx 111 \unit{mg}$ are dried in a $100 \unit{mL}$ three-neck round-bottom flask and dispersed in $50 \unit{mL}$ of cyclooctane.
  %   The dispersion is heated to reflux at $150 \unit{^\circ}$ and held here for $2 \unit{h}$ under constant infusion of pressurized air and without magnetic stirring.
  %   After the dispersion is cooled back to room temperature, the nanocubes are precipitated by the addition of an $50 \unit{mL}$ of ethanol.
  %   After centrifugation at $5000 \unit{rpm}$ for $10 \unit{min}$, the nanocubes are redispersed in $10 \unit{mL}$ of n-hexane.
  %   A gravimetric measurement of the final dispersion gives a particle concentration of $8.3 \unit{mg\, mL^{-1}}$, and therefore a yield of $75 \%$.
  %   The dispersion shows a brown color.
  %   The nanocubes are, however, not long-term stable but precipitate after a short moment and can be redispersed by shaking of the dispersion.
  %   Due to the instability, the original fraction  of the stock solution that was not actively oxidized is used for all further studies if not explicitly mentioned otherwise.
    % To stabilize the dispersion, the nanocubes are precipitated again and redispersed with $25 \unit{\musf L}$ of oleic acid and $5 \unit{mL}$ of \textit{n}-hexane, which removes part of the unstable fraction and comes with a loss of $34 \%$ of the nanocubes.
\end{document}