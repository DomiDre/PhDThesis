\providecommand{\main}{../../../..}
\documentclass[\main/dresen_thesis.tex]{subfiles}

\begin{document}
  \label{sec:colloidalCrystals:nanoparticle:synthesisOleatesAcAc}
  For the synthesis of iron oxide nanocubes, the synthesis route from iron oleate is chosen.
  In a first step iron oleate is prepared, which is subsequently used in a heating-up synthesis with the aim to obtain homogeneously shaped nanocubes with a small size distribution and strong magnetic properties.
  In the following, the synthesis steps for the oleate preparation and the heating-up synthesis are described.
  
  % REWRITE THIS!
  \paragraphNewLine{Preparation of Iron Oleate \cite{Hyeon_2003_Chemi, Wetterskog_2014_Preci}}
    In a first step, a clear solution of sodium oleate is prepared by dissolving $96 \unit{mmol}$ ($3839.71 \unit{mg}$) of \ch{NaOH} in $20 \unit{mL}$ of each \ch{H2O} and \ch{EtOH} and subsequently adding drop wise $96 \unit{mmol}$ ($33.66 \unit{mL}$) of oleic acid under constant stirring.
    Then $12 \unit{mmol}$ ($2855.17 \unit{mg}$) of \ch{CoCl2 * 6 H2O} and $24 \unit{mmol}$ ($6487.10 \unit{mg}$) of \ch{FeCl3 * 6 H2O} are dissolved in $5 \unit{mL}$ \ch{H2O} and $15 \unit{mL}$ \ch{EtOH} each and added to the solution.
    After addition of $80 \unit{mL}$ \ch{H2O} and \ch{EtOH} each, as well as $160 \unit{mL}$ n-hexane, the mixture is held at reflux ($60 \unit{^\circ C}$) for $4 \unit{h}$ under constant strong magnetic stirring.
    Once the mixture is cooled back to room temperature, it is washed three times in a separatory funnel with $30 \unit{mL}$ \ch{H2O} each to remove \ch{NaCl}.
    The remaining n-hexane, ethanol and water is removed using a rotary evaporator.
    In the end, approximately $32 \unit{g}$ of a dark red and highly viscous metal oleate complex is obtained, which is then ready to be used for the nanoparticle synthesis.

  \paragraphNewLine{Preparation of \ch{FeO}/\ch{Fe_{3-$\delta$}O4} Nanocubes from Oleate \cite{Wetterskog_2014_Preci}}
    To prepare nanoparticles, $10 \unit{mmol}$ ($8070 \unit{mg}$) of the cobalt ferrite oleate is dissolved in $50 \unit{mL}$ 1-octadecene within a $250 \unit{mL}$ three-neck round-bottom flask.
    To obtain cubically shaped nanoparticles, sodium oleate is prepared separately by dissolving $2.5 \unit{mmol}$ ($100 \unit{mg}$) \ch{NaOH} in $10$ drops of \ch{H2O} and \ch{EtOH} and adding $2.5 \unit{mmol}$ of oleic acid drop wise while ultra sonificating the mixture.
    The sodium oleate is added to the dissolved oleate together with additional $2.5 \unit{mmol}$ ($0.877 \unit{mL}$) oleic acid.
    The mixture is heated and held at $150 \unit{^\circ C}$ for one hour under constant magnetic stirring until all water and ethanol is evaporated.
    A fractionating column is put on the round-bottom flask and nitrogen is gently bubbled into the mixture.
    Using a temperature controller, the mixture is heated to reflux at approximately $315 \unit{^\circ C}$ with a gradient of $2.5 \unit{^\circ C min^{-1}}$, where it is held for $30 \unit{min}$.
    After cooling the reaction naturally to room temperature, the particles are precipitated with \ch{EtOAc} and \ch{EtOH}, centrifuged at $8000 \unit{rpm}$ and redispersed in n-hexane until the supernatant is clear.
    In the last step the mixture is centrifuged without adding \ch{EtOAc}/\ch{EtOH} and the supernatant fluid is taken as dispersion, where as the precipitate is thrown away as being unstable.
    This synthesis yields approximately $500 \unit{mg}$ nanocubes (yield $\approx 20 \%$) and is referred to in the following as Ol-Fe-C.
\end{document}