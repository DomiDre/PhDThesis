\providecommand{\main}{../../../..}
\documentclass[\main/dresen_thesis.tex]{subfiles}

\begin{document}
  \paragraphNewLine{Scanning Electron Microscopy}
    All colloidal crystals were studied by scanning electron microscopy measured with a Neon Zeiss 40 (\refsec{ch:instruments:laboratoryInstruments:sem}).
    For selected samples, cross-sectional views were obtained by cutting the sample on two opposing sides of the silicon substrate with a diamond cutter and subsequently breaking the wafer downward to obtain a clean breaking line.
    All micrographs are obtained at $5 \unit{kV}$ with the detector for the the backscattering electrons.

  \paragraphNewLine{Grazing-Incidence Small-Angle X-ray Scattering}
    For CC-Fe-0.25 a GISAXS experiment was performed at the GALAXI instrument in the \textsc{Forschungszentrum J\"ulich} (\refsec{sec:lss:galaxi}).
    Incident angle ...
    Wavelength ... 
    Sample-to-detector distance...
    collimation ...
    Evaluation...
    % All spin-coated samples were measured at the beam line BM26B \refsec{ch:lss:BM26B} in the ESRF at a wavelength of $\lambda \eq 1.03 \unit{\angstrom}$.
    % For each sample a measurement was performed at a large sample-to-detector distance of $6.54 \unit{m}$ and at a shorter sample-to-detector distance of $2.90 \unit{m}$.
    % The collimation slit is set to $0.3 \times 0.5 \unit{mm^2}$ for each measurement, and every sample is evaluated at an incident angle of $0.2 \unit{^\circ}$.

    % For both samples, a strip of $0.02 \nm^{-1}$ width along the Yoneda band is integrated and compared to the form factor obtained by small-angle X-ray scattering.
    % The scattered intensity for the nanostructure along the Yoneda band is calculated as product of a structure factor and the form factor
    % \begin{align}
    %   I(q) \eq I_0 S(q) |P(q)|^2 + I_\mathrm{bg},
    % \end{align}
    % where $I_0$ is a scaling factor and $I_\mathrm{bg}$ is a incoherent noise background that is not directly associated with the scattering from the nanoparticles.
    % The used form factor $|P(q)|^2$ is hereby given by using the best fit of the nanoparticles from SAXS as determined in \refsec{sec:looselyPackedNS:nanoparticle:sas}.
    % As no calibration measurement has been performed at BM26B, the data is given in the arbitrary count units of the detector.
    % The structure factor for hard spheres of radius $R_\mathrm{HS}$ and with a packing fraction $\eta$ can be calculated analytically in the Percus-Yervick approximation \cite{Percus_1958_Analy, Wertheim_1963_Exact, Pedersen_1997_Analy} and is given by
    % \begin{align}
    %   S(q) &\eq \frac{1}{1 + 24 \eta \frac{G(2 q R_\mathrm{HS})}{2 q R_\mathrm{HS}} }
    % \end{align}
    % with
    % \begin{align}
    %   \begin{split}
    %     G(x)   &\eq \frac{(1 + 2\eta )^2}{(1 - \eta )^4} \cdot \frac{ \sin(x) - x \cos(x)}{x^2}\\
    %            & + \frac {-6 \eta (1 + \eta / 2)^2}{(1 - \eta )^4} \cdot \frac{2 x sin(x) + (2 - x^2) \cos(x) - 2}{x^3}\\
    %            & + \frac{\eta (1 + 2\eta )^2}{2(1 - \eta )^4} \cdot \frac{-x^4 \cos(x) + 4 [(3 x^2 - 6) \cos(x) +(x^3 - 6 x) \sin(x) + 6]}{x^5}\\
    %   \end{split}
    % \end{align}

  \paragraphNewLine{X-Ray Reflectometry}
    X-ray reflectometry has been measured on the GALAXI instrument at the \textsc{Forschungszentrum J\"ulich} (\refsec{sec:lss:galaxi}) for CC-Fe-0.25, CC-Fe-0.37 and CC-Fe-0.50.

    The wavelength of GALAXI is given by the Ga-K$\alpha$ line with $\lambda \eq 1.3414 \unit{\angstrom}$ and the beam slits for collimation are set to $1 \unit{mm}$.
    For the measurement, the sample is rotated up to an incident angle of $\alpha_i \eq 2^\circ$ in $0.01 ^\circ$ steps and at each angle the detector is moved by a few millimeters to three positions, to remove gaps on the detector images.
    The angle range is split into two parts, where a higher counting time is set for larger incident angles to improve the counting statistics at the lower scattering intensities for higher $q$.
    From $0 ^\circ \ldots 0.4 ^\circ$ each detector image is measured for $5 \unit{s}$, whereas for the range up to $2 ^\circ$ a measurement time of $30 \unit{s}$ is set.
    An overlap range of $0.1 ^\circ$ is set to obtain an overlap of the two angle ranges.
    To account for a fluctuating intensity of the incident beam and varied counting times, each detector image is scaled to it's respective monitor value.

    The obtained detector data are evaluated by defining a region of interest around the specular peak, which moves along the detector with increasing incident angle.
    The region of interest is integrated along the dimension perpendicular to the scattering direction, such that an intensity map with respect to the remaining detector dimension and the incident angle is obtained.
    Using the pixel-splitting rebinning algorithm described in \refapp{ch:appendix:numericalMethods:rebinningPixelSplitting}, the intensity map is transformed and rebinned to a ($\alpha_i - \alpha_o$, $\alpha_i+\alpha_o$) coordinate system, from which the specular intensity is obtained by integrating a box of $0.2 ^\circ$ width around $\alpha_i - \alpha_o \eq 0$.
    Additionally, the data is corrected for diffuse scattering around the specular peak, by integrating boxes in the range $\alpha_i - \alpha_o \eq \pm (0.1 \ldots 0.2)$, and subtracting the diffuse scattering intensity from the specular intensity.


  \paragraphNewLine{Polarized Neutron Reflectometry}
    The three colloidal crystals, CC-Fe-0.25, CC-Fe-0.37 and CC-Fe-0.50 were characterized on the polarized neutron reflectometer MARIA at the MLZ \refsec{sec:lss:maria}.
    The neutron wavelength was selected to $5 \unit{\angstrom}$ and a wavelength spread of $10 \unit{\%}$ (FWHM) is given by the instrumental scientists.
    The collimation and sample slits along the scattering direction are set to $2 \unit{mm}$, the collimation-to-sample distance was $4.1 \unit{m}$ and the sample-to-detector distance was $2.093 \unit{m}$.
    Each sample was measured at a temperature of $10 \unit{K}$ after cooling in near zero field ($0.9 \unit{mT}$ guide field), subsequent at a saturating field of $1.08 \unit{T}$ and again in remanence at guide field.
    After that, each sample was warmed to $250 \unit{K}$, sufficiently above the blocking temperature, and then cooled back to $10 \unit{K}$ while applying the saturating field.
    After field-cooling, the sample was then measured once more in saturation and remanence, which totals in five neutron reflectometry measurements for each sample at $10 \unit{K}$.
    For each case, the reflectivity $R^{+}$, for neutrons polarized parallel to the magnetic field direction, and the reflectivity $R^{-}$, for neutrons that are polarized anti-parallel to it, is measured.
    No polarization analysis is performed for the sake of having higher counting statistics in the two channels.

    For the measurements, the incident angle was increased up to $6 ^\circ$, where the angle step is increased for three ranges.
    The first range up to $1 ^\circ$ has a step of $0.02 ^\circ$ is set.
    For a second range from $0.8 ^\circ$ up to $3.5 ^\circ$ a step of $0.05 ^\circ$ is chosen and for the last range from $2.8 ^\circ$ up to $6 ^\circ$ a step of $0.2 ^\circ$ is set.
    The three ranges overlap intentionally such that they can be stitch together in case the scaling of the data to the monitor would show a disagreement, which was not the case in the executed experiments.
    At the first range, each point is measured for $30 \unit{s}$, whereas for the second $120 \unit{s}$ and the third range $135 \unit{s}$ are set due to the lower scattering intensity and to to obtain better statistics.

    For the data reduction, each detector image is corrected for the detector sensitivity and integrated along the dimension with relaxed collimation.
    By this integration the data is given as a map of the scattered intensity with respect to the detector pixel $x$ along the scattering direction and with respect to the incident angle $\alpha_i$.
    This map is transformed and rotated to ($\alpha_i - \alpha_o$, $\alpha_i+\alpha_o$) coordinates by calculating $\alpha_o$ from
    \begin{align}
      \alpha_o = \alpha_i + \arctan \biggl( \frac{x - x_\mathrm{spec}}{L_\mathrm{SDD}} \biggr),
    \end{align}
    where $L_\mathrm{SDD}$ is the sample-to-detector distance and $x_\mathrm{spec}$ the center pixel where the specular beam is hitting.
    For the transformation and rotation, the data needs to be rebinned in the new coordinate system.
    To avoid the loss of resolution, a pixel-splitting algorithm is used for the rebinning, which is described in \refapp{ch:appendix:numericalMethods:rebinningPixelSplitting}.

    The specular intensity is obtained, by integrating on the transformed data a strip of $0.1 ^\circ$ width around the specular position on the intensity beam.
    The data is further corrected for the diffuse scattering by additionally integrating strips of $0.1 ^\circ$ width in the off-specular region from $\pm (0.1 ^\circ \ldots 0.2)$, which are subtracted from the specular intensity after being rescaled to adjust for a different numbers of contributing bins.

    Finally a footprint correction if performed for the data, where a equidistributed intensity of the beam is assumed.
    Using the mean intensity of $R^{+}$ below the critical angle, the data is rescaled to have the plateau on unity.
    The curve of $R^{-}$ is rescaled using the same factor, to avoid the introduction of systematic errors by using two different scaling factors.


  \paragraphNewLine{Vibrating Sample Magnetometry}
    % Yip

\end{document}