\providecommand{\main}{../../../..}
\documentclass[\main/dresen_thesis.tex]{subfiles}

\begin{document}
  \paragraphNewLine{Scanning Electron Microscopy}
    The prepared colloidal crystals were studied by scanning electron microscopy using a Neon Zeiss 40 (\refsec{ch:instruments:laboratoryInstruments:sem}).
    Cross-sectional views were obtained by cutting the sample on two opposing sides of the silicon substrate with a diamond cutter and subsequently breaking the wafer downward to obtain a clean breaking line.
    All micrographs are obtained at $5 \unit{kV}$ using the detector for the back-scattering electrons.

  \paragraphNewLine{Grazing-Incidence Small-Angle X-ray Scattering}
    For CC-Fe-0.37 a GISAXS experiment was performed at the GALAXI instrument in the \textsc{Forschungszentrum J\"ulich} (\refsec{ch:lss:galaxi}).
    The sample was measured at the incident angles of $0.155 ^\circ$ and $0.205 ^\circ$ with a wavelength of $\lambda \eq 1.341 \unit{\angstrom}$.
    The sample-to-detector distance was set to $1.73 \unit{m}$ and a total integrated time of $5 \unit{h}$ was measured for each incident angle.
    The measurement time is sub-divided into five parts, where each is measured with a vertically shifted detector position to fill the horizontal gaps between the Pilatus 1M detector sub modules.

    To determine the structure of the colloidal crystal, the detector image obtained at the incident angle of $0.205 ^\circ$, the reflection positions are manually determined from the detector data and a orthorhombic unit cell with $a \neq b \neq c$ and $\alpha \eq \beta \eq \gamma \eq 90 ^\circ$ is determined to index the reflections.
    The $q^{hkl}_y, \, q^{hkl}_z$ positions of the orthorhombic unit cell are calculated by
    \begin{align}
      \begin{split}
        \label{eq:colloidalCrystals:layerChar:gisaxs:reflexPosition}
        q^{hkl}_y \eq& 2 \pi \sqrt{\biggl( \frac{h^2}{a^2} + \frac{k^2}{b^2} \biggr)},\\
        q^{hkl}_z \eq& k_{i,\, z} + \sqrt{k_{c,\, z}^2 + \biggl( 2\pi \frac{l}{c} \pm \sqrt{k_{i,\, z}^2 - k_{c,\, z}^2} \biggr)^2},
      \end{split}
    \end{align}
    where for $q^{hkl}_z$ refraction is additionally accounted by the combination of Snell's and Bragg's laws \cite{Disch_2011_Shape}.
    The $z$-component of the incident wave vector $k_{i,\, z}$ is calculated from the incident angle and the critical wave vector $k_{c,\, z}$ can be directly read from the $q_z$ position of the sample's X-ray reflectometry measured at the same wavelength by
    \begin{align}
      \begin{split}
        k_{i,\, z} \eq& \frac{2 \pi}{\lambda} \sin(\alpha_i),\\
        k_{c,\, z} \eq& \frac{1}{2} q_{c,\,z}.
      \end{split}
    \end{align}

  \paragraphNewLine{X-Ray Reflectometry}
    X-ray reflectometry has been measured on the GALAXI instrument at the \textsc{Forschungszentrum J\"ulich} (\refsec{ch:lss:galaxi}) for CC-Fe-0.25, CC-Fe-0.37 and CC-Fe-0.50.

    Using the GALAXI instrument for XRR instead of a dedicated reflectometer, such as the Bruker D8 Advance used for most samples in this thesis, provides in a relatively short time frame a low signal-to-noise ratio and a large dynamic range.
    The wavelength of GALAXI is given by the Ga-K$\alpha$ line with $\lambda \eq 1.3414 \unit{\angstrom}$ and the beam slits for collimation are set to $1 \unit{mm}$.
    For the measurement, the sample is rotated up to an incident angle of $\alpha_i \eq 2^\circ$ in $0.01 ^\circ$ steps and at each angle the detector is moved by a few millimeters to three positions, to remove gaps on the detector images.
    The angle range is split into two parts, where a higher counting time is set for larger incident angles to improve the counting statistics at the lower scattering intensities for higher $q$.
    From $0 ^\circ \ldots 0.4 ^\circ$ each detector image is measured for $5 \unit{s}$, whereas for the range up to $2 ^\circ$ a measurement time of $30 \unit{s}$ is set.
    An overlap range of $0.1 ^\circ$ is set to obtain an overlap of the two angular ranges.
    To account for a fluctuating intensity of the incident beam and varied counting times, each detector image is scaled to it's respective monitor value.

    The obtained detector data are evaluated by defining a region of interest around the specular peak, which moves along the detector with increasing incident angle.
    The region of interest is integrated along the dimension on the detector perpendicular to the scattering direction, such that an intensity map with respect to the remaining detector dimension and the incident angle is obtained.
    Using the pixel-splitting rebinning algorithm described in \refapp{ch:appendix:numericalMethods:rebinningPixelSplitting}, the intensity map is transformed and rebinned to a ($\alpha_i - \alpha_o$, $\alpha_i+\alpha_o$) coordinate system, from which the specular intensity is obtained by integrating a box of $0.2 ^\circ$ width around $\alpha_i - \alpha_o \eq 0$.
    Additionally, the data is corrected for diffuse scattering around the specular peak, by integrating boxes in the range $\alpha_i - \alpha_o \eq \pm (0.1 \ldots 0.2)$, and subtracting the diffuse scattering intensity from the specular intensity.


  \paragraphNewLine{Vibrating Sample Magnetometry}
    The samples CC-Fe-0.25, CC-Fe-0.37 and CC-0.50 are characterized by VSM using a PPMS Evercool II (\refsec{ch:instruments:laboratoryInstruments:vsm}).
    Each sample is cut to a size of approximately $5 \times 5 \unit{mm^2}$ using a diamond cutter and stuck to a Quartz sample holder using a low temperature varnish (GE 7031).
    The silicon substrate mass is measured by a precision scale and tabulated in \reftab{tab:colloidalCrystals:layerCharacterization:ppmsMasses}.
    The mass variation is directly proportional to the variation in sample area, as each wafer has the same thickness of $d_\mathrm{Si} \eq 0.52 \unit{mm}$ and the iron oxide nanocube mass contribution is negligible.
    Considering a silicon density of $\rho_\mathrm{Si} \eq 2.32 \unit{g \, cm^{-3}}$, the magnetization is scaled to the wafer area by
    \begin{align}
      \bar{M} \eq \frac{\rho_\mathrm{Si}}{m_\mathrm{sample} d_\mathrm{Si}} M,
    \end{align}
    from units in $\mathrm{memu}$ to $\mathrm{memu / cm^2}$.

    \begin{table}[htbp]
      \centering
      \caption{\label{tab:colloidalCrystals:layerCharacterization:ppmsMasses}Mass of the colloidal crystal samples used for the VSM measurements.}
      \begin{tabular}{ l | l}
        \rule{0pt}{2ex} \textbf{Sample}  & $m_\mathrm{sample} \, / \unit{mg}$ \\
        \hline
        \rule{0pt}{2ex} CC-Fe-0.25   & $30.67(2)$ \\
        \rule{0pt}{2ex} CC-Fe-0.37   & $17.51(2)$ \\
        \rule{0pt}{2ex} CC-Fe-0.50   & $35.10(2)$ \\
        \hline
      \end{tabular}
    \end{table}

    Field-dependent measurements are performed for each sample at $300 \unit{K}$ and $10 \unit{K}$ in a range of $\pm 9 \unit{T}$ with a sweeping rate of $5 \unit{mT \, s^{-1}}$.
    Furthermore, temperature-dependent measurements are performed by zero-field cooling the sample to $10 \unit{K}$ and subsequently measuring the magnetization at $10 \unit{mT}$ while warming the sample with a rate of $1.5 \unit{K \, s^{-1}}$.
    Another field-cooled measurement is performed afterwards by first cooling the sample in a field of $10 \unit{mT}$ and subsequently measuring the magnetization while warming.

    From the room temperature measurements, the spontaneous magnetization and susceptibility of the samples is determined by a linear fit at high fields of $5 - 9 \unit{T}$, which are discussed after being scaled to the respective sample area using the mass of the silicon substrates.
    To study interaction effects that arise at low temperatures and to correct for the varying sample amounts on the substrates, the data is scaled to the spontaneous magnetization and the excess susceptibility is corrected as determined from the room temperature measurement.
    The peak positions from the zero-field and field cooled magnetization measurements are determined by a local parabola fit in a range of $\pm 20 \unit{K}$ around the estimated peak positions.
    The magnetization data is discussed qualitatively and the characteristics are compared across the three samples.


  \paragraphNewLine{Polarized Neutron Reflectometry}
    The three colloidal crystals, CC-Fe-0.25, CC-Fe-0.37 and CC-Fe-0.50 were characterized on the polarized neutron reflectometer MARIA at the MLZ \refsec{ch:lss:maria}.
    The neutron wavelength was selected to $5 \unit{\angstrom}$ and a wavelength spread of $10 \unit{\%}$ (FWHM) is given by the instrumental scientists.
    The collimation and sample slits along the scattering direction are set to $2 \unit{mm}$, the collimation-to-sample distance was $4.1 \unit{m}$ and the sample-to-detector distance was $2.093 \unit{m}$.
    Each sample was measured at a temperature of $10 \unit{K}$ after cooling in near zero field ($0.9 \unit{mT}$ guide field), subsequent at a saturating field of $1.08 \unit{T}$ and again in remanence at guide field.
    After that, each sample was warmed to $250 \unit{K}$, sufficiently above the blocking temperature, and then cooled back to $10 \unit{K}$ while applying the saturating field.
    After field-cooling, the sample was then measured once more in saturation and remanence, which totals in five neutron reflectometry measurements for each sample at $10 \unit{K}$.
    For each case, the reflectivity $R^{+}$, for neutrons polarized parallel to the magnetic field direction, and the reflectivity $R^{-}$, for neutrons that are polarized antiparallel to it, is measured.
    No polarization analysis is performed for the sake of having higher counting statistics in the two channels.

    For the measurements, the incident angle was increased up to $\alpha_i \eq 6 ^\circ$, where the angle step is increased for three ranges.
    For the first range up to $\alpha_i \eq 1 ^\circ$ a step of $0.02 ^\circ$ is set.
    For a second range from $\alpha_i \eq 0.8 ^\circ$ up to $\alpha_i \eq 3.5 ^\circ$ a step of $0.05 ^\circ$ is chosen and for the last range from $\alpha_i \eq 2.8 ^\circ$ up to $\alpha_i \eq 6 ^\circ$ a step of $0.2 ^\circ$ is set.
    The three ranges overlap intentionally in case the scaling of the data to the monitor would show a disagreement, which was not the case in the executed experiments.
    At the first range, each point is measured for $30 \unit{s}$, whereas for the second $120 \unit{s}$ and the third range $135 \unit{s}$ are set due to the lower scattering intensity and for improved statistics.

    For the data reduction, each detector image is corrected for the detector sensitivity and integrated along the dimension with relaxed collimation.
    By this integration the data is given as a map of the scattered intensity with respect to the detector pixel $x$ along the scattering direction and with respect to the incident angle $\alpha_i$.
    This map is transformed and rotated to ($\alpha_i - \alpha_o$, $\alpha_i+\alpha_o$) coordinates by calculating $\alpha_o$ from
    \begin{align}
      \alpha_o = \alpha_i + \arctan \biggl( \frac{x - x_\mathrm{spec}}{L_\mathrm{SDD}} \biggr),
    \end{align}
    where $L_\mathrm{SDD}$ is the sample-to-detector distance and $x_\mathrm{spec}$ the center pixel of the specular beam.
    For the transformation and rotation, the data needs to be rebinned in the new coordinate system.
    To avoid the loss of resolution, a pixel-splitting algorithm is used for the rebinning, which is described in \refapp{ch:appendix:numericalMethods:rebinningPixelSplitting}.

    The specular intensity is obtained, by integrating on the transformed data a strip of $0.1 ^\circ$ width around $\alpha_i - \alpha_o \eq 0$.
    The data is further corrected for the diffuse scattering by additionally integrating strips of $0.1 ^\circ$ width in the off-specular region from $\pm (0.1 ^\circ \ldots 0.2)$, which are subtracted from the specular intensity after being rescaled to adjust for a different numbers of contributing bins.

    Finally a footprint correction is performed for the data, where an equidistributed intensity of the beam is assumed.
    Using the mean intensity of $R^{+}$ below the critical angle, the data is rescaled to have the plateau on unity.
    The curve of $R^{-}$ is rescaled using the same factor, to avoid the introduction of systematic errors by using two different scaling factors.
\end{document}