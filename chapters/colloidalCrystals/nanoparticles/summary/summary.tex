\providecommand{\main}{../../../..}
\documentclass[\main/dresen_thesis.tex]{subfiles}

\begin{document}
  \label{sec:looselyPackedNS:nanoparticle:discussion:summary}
  In summary the obtained nanoparticles from the collaboration are characterized structurally and magnetically by multiple complimentary experiments.

  Using TEM, XRD and SAS the particle diameter of the nanospheres IOS-11 is determined to be $10.9(1) \unit{nm}$ and a core-shell structure with a w\"ustite core and magnetite shell could be confirmed.
  The particle size distribution is in the order of $5.4 \unit{\%}$.
  It became apparent that the exact phase of the nanospheres varies greatly as the magnetite shells continues to oxidize with time and contact to ambient conditions, turning the w\"ustite core slowly to magnetite.
  The magnetization of a nanosphere IOS-11 was determined to be in the order of $200 \unit{kA \, m^{-1}}$ for a particle on average, where the magnetizations comes primarily from the shell, which is close to the magnetite bulk value as observed by SANSPOL.

  For IOS-7, the spherical shape and size was determined by TEM and SAS, where a bimodal distribution was observed by both experiments.
  The primary mode has a diameter of $7.0(1) \unit{nm}$ and size distribution of $7.5 \unit{\%}$, whereas the second mode consists of small nanoparticles in the order of $1 \unit{nm}$ with a large size distribution of $60 \%$.
  The nanospheres of IOS-7 appear to be fully oxidized from SAXS and SANS and the magnetization of the nanospheres is also in the order of $200 \unit{kA \, m^{-1}}$.

  From the near monodisperse distribution, IOS-11 will be the primary sample of interest to study nanostructures prepared by spin-coating, as the narrow size distribution allows an easier access to discuss higher-order nanostructures on the basis of the nanoparticle building blocks.
\end{document}