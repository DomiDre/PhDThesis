\providecommand{\main}{../../../..}
\documentclass[\main/dresen_thesis.tex]{subfiles}

\begin{document}
  \label{sec:colloidalCrystals:nanoparticle:summary}
  The nanocubes Ol-Fe-C that were prepared from iron oleate are characterized by multiple complimentary experiments to determine the nanoparticle shape, size, size distribution and magnetic properties.
  Transmission electron microscopy shows the cubical shape of the nanoparticles and yields a mean edge length of $13.42(9) \unit{nm}$ and size distribution of $7.3(5) \%$ from evaluating a specimen.
  Using XRD, the composition of the nanocubes is confirmed to be of two phases, one inverse spinell phase and one w\"ustite phase.
  From the magnitude of the lattice constant the shell can be considered as close to magnetite and the core to w\"ustite.

  Both results are consistent with the results from small-angle scattering, which is described by a superball form factor.
  Here, a superball edge length of $12.22(5) \unit{nm}$ is observed with a size distribution of $7.2(2) \%$ from SAXS.
  The superball exponent is obtained as $2.2(1)$, which corresponds to cube with round edges.
  From SANS, the data is consistent with the same parameters and a surfactant shell thickness of $1.41 \unit{nm}$.
  The w\"ustite core and magnetite shell fraction of the nanocubes is fit independently from one another due to a large gap in time between the two experiments, where the particle may have further oxidized.
  In the SANS experiment that was performed first, a shell thickness of $3.3(2) \unit{nm}$ is observed, whereas for the later SAXS experiment a thickness of $5.5(2) \unit{nm}$ is seen.
  
  The magnetic properties of the nanocubes are characterized by SANSPOL and VSM.
  From VSM right after synthesis a magnetic moment of $25600(264) \mu_B$ per particle is seen, which corresponds with the particle size to a spontaneous magnetization of $155 \unit{kA \, m^{-1}}$.
  Additional a paramagnetic susceptibility and a second contribution is seen from fitting the magnetization behaviour, which has a lower magnetic moment of $3605(65) \mu_B$.
  The nature of the second magnetic phase is argumented to originate from the core-shell structure.
  The SANSPOL measurement that has been chronologically performed after the VSM experiment, resolves the magnetic core-shell structure further and yields a magnetization of $274(64) \unit{kA \, m^{-1}}$ for the core and $399(13) \unit{kA \, m^{-1}}$ for the shell.
  By volume averaging the magnetization, this corresponds to a particle magnetization of $387(13) \unit{kA \, m^{-1}}$ and is therefore significantly larger than the spontaneous magnetization observed by VSM, which confirms that the nanocubes have further oxidized in the time between synthesis and characterization.

  By the characterization, the nanoparticle size, shape and size distribution is well defined.
  Only the magnetic structure is not well-defined due to the metastable core-shell phase of the nanocubes that oxidizes over a larger time frame.
\end{document}