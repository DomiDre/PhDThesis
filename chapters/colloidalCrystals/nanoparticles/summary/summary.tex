\providecommand{\main}{../../../..}
\documentclass[\main/dresen_thesis.tex]{subfiles}

\begin{document}
  \label{sec:colloidalCrystals:nanoparticle:summary}
  The nanocubes Ol-Fe-C that were prepared from iron oleate are characterized by multiple complimentary experiments to determine the nanoparticle shape, size, size distribution and magnetic properties.
  Transmission electron microscopy shows the cubic shape of the nanoparticles and yields a mean edge length of $13.42(9) \unit{nm}$ and size distribution of $7.3(5) \%$ from evaluating a specimen.
  Using XRD, the composition of the nanocubes is described by two phases, one inverse spinell phase and one w\"ustite phase.
  From the magnitude of the lattice constant $a_\mathrm{inv. spinell} \eq 8.3841 \unit{\angstrom}$ the inverse spinell phase can be considered as close to magnetite.
  A closer inspection of the lattice constant of the w\"ustite phase reveals that the value is with $a_\textsf{w\"ustite} \eq 4.1809 \unit{\angstrom} \approx \tfrac{1}{2} a_\mathrm{inv. spinell}$ substantially smaller than the literature value for w\"ustite.
  By comparison with literature discussions \cite{Wetterskog_2013_Anoma}, it is concluded that the two phases actually represent the discrepancy between the long-range order of the tetrahedral and octahedral sublattice in the spinell phase tetrahedral and octahedral sublattices due to anti-phase boundaries in the nanoparticle from topotaxial oxidation.

  Small-angle X-ray and neutron scattering data is furthermore discussed by a superball form factor.
  Here, a superball edge length of $12.22(5) \unit{nm}$ is observed with a size distribution of $7.2(2) \%$ from SAXS.
  The superball exponent is obtained as $2.2(1)$, which corresponds to cube with round edges.
  From SANS, the data is consistent with the same parameters and a surfactant shell thickness of $1.38(4) \unit{nm}$.
  The surfactant shell thickness is comparable to the order of magnitude observed for similar sized and shaped nanoparticles in this thesis in \refch{ch:monolayers}, as well as with other oleic acid-ligated nanoparticles discussed in literature by small-angle scattering \cite{Disch_2012_Quant}.
  For SANS a w\"ustite core and magnetite shell needs to be used for a fit of the data, whereas for SAXS a single phase superball provides the best fit.
  In the SANS experiment that was chronologically performed first, a shell thickness of $3.3(2) \unit{nm}$ is observed, which is however no longer visible in SAXS even though w\"ustite and magnetite have also a strong contrast for X-rays.
  It is concluded that within the large gap in time between the two experiments, the particles have fully oxidized.
  \\

  The magnetic properties of the nanocubes are characterized by SANSPOL and VSM.
  From VSM performed right after synthesis, two magnitudes of magnetic moments are observed in a dried sample of the nanocubes.
  One primary mode with a moment of $25600(264) \mu_B$ and one mode with a moment $3605(65) \mu_B$.
  Using the primary mode, the data is used to estimate the number of particles in the measured sample, which is used with the single particle volume determined from small-angle scattering to estimate the volume of the magnetic material in the sample.
  Scaling the observed magnetization to this moment, the primary mode scaled to the complete particle corresponds to a spontaneous magnetization of $155(1) \unit{kA \, m^{-1}}$.
  Determining the number of moments in the second mode and equidistributing these on the number of particles estimated from the first mode, the second mode is estimated to have a spontaneous magnetization of $177(1) \unit{kA \, m^{-1}}$.
  It is argued however that due to the complex core-shell structure of the as-synthesized nanoparticles the spontaneous magnetization values under estimate the true value as the observed magnetism is initially not given by the whole nanoparticle volume, but from an oxidized fraction in the nanoparticle shell.
  Furthermore, a paramagnetic excess susceptibility is observed in the VSM measurement of the as-synthesized nanoparticles, which is associated with the w\"ustite phase of the nanoparticles.
  The observed value of the excess susceptibility of $\mu_0 \chi \eq 53000(900) \cdot 10^{-6}$, however, is significantly larger than the expected value for w\"ustite at ambient condition of $\mu_0 \chi \eq 7230 \cdot 10^{-6}$ \cite{Lide_2004_Handb}, which is reasoned to a systematic error originating from the limited measured range of $\pm 2 \unit{T}$.

  The SANSPOL measurement that has been chronologically performed three months after the VSM experiment, resolves the magnetic core-shell structure further and yields a magnetization of $274(64) \unit{kA \, m^{-1}}$ for the core and $399(13) \unit{kA \, m^{-1}}$ for the shell.
  By volume averaging the magnetization, this corresponds to a particle magnetization of $387(13) \unit{kA \, m^{-1}}$ and is therefore significantly larger than the spontaneous magnetization observed by VSM, which confirms that the nanocubes have further oxidized in the time between synthesis and characterization.

  By the characterization, the nanoparticle size, shape and size distribution is well defined.
  Only the magnetic structure is not well-defined due to the metastable core-shell phase of the nanocubes that oxidizes over a larger time frame.
  An additional VSM measurement of the nanoparticles at a later stage in time was not performed within this thesis as by the time the batch was completely depleted.
\end{document}