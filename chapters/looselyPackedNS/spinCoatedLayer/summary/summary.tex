\providecommand{\main}{../../../..}
\documentclass[\main/dresen_thesis.tex]{subfiles}

\begin{document}
  \label{sec:looselyPackedNS:layers:summary}
  Two spin-coated layers, one from the nanospheres IOS-11 and one from the smaller nanospheres IOS-7, have been characterized structurally and magnetically in this chapter using multiple complementary experiments.
  While SEM shows qualitatively that a homogeneous thin layer structure is achieved and that the nanospheres show no long-range order as far as can be told from the top and cross-sectional view.
  GISAXS and reflectometry is used to quantify the packing and average particle spacing in the layers.
  From GISAXS, a packing density of $43.88(3) \%$ and $34.20(9) \%$ is observed by evaluating the structure factor for SC-IOS-11 and SC-IOS-7 with a liquid-like hard-sphere model respectively.
  The packing fraction is given in terms of the iron oxide cores within the volume and neglects the oleic acid surfactant shell, which is visible from the hard-sphere radius that corresponds close to the particle radius from small-angle scattering.
  Relating the packing fraction to the average spherical volume every particle is given in a comparatively close packing, sizes that are approximately $1.3 - 1.5 \unit{nm}$ larger than the particle radius are observed, which connects to the oleic acid surfactant shell of each particle.
  As an oleic acid chain has a theoretical length of $2.1 \unit{nm}$, the reduced spacing tells that the respective layers of the spherical particles are penetrating one another as they have to physically overlap for this result.

  From reflectivity the structure is analyzed by evaluating the average vertical density of the samples as seen by X-rays and by neutrons.
  Using a model of layered spheres, the average layer packing fraction is determined as well as the average distance of the spheres.
  The visualization of the scattering length density profiles tells a comparative story to GISAXS, where the nanosphere surfactant shells are overlapping.
  Furthermore, low-temperature neutron reflectometry reveal that the layers compress upon freezing, which results in a higher packing of the nanostructures.
  Comparing reflectometry and GISAS, in both cases a loose packing of the nanospheres is observed.
  On average the nanospheres in the samples have a layered structure in the vertical structure across the sample as seen by reflectometry, but no notable long-ranged lateral order is seen as GISAXS would show otherwise visible higher order correlation peaks instead of the liquid-like structure factor.

  Using the structural characterization, the sample magnetization is studied.
  From field-dependent vibrating sample magnetometry at room temperature, the magnetic moment of the nanospheres is determined, which is found to be $20 \%$ higher than obtained from the nanospheres in dispersion.
  Temperature-dependent measurements of the thin layers show blocking temperatures of $102 \unit{K}$ and $47 \unit{K}$ for SC-IOS-11 and SC-IOS-7 respectively, which is especially in the case of SC-IOS-11 increased to the frozen dispersion obtained values of $95.5 \unit{K}$ and $46.5 \unit{K}$.
  The samples show a small coercivity at $30 \unit{K}$ of $15 \unit{mT}$ and $9 \unit{mT}$ respectively and an exchange bias effect is seen after field cooling by both samples when comparing the hysteresis after ZFC and FC.
  The latter is connected to the anomalous magnetic properties of iron oxide nanoparticles prepared from iron oleate, which have a w\"ustite/magnetite core-shell structure and anti-phase boundaries throughout the magnetite phase.
  The shifted blocking temperature however is a signature of magnetic particle interaction.
  Seeing the shift significantly only in SC-IOS-11 and not in SC-IOS-7 relates to the greater relative distance achieved by the oleic acid shell with respect to the nanoparticle size.
  As the dipolar interaction strength is proportional to $\propto r^3 / d_\mathrm{p-p}^3$, where $r$ is the size of the magnetic particle and $d_\mathrm{p-p}^3$ the interparticle distance, the oleic acid shell, which should be approximately the same in both particle batches, provides a greater attenuation for the smaller nanoparticle batch.

  Using polarized neutron reflectometry, the magnetization is further studied with depth-resolution.
  For SC-IOS-7, the magnetic splitting in PNR is too weak for a quantitative analysis.
  For SC-IOS-11, a homogeneously magnetized model and a model with varied layer magnetization is studied.
  The homogeneously magnetized model reproduces the sample magnetization observed in VSM, whereas the freely varied layers show that the spin density is highest in the layers where the nuclear density is lowered.
  The same is observed after zero-field and field cooling of the samples.

  Measurements of the remanent sample state after ZFC and FC and application of a strong magnetic field show no visible magnetic splitting, which would however be expected for the nanoparticles being far below their blocking temperature.
  Also the spin-flip channel showed no significant signal beyond spin-leakage from the non-spin-flip channels, due to finite polarization efficiency.
  The latter observations tells that no large magnetic domain perpendicular to the originally applied magnetic field direction in the sample plane is present and the first observation that within the coherent scattering volume the sample magnetization averages to a near zero value at remanence.
  Any strong anti-ferromagnetic coupling between the nanosphere layers at remanence can be excluded as they would have shown a visible signature in the polarized neutron reflectometry measurement.

  Due to the directional nature of dipolar interaction, it should be expected that higher structural order is necessary in the samples to see collective magnetism that significantly differs from the single nanoparticle properties.
  The short-ranged ordered packing of the nanospheres thus seems insufficient to show anything more but slight modifications of the blocking temperature as signature of dipolar interaction.

\end{document}