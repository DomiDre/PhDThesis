\providecommand{\main}{../../..}
\documentclass[\main/dresen_thesis.tex]{subfiles}

\begin{document}
  Magnetic nanoparticles have become an highly active field of research within the last decades, where it is nowadays possible to routinely synthesize them for a large class of materials, such as metals, rare earth metals, magnetic alloys or metal oxides \cite{Gubin_2005_Magne}.
  The interest is driven by their promising applications in clinical applications \cite{Thanh_2012_Magne}, as well as in electronics, optics, energy conversion and storage, where nanoparticles can be used as building blocks to construct nanostructured materials with unique properties \cite{Singamaneni_2011_Magne}.
  For this purpose, a common choice to form packed structures of nanoparticles is the bottom-up process of self-assembly, where the nanoparticles spontaneously organize due to direct specific interactions and/or indirect through the environment \cite{Grzelczak_2010_direc, Whitesides_2002_Selfa}.
  This provides a large number of variety to direct the self-assembly process \ie by varying the material, by varying the deposition method, by chemically modifying the surface of the nanoparticles or by applying electric or magnetic fields during the process.

  In this work, magnetic nanospheres of iron oxide with an oleic acid surface are studied, as well as the self-assembled packings obtained by spin-coating the studied nanoparticles from dispersion.
  Iron oxide nanoparticles are well studied in recent literature, due to their easy availability and their potential in medical applications, and can be synthesized by multiple routes \cite{Laurent_2008_Magne}.
  The synthesis from iron oleate \cite{Hyeon_2003_Chemi}, is especially well suited to obtain nanoparticles in the order of $10 \unit{nm}$ with a small size distribution \cite{Wetterskog_2014_Preci}.
  The iron oxide phases of maghemite ($\gamma-\ch{Fe2O3}$) and magnetite $\ch{Fe3O4}$) furthermore show a strong ferrimagnetism and are therefore well suited when the nanoparticles are used as probes to study nanoparticular interaction mediated to dipolar forces.

  The self-assembly by spin-coating is known to be a quick method to obtain thin layers that have a low surface roughness \cite{Xu_2012_Trans}.
  Spin-coated iron oxide nanospheres have been recently studied in literature by Mishra \etal \cite{Mishra_2012_Selfa} by GISAXS and X-ray and polarized neutron reflectivity, for commercially obtained spheres with $20 \unit{nm}$ diameter.
  In their study they found for spin coated layers, which consisted of four/five sub-layers of iron oxide nanoparticles, a short-ranged hexagonal in-plane order of the particles by GISAXS and that the order increases for layers away from the substrate from XRR.
  By PNR, they found at a saturating field of $500 \unit{mT}$ that the magnetization of the iron oxide layer has on average a non-collinear contribution of $\braket{\cos{\gamma}} \eq 0.8$ with respect to the magnetic field direction.
  This $20 \%$ reduction is explained by frustration effects caused by competing dipolar forces of small quasi domains of nanoparticles.
  Furthermore, they observe that the magnitude of the magnetic scattering length density is further $10 \%$ reduced from the bulk value of iron oxide, which they explain by an additional w\"ustite phase.
  After removal of the magnetic field, the splitting of the two channels in PNR vanishes, which they explain by a reduction of the non-collinearity parameter to $\braket{\cos{\gamma}} \eq 0.12$, but a remaining strong magnetic scattering length density.
  They then conclude that the sample comprises of large quasi-domains, which after incoherent averaging net to a zero signal and thereby on the presence of dipolar interactions.

  In a further study with polarization analysis on a spin-coated layer that consists primarily of one nanoparticle layer of iron oxide spheres \cite{Mishra_2015_Polar}, they produce similar findings and notice that no spin-flip scattering is observed beyond that due to imperfect spin polarizing and analyzing efficiencies.
  The observation of a spin-flip contribution would however strengthen the assumption that at remanence the nanoparticle assembly comprises of quasi domains with random magnetization orientation.
  Thus the paper argues that the iron oxide nanostructure comprises only of domains that are pointing nearly parallel and anti-parallel to the formerly applied magnetization direction, and $\braket{\cos{\gamma}}$ gives the dominance of the parallel domains over the anti-parallel domains.
  Here, a electron holography study of an iron oxide nanoparticles array is given as reference \cite{Yamamoto_2011_Dipol}, which shows a region of anti-parallel alignment of quasi-domains.
  The size of such correlated domains is given to be greater than a few dozen micrometers, as no off-specular scattering is observed in reflectivity.

  An important detail that is missing in the presented discussion of nanostructured arrays is however the proper characterization of the single-nanoparticle properties.
  It is known by now that iron oxide spheres can have a magnetization distribution that is already far off from the bulk properties due to core-shell structures and a defect structure in the nanoparticle \cite{Disch_2012_Quant, Wetterskog_2013_Anoma}.
  Describing a nanoparticle structure by assuming bulk properties and inferring conclusions from the observed differences is therefore always questionable.
  Also, the study is performed at room temperature as it is assumed that for the given nanoparticles of $20 \unit{nm}$ diameter with bulk iron oxide magnetic properties, the dipolar forces dominate the thermal fluctuations at remanence.
  A study with structures of nanoparticle arrays, where the effect of thermal fluctuations is considered is not found.

  In the following, this thesis presents a study where iron oxide nanoparticles are characterized individually in dispersion by small-angle scattering as first step, and then spin-coated layers of the same nanoparticles are studied at room temperature, as well as after cooling below the superparamagnetic blocking temperature of the individual nanoparticles with and without an applied magnetic field.
  The idea is then that the comparison of both the nanoparticle characterization and the characterization of the spin-coated layers provides a clear access to the dipolar interactions, where no assumptions from the bulk properties are made.
\end{document}