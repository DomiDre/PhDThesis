\providecommand{\main}{../../..}
\documentclass[\main/dresen_thesis.tex]{subfiles}

\begin{document}
  Magnetic nanoparticles have become an highly active field of research within the last decades, where it is nowadays possible to routinely synthesize them for a large class of materials, such as metals, rare earth metals, magnetic alloys or metal oxides \cite{Gubin_2005_Magne}.
  The interest is driven by their promising applications in the field of medicine \cite{Thanh_2012_Magne}, as well as in the field of electronics, optics, for energy conversion or for energy storage, where nanoparticles can be used as building blocks to construct nanostructured materials with unique properties \cite{Singamaneni_2011_Magne}.
  For this purpose, a common choice to form packed structures of nanoparticles is the bottom-up process of self-assembly, where the nanoparticles spontaneously organize due to direct specific interactions and/or indirect through the environment \cite{Grzelczak_2010_direc, Whitesides_2002_Selfa}.
  This provides a large number of variety to direct the self-assembly process \ie by varying the material, by varying the deposition method, by chemically modifying the surface of the nanoparticles or by applying an electric or magnetic field during the process.

  In this thesis, the properties of magnetic nanoparticles in a non-interacting state in dispersion, as well as in packed array is studied to gain a deeper understanding of the emerging effects resulting from dipolar interparticle interaction.
  The self-assembly by spin-coating is known to be a quick method to obtain thin layers that have a low surface roughness \cite{Xu_2012_Trans}.
  Spin-coated iron oxide nanospheres have been recently studied in literature by Mishra \etal \cite{Mishra_2012_Selfa} by GISAXS and X-ray and polarized neutron reflectivity.
  A short-ranged hexagonal in-plane order of the particles is found by GISAXS for spin coated layers consisting of four/five sub-layers of iron oxide nanoparticles, and that the order increases for layers away from the substrate from XRR.
  By PNR at a saturating field, a non-collinear contribution of the magnetization of the iron oxide layer with respect to the magnetic field direction is found.
  This reduction is explained by frustration effects caused by competing dipolar forces of small quasi domains of nanoparticles.
  Furthermore, a reduced magnetization in comparison to bulk magnetite is observed and ascribed to phase impurities in the nanoparticles.
  After removal of the magnetic field, the splitting of the two channels in PNR vanishes, which indicates an increase of the non-collinear orientation of the nanoparticle spins.
  In a polarization analysis study \cite{Mishra_2015_Polar}, no spin-flip scattering is observed neither at saturation nor at remanence, which indicates that no significant amount of domains with an in-plane magnetization perpendicular to the applied field direction is found in the sample.
  An electron holography study of an iron oxide nanoparticles array by Yamamoto \etal \cite{Yamamoto_2011_Dipol}, supports this observation of nanoparticle domains with antiparallel alignment.
  \\

  Iron oxide nanoparticles are well studied in recent literature and can be synthesized by multiple routes \cite{Laurent_2008_Magne}.
  The synthesis from iron oleate \cite{Hyeon_2003_Chemi}, is especially well suited to obtain nanoparticles in the order of $10 \unit{nm}$ with a small size distribution \cite{Wetterskog_2014_Preci}.
  The iron oxide phases of maghemite ($\gamma-\ch{Fe2O3}$) and magnetite ($\ch{Fe3O4}$) furthermore show a strong ferrimagnetism and are therefore well suited when the nanoparticles are used as probes to study nanoparticular interaction through dipolar forces.

  Wetterskog \etal published a study on the phase and oxidation of iron oxide nanoparticles synthesized from iron oleate \cite{Wetterskog_2013_Anoma}.
  Similar to earlier studies \cite{Hai_2010_Sizec, Chen_2010_Chara}, in the synthesis a w\"ustite core is formed first, and a shell with a phase between magnetite and maghemite, summarized as \ch{Fe_{3-$\delta$}O4}, oxidizes afterwards from the w\"ustite structure.
  Due to multiple possible ways the rock salt structure of w\"ustite can transform to the inverse spinell of \ch{Fe_{3-$\delta$}O4}, antiphase boundaries are formed throughout the nanoparticle as defects.
  These defects are not removed without a significant diffusion of cations \cite{Margulies_1997_Origi} and thus high annealing temperatures.
  Wetterskog \etal show that the antiphase boundaries are the source of a persisting reduced magnetization, as well as an exchange-bias effect even in single-phased nanoparticles, as to why the magnetism of such nanoparticles appears to be anomalous.

  From small-angle neutron scattering experiments \cite{Disch_2012_Quant}, the spatial magnetization distribution in non-interaction iron oxide spheres prepared from oleates has been resolved.
  Additionally to a reduced magnetization in the shell of the nanoparticles, a significantly reduced magnetization across the particle volume is observed in comparison to the expectation from the bulk properties.
  This motivates for the study of dipolar interparticle interaction, to include a study of the non-interacting nanoparticle properties as reference.
  \\

  In the following, this thesis presents a study where iron oxide nanoparticles are characterized individually in dispersion by small-angle scattering as first step, and then spin-coated layers of the same nanoparticles are studied at room temperature, as well as after cooling below the superparamagnetic blocking temperature of the individual nanoparticles with and without an applied magnetic field.
  The idea is then that the comparison of both the nanoparticle characterization and the characterization of the spin-coated layers provides a clear access to the dipolar interactions, where no assumptions from the bulk properties are made.
\end{document}