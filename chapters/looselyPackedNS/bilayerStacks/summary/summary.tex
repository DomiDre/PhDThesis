\providecommand{\main}{../../../..}
\documentclass[\main/dresen_thesis.tex]{subfiles}

\begin{document}
  \label{sec:looselyPackedNS:bilayerStacks:summary}
  Layered arrays of magnetic nanospheres with non-magnetic PMMA interlayer spacers have been studied by electron microscopy, GISAXS and reflectometry.
  From the SEM micrographs, a clear separation of the layers and a thickness reproducibility of the respective layers in a sample in the order of $5 - 15 \%$ is visible.
  From GISAXS, the same picture as for the single layers of nanospheres in the previous chapters is observed: no long-range ordered lateral correlations are observed but a modification from the single-particle form factor that resembles a hard-sphere liquid structure factor is visible.
  The packing fractions for the bilayer samples from IOS-11 have an order of magnitude of $42 - 43 \%$ and the samples from IOS-7 $35 - 36 \%$.
  The results reproduce the observations made for the one layer of spin coated nanospheres, which was discussed to be an only short-ranged ordered packing where the oleic acid surfactant shells of the nanospheres are overlapping.

  All samples were measured with X-ray and neutron reflectometry, where the layered structure clearly reflects by the emergence of Bragg peaks.
  Only the bilayer stacks from IOS-7 are inaccessible, probably due to a bad contrast between the nanoparticle layers and the PMMA in this case.
  From the Bragg peaks, the average bilayer thickness is determined and the result from the three structures is compared.
  In all cases, SEM provides the smallest estimate of the average bilayer thickness, and XRR provides a larger thickness than NR, which is explained by the differently perceived contrast of nanoparticles and PMMA by X-rays and neutrons.
  Furthermore, upon cooling of the samples to $30 \unit{K}$, it is observed by neutron reflectometry that the structures compress as a reduced bilayer thicknesses is determined from an increased spacing of the Bragg peak positions.
  A quantitative layer-by-layer evaluation by a scattering length density profile model has not been achieved, due to the high complexity of the sample structure and the thereby connected large number of independent variables that have to be determined.

  Using polarized neutron reflectometry, the spin density of the samples is further probed with depth resolution.
  Analogue to the study of the one layer of nanospheres, the bilayer stacks show a homogeneous splitting of $R^{+}$ and $R^{-}$ at high fields, but no splitting after either ZFC or FC at remanence and no spin-flip scattering beyond spin leakage.
  The absence of any peculiar features in the data and no spin-flip scattering alludes that the spacing of the layers might be too large or interlayer coupling a too weak effect in nature to measure with PNR.
  As it was not possible to resolve the nuclear structure, it is also not possible to study the layer-by-layer magnetization quantitatively.
  However, it should not be expected that strong effects would be observed from a fit of the data from the qualitative evaluation.

\end{document}