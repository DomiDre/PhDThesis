\providecommand{\main}{../../..}
\documentclass[\main/dresen_thesis.tex]{subfiles}
  \renewcommand{\thisPath}{\main/chapters/looselyPackedNS/experimentalMethods/}
\begin{document}
  Iron oxide nanospheres were obtained through a collaboration with the group of Prof. Tremel from the inorganic chemistry department of the University of Mainz.
  The nanoparticles were prepared by following a synthesis route where iron oleate is prepared in a first step, and subsequently nanospheres are obtained by a controlled thermal decomposition of the oleate in a solvent with a high boiling.
  A similar synthesis has been applied in this thesis for the preparation of cobalt ferrite nanocubes and is explained in \refch{ch:monolayers} in detail.

  Two nanoparticle batches, IOS-11 and IOS-7, were obtained in a concentrated dispersion in toluene and are used without further chemical modification.
  Furthermore, the collaboration provided spin-coated layers of the same nanoparticles for both batches.
  Spin-coated single layers are provided, as well as organic-inorganic multilayers consisting of 8 subsequent layers of spin-coated nanospheres on top of spin-coated PMMA.
  For each nanoparticle batch, two multilayers with varied PMMA thickness are provided.
  The spin-coated layers are referenced as SC-IOS-7 and SC-IOS-11 respectively, where as the multilayers are referenced in the pattern 8BL-$x$-IOS-$y$, where $x$ is an estimate of the PMMA layer thickness and $y$ the respective nanoparticle size.

  The samples are summarized in \reftab{tab:looselyPackedNS:expMethods:samples}.
  \begin{table}[!htbp]
    \centering
    \caption{\label{tab:looselyPackedNS:expMethods:samples}Provided samples studied in this chapter.}
    \begin{tabular}{ l | l }
      \textbf{Sample}  & Description \\
      \hline
      IOS-11        & iron oxide nanospheres\\
      SC-IOS-11     & IOS-11 spin-coated on silicon\\
      8BL-15-IOS-11 & eight spin-coated bilayers of IOS-11 and PMMA on silicon\\
      8BL-40-IOS-11 & eight spin-coated bilayers of IOS-11 and PMMA on silicon\\
      \\
      IOS-7         & iron oxide nanospheres\\
      SC-IOS-7      & IOS-7 spin-coated on silicon\\
      8BL-15-IOS-7  & eight spin-coated bilayers of IOS-7 and PMMA on silicon\\
      8BL-30-IOS-7  & eight spin-coated bilayers of IOS-7 and PMMA on silicon\\
      \hline
    \end{tabular}
  \end{table}

  In the following the experimental methods for the characterization of the nanospheres, as well as the single layers and multilayers are described.

  \subsection{Nanoparticle Characterization}
    \subfile{\thisPath/nanoparticleCharacterization/nanoparticleCharacterization}
    \FloatBarrier
    \clearpage
  
  \subsection{Preparation of the Nanosphere Layers}
    \subfile{\thisPath/preparation/preparation}
    \FloatBarrier

  \subsection{Characterization of Nanosphere Layers}
  \subfile{\thisPath/layerCharacterization/layerCharacterization}
    \FloatBarrier

\end{document}