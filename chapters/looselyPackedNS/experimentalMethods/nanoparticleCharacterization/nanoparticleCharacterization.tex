\providecommand{\main}{../../../..}
\documentclass[\main/dresen_thesis.tex]{subfiles}

\begin{document}
  \paragraphNewLine{Transmission Electron Microscopy}
    Micrographs from transmission electron microscopy were obtained by the collaboration from the University of Mainz that prepared the nanoparticles.
    The micrographs were measured on a FEI Tecnai T-12-TEM with \ch{LaB6} cathode at an acceleration voltage of $120 \unit{kV}$.
    The nanosphere diameter and size distributio of over 200 nanoparticles was measured manually using the Fiji distribution \cite{Schindelin_2012_Fijia} and evaluated by fitting a log-normal distribution as described in \refch{ch:methods:em}.

    Due to the occurence of a large asymmetric size distribution in one batch, it is necessary to fit a bimodal log-normal distribution to properly describe the observed probability distribution.
    The bimodal distribution is defined by
    \begin{align}
      p_\mathrm{bimodal}(d) \eq (1 - \alpha) p_1(d; \mu_1, \sigma_1) + \alpha p_2(d; \mu_2, \sigma_2),
    \end{align}
    where $\alpha$ is the fraction of particles in the second mode and $p_i(d; \mu_i, \sigma_i)$ the log-normal size distribution of mode $i$ with parameters $\mu_i$ and $\sigma_i$ ($i \eq 1,\,2$).

  \paragraphNewLine{X-Ray Diffraction}
    X-ray diffraction of IOS-11 was measured in cooperation with the group of Daniel Nižňanský from the Department of Inorganic Chemistry at the Charles University in Prague on an PANanalytical X'Pert PRO, which is described in \refch{ch:instruments:laboratoryInstruments:xrd}.
    The sample was dried for the transport and redispersed at the laboratory in Prague where it was then evaporated on a glass substrate for the measurement.
    The experiment was performed with a Cu-K$\alpha$ source ($\lambda \eq 1.54 \angstrom$) and a $2 \theta \eq 5^\circ \ldots 80^\circ$ has been measured.
    The instrumental broadening is determined from a LaB6 reference measurement (SR 660b, NIST).

    To evaluate the data, a LeBail fit is performed using the FullProf suite \cite{Rodriguez_1993_Recen} as described in \refch{ch:methods:xrd}.
    The background was estimated manually by selecting around 20 background points, from which the measured range is interpolated linearly.
    Due to a large background coming from the glass substrate at low angles, the data between $5 ^\circ \ldots 15 ^\circ$ is excluded from the refinement.
    As it is shown that the inverse spinell phase (space group $Fd\bar{3}m$, No. 227) of magnetite/maghemite is not sufficient to describe the XRD, a combination of an inverse spinell phase and a w\"ustite phase (space group $Fm\bar{3}m$, No. 225) is refined.

  % \paragraphNewLine{Energy-Dispersive X-Ray Spectroscopy}
  %   By using the scanning electron microscope Neon Zeiss 40 (\refch{ch:instruments:laboratoryInstruments:sem}) in EDX mode, it is confirmed for both nanoparticle batches that they only the characteristic signatures of iron, carbon and oxygen are visible and no elemental impurities are present.

  \paragraphNewLine{Small-Angle Scattering}
    Dispersions of IOS-11 and IOS-7 were measured using SAXS at GALAXI (\refch{ch:lss:galaxi}) and SANS(POL) at D33 (\refch{ch:lss:d33}) to determine the electronic, nuclear and magnetic structure of the nanospheres.

    For the SAXS measurement, the dispersions are filled in borosilicate capillaries (Hilgenberg) with $1.5 \unit{mm}$ diameter and a $0.01 \unit{mm}$ wall thickness.
    The capillaries are sealed vacuum-tight on top with a plastic stopper and a liquid glue that cures under UV light (Norland Optical Adhesive).
    Both samples are dispersed in cyclohexane for the measurement, where concentrations in the order of $10 \unit{mg \, mL^{-1}}$ were prepared.
    The samples are measured on the largest ($3.53 \unit{m}$) and shortest ($0.83 \unit{m}$) sample-to-detector distances possible at GALAXI at the Ga-K$\alpha$ wavelength of $\lambda \eq 1.3414 \angstrom$.
    Additionally, a capillary filled with cyclohexane and an empty capillary is measured under the same conditions for subtraction of the background.
    The SAXS data is scaled to absolute units according to the procedure described in \refch{ch:methods:saxs}.

    For the SANSPOL measurement on the D33 instrument, part of the IOS-11 and IOS-7 solutions are dried over night at ambient conditions, and then redispersed in toluene-$\mathit{d_8}$ by sonification.
    The dispersions are measured in Hellma quartz cuvettes with a thickness of $2 \unit{mm}$.
    The wavelength at D33 was set to $6 \unit{\angstrom}$, where the selector provides a wavelength spread of $10 \%$ (FWHM).
    A sample aperture  of $7 \times 10 \unit{mm^2}$ was used and a collimation aperture of $30 \times 30 \unit{mm^2}$ is set for all experiments.
    The samples are measured both at a sample-to-detector distance of $2 \unit{m}$ and at $8 \unit{m}$ with a collimation distance of $5.3 \unit{m}$ and $7.8 \unit{m}$ respectively.
    The nanoparticles are measured at an magnetic field of $515 \unit{mT}$, which is applied perpendicular to the beam direction, in the horizontal direction.
    Each sample is measured over a time of $20 \unit{min}$ at the long sample-to-detector distance and $10 \unit{min}$ at the short sample-to-detector distance.
    The empty beam and a sample of toluene-$\mathit{d_8}$ is additionally measured for background subtraction.
    For the evaluation of the magnetic scattering, a $20^\circ$ sector around the vertical dimension is azimuthally integrated by using the GRASP software.
    Furthermore, the polarization efficiency of $97 \%$ and a flipping efficiency of $99 \%$ of D33, according to the instrument specifications given by the local contact, is corrected with the GRASP software during data reduction.

    From the distances and aperture sizes, an angular divergence of $2.1 \unit{mrad}$ and $3.8 \unit{mrad}$ are calculated for the long and short detector distance respectively by \refeq{eq:methods:sans:formulaResolution}, which is fixed together with the wavelength spread for the data evaluation.

    To evaluate the SAXS data, a spherical core-shell form factor is assumed with a w\"ustite core and a \ch{Fe_{3-$\delta$}O4} shell.
    The scattering length density of the core and shell material is calculated by
    \begin{align}
      \rho^\mathrm{X-ray}_\mathrm{inv.\,spinell}   &\eq   8 r_e \frac{(3-\delta) f_\mathrm{Fe} + 4 f_\mathrm{O}}
                                                                     {a_\mathrm{inv.\,spinell}^3}\\
      \rho^\mathrm{neutron}_\mathrm{inv.\,spinell} &\eq   8 \frac{(3-\delta) b_\mathrm{Fe} + 4 b_\mathrm{O}}
                                                                            {a_\mathrm{inv.\,spinell}^3}\\
      \rho^\mathrm{X-ray}_\textsf{w\"ustite}   &\eq   4 r_e \frac{f_\mathrm{Fe} + f_\mathrm{O}}
                                                                 {a_\textsf{w\"ustite}^3}\\
      \rho^\mathrm{neutron}_\textsf{w\"ustite} &\eq   4 \frac{b_\mathrm{Fe} + b_\mathrm{O}}
                                                             {a_\textsf{w\"ustite}^3}
    \end{align}
    %     \eq & 38.56 \cdot 10^{-6} \angstrom^{-2},\\
    %     \eq & 6.57 \cdot 10^{-6} \angstrom^{-2},\\
    %     \eq & 52.07 \cdot 10^{-6} \angstrom^{-2},\\
    % \eq & 8.34 \cdot 10^{-6} \angstrom^{-2}.
    with $r_e$ the classical electron radius, the  atomic form factor $f$ and nuclear coherent scattering length $b$ of iron and oxygen, both are found in literature for the elements of the periodic table and are tabulated for iron and oxygen in \reftab{tab:looselyPackedNS:charMethod:scatteringLenghts}.
    The lattice constants are estimated from the values that are obtained in XRD for IOS-11.
    The parameter $\delta$ determines whether the shell phase is closer to magnetite ($\delta \eq 0$) or maghemite ($\delta \eq \tfrac{1}{3}$).
    The choice of $\delta$ is also discussed in the evaluation of the XRD data.
    \begin{table}[ht]
      \centering
      \caption{\label{tab:looselyPackedNS:charMethod:scatteringLenghts}Atomic form factor $f$ (at $\lambda \eq 1.3414 \unit{\angstrom}$) and the nuclear coherent scattering length $b$ of iron and oxygen \cite{Sears_1992_Neutr, BerkeleyLab_1993_asf}.}
      \begin{tabular}{ c | l | c }
                  & $f$       & $b \, / \unit{fm}$ \\
        \hline
        $\ch{O}$  & 8.04077   & 5.803   \\
        $\ch{Fe}$ & 25.7468   & 9.45  \\
        \hline
      \end{tabular}
    \end{table}

    The SLD of the surfactant oleic acid is calculated from literature
    \begin{align}
      \rho^\mathrm{X-ray}_\mathrm{oleic\,acid} &\eq 8.52 \cdot 10^{-6} \angstrom^{-2},\\
      \rho^\mathrm{neutron}_\mathrm{oleic\,acid} &\eq 0.078 \cdot 10^{-6} \angstrom^{-2},
    \end{align}
    which assumes a density of $0.895 \unit{g\,mL^{-1}}$ for the oleic acid.
    As well as the solvent scattering length densities, which are fixed to the values calculated from literature
    \begin{align}
      \rho^\mathrm{X-ray}_{\mathrm{cyclohexane}} &\eq 7.55 \cdot 10^{-6} \angstrom^{-2},\\
      \rho^\mathrm{neutron}_{\mathrm{toluene-}d8}  &\eq 5.66 \cdot 10^{-6} \angstrom^{-2},
    \end{align}
    where a density of $0.779 \unit{g\,mL^{-1}}$ for cyclohexane, and $0.943 \unit{g\,mL^{-1}}$ for toluene-$\mathit{d8}$.

    In the minimization procedure of both SAXS and SANS models, a Levenberg-Marquardt algorithm\cite{Marquardt_1963_Analgo, Oliphant_2006_Guide} is applied for which a modified $\chi^2$ on the logarithmic scale is used
    \begin{align}
      \chi^2 \eq \frac{1}{N-p} \sum_{i\eq 1}^{N} \frac{(\log(I_i) - \log(I_\mathrm{model}))^2}{\sigma_i^2} I^2_i.
    \end{align}

    Furthermore, as the particle size and size distribution is very well determined by SAXS, which has a larger dynamic range in scattering vector, these parameters are not varied for the SANS form factor, where vice-versa the surfactant thickness is only determined by SANS and then fixed in the SAXS form factor.
    This is done until both calculated form factors are self-consistent with each other.
    However, the shell thickness of the maghemite phase is allowed to vary independently for both SAXS and SANS, as both experiments are sensitive to the shell thickness and as the SANS particles might have oxidized differently to the SAXS particles due to the solvent changing, where the particles were dry for a night at ambient conditions.

    The obtained particle number density $n$ obtained as parameter from SAS, is transformed to a mass concentration of the particles by
    \begin{align}
      c_m \eq n (\rho_\textsf{w\"ustite} V_p^\textsf{w\"ustite} + \rho_\textsf{inv.\, spinell} V_p^\textsf{inv.\, spinell}),
    \end{align}
    where $V^x_p$ is the average volume of the respective phase $x$ in a single nanosphere and $\rho$ the mass density of w\"ustite and \ch{Fe_{3-$\delta$}O4}.

    For the magnetic form factor in SANSPOL, the geometrical parameters are fixed from the nuclear form factor of SANS and only the scattering length density of the nanoparticles are varied.
    For the core-shell nanospheres, where the shell is supposed to be magnetite and the core w\"ustite, the shell is initiated with a high magnetic scattering length density for the fit, while the core is fixed at zero.
    Only subsequently the core is fitted, while the shell is fixated at the fitted value.
    This bias has to be done, as otherwise the fit favors a core with high magnetic SLD and a shell with low magnetic SLD, which is nonphysical in the context and due to an ambiguity in the sign of the contrast.

  \paragraphNewLine{Vibrating Sample Magnetometry}
    To study the magnetic structure of the nanoparticles, vibrating sample magnetometry (VSM) is applied to measure the macroscopic magnetization of a dispersion as described in \refapp{app:additionalExperimentalTechniques:vsm}.
    For both samples, $40 \unit{\musf L}$ of the same dispersions as used in SANS are used, where the particles are dispersed in toluene-$\mathit{d8}$.
    The vials are measured on a PPMS Evercool II (\refsec{ch:instruments:laboratoryInstruments:vsm}) at $300 \unit{K}$ and at a low temperature of $10 \unit{K}$.
    From the magnetization data at room temperature, the magnetic moment of the single nanospheres is determined by fitting a Langevin curve with excess susceptibility, which is then used together with the size determined by SAXS to scale the data to the magnetic volume as described in \refsec{ch:methods:vsm}.
    Furthermore, for both samples temperature dependent magnetization curve is measured by cooling the sample down to $10 \unit{K}$ once with a field of $10 \unit{mT}$ and once without a field.
    From the low temperature the sample is heated with $1.5 \unit{K \, min^{-1}}$, while measuring the magnetization.
    To adjust for an expansion of the vibration rod, the sample recenters every $10 \unit{min}$.
\end{document}