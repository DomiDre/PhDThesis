\providecommand{\main}{../../../..}
\documentclass[\main/dresen_thesis.tex]{subfiles}

\begin{document}
  \paragraphNewLine{Scanning Electron Microscopy}
    The spin-coated nanospheres on silicon substrate are qualitatively viewed by SEM micrographs measured with a Neon Zeiss 40 (\refsec{ch:instruments:laboratoryInstruments:sem}).
    To obtain cross-sectional views, a piece of a sample is cut on two opposing sides with a diamond cutter and then subsequently broken downward to obtain a clean breaking line, from which SEM measurements can be performed.
    The micrographs are measured at $5 \unit{kV}$ and the images from the backscattering electrons are shown to obtain a strong contrast.

  \paragraphNewLine{Grazing-Incidence Small-Angle X-ray Scattering}
    All spin-coated samples were measured at the beam line BM26B \refsec{ch:lss:BM26B} in the ESRF at a wavelength of $\lambda \eq 1.03 \unit{\angstrom}$.
    For each sample a measurement was performed at a large sample-to-detector distance of $6.54 \unit{m}$ and at a shorter sample-to-detector distance of $2.90 \unit{m}$.
    The collimation slit is set to $0.3 \times 0.5 \unit{mm^2}$ for each measurement, and every sample is evaluated at an incident angle of $0.2 \unit{^\circ}$.

    For both samples, a strip of $0.02 \nm^{-1}$ width along the Yoneda band is integrated and compared to the form factor obtained by small-angle X-ray scattering.
    The scattered intensity for the nanostructure along the Yoneda band is calculated as product of a structure factor and the form factor
    \begin{align}
      I(q) \eq I_0 S(q) |P(q)|^2 + I_\mathrm{bg},
    \end{align}
    where $I_0$ is a scaling factor and $I_\mathrm{bg}$ is a incoherent noise background that is not directly associated with the scattering from the nanoparticles.
    The used form factor $|P(q)|^2$ is hereby given by using the best fit of the nanoparticles from SAXS as determined in \refsec{sec:looselyPackedNS:nanoparticle:sas}.
    As no calibration measurement has been performed at BM26B, the data is given in the arbitrary count units of the detector.
    The structure factor for hard spheres of radius $R_\mathrm{HS}$ and with a packing fraction $\eta$ can be calculated analytically in the Percus-Yervick approximation \cite{Percus_1958_Analy, Wertheim_1963_Exact, Pedersen_1997_Analy} and is given by
    \begin{align}
      S(q) &\eq \frac{1}{1 + 24 \eta \frac{G(2 q R_\mathrm{HS})}{2 q R_\mathrm{HS}} }
    \end{align}
    with
    \begin{align}
      \begin{split}
        G(x)   &\eq \frac{(1 + 2\eta )^2}{(1 - \eta )^4} \frac{ \sin(x) - x \cos(x)}{x^2}\\
               & + \frac {-6 \eta (1 + \eta / 2)^2}{(1 - \eta )^4} \frac{2 x sin(x) + (2 - x^2) \cos(x) - 2}{x^3}\\
               & + \frac{\eta (1 + 2\eta )^2}{2(1 - \eta )^4} \frac{-x^4 \cos(x) + 4 [(3 x^2 - 6) \cos(x) +(x^3 - 6 x) \sin(x) + 6]}{x^5}\\
      \end{split}
    \end{align}

  \paragraphNewLine{X-Ray Reflectometry}
    Yip

  \paragraphNewLine{Vibrating Sample Magnetometry}
    Yip

  \paragraphNewLine{Polarized Neutron Reflectometry}
    Yip
\end{document}