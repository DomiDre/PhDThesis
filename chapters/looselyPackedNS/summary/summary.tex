\providecommand{\main}{../../..}
\documentclass[\main/dresen_thesis.tex]{subfiles}
  \renewcommand{\thisPath}{\main/chapters/looselyPackedNS/summary}

\begin{document}
  In this first chapter, two batches of iron oxide nanospheres, as well as spin-coated layers and bilayer stacks of said nanospheres are characterized structurally and magnetically.
  Seeing that the nanospheres themselves have magnetic properties that deviate strongly from the bulk properties expected from any common iron oxide phase, such as magnetite, maghemite or w\"ustite, the first step is crucial.
  Without, it is impossible to evaluate whether magnetic properties observed in packed nanostructures are a result of the interparticle interaction or a result of the nanoparticle properties themselves.
  \\

  For the batch of larger nanospheres, IOS-11, an inverse spinell structure close to magnetite with anti-phase boundaries in the nanoparticle volume is observed by complimentary small-angle scattering and X-ray diffraction experiments.
  The observed spherical morphology and size is confirmed by electron microscopy and a reduced magnetization due to the anti-phase boundaries is observed by vibrating sample magnetometry and SANSPOL.

  The spin-coated layer shows laterally short-range order and vertically a structure of layered nanospheres.
  Using GISAXS and reflectometry in combination with the results obtained from the nanoparticle characterization, it is shown that the sample can be structurally described by layers of nanospheres with overlapping oleic acid shells.
  The spontaneous magnetization and the magnetic moment of the nanospheres in the layer are in a similar order of magnitude as it is observed for the non-interacting nanospheres in dispersion.
  The layer has an enlarged blocking temperature in comparison to the non-interacting nanospheres in dispersion.
  This enhanced blocking temperature can be interpret as a typical signature of dipolar interaction in the sample.
  As a byproduct of this study, it was observed that the nanosphere layer compresses strongly upon cooling, which as observable effect has not yet been found in literature.

  The bilayer stacks from IOS-11 show qualitatively a similar magnetic behavior in polarized neutron reflectometry as the single layer.
  No peculiar effects are observed in the magnetic splitting of the Bragg peaks in the remanent states after zero-field cooling nor field cooling.
  Neither for the sample with large non-magnetic PMMA spacing between the nanospheres layers, nor for the reduced spacer thickness.
  The observed signal in the spin-flip channel is attributed to spin leakage.
  It can be deduced that if dipolar interlayer interaction is present in the samples, it is a weak effect that can not be observed by a qualitative discussion.
  For a quantitative layer-by-layer evaluation of the magnetization, the samples have proven as being too complex to discuss due to the large number of unknown parameters.
  \\

  For the batch of smaller sized nanospheres IOS-7, small angle scattering suggests also a single-phase of magnetic nanoparticles.
  The same discussion as for IOS-11 has been performed, where the structure and magnetism of the non-interacting nanoparticles in dispersion is determined.
  The spin-coated layer shows a low density packing from GISAXS and reflectometry.
  And magnetically a close agreement in blocking temperature with the nanospheres in dispersion is observed.
  On one hand, a measurable magnetic moment in the order of $220 \unit{kA \, m^{-1}}$ is seen for the spin coated layer from vibrating sample magnetometry, but no strong magnetic splitting is observed in polarized neutron reflectometry on the other hand.
  The bilayer stacks provide a similar story as observed for the samples discussed from IOS-11.
  \\

  Overall the considered system proves to show only subtle differences in the packed structures in direct comparison to the nanospheres in dispersion.
  As dipolar interaction is directional, a simpler nanoparticle arrangement would allow an easier access to correlate observed effects with the nanoparticle superstructure.
  To achieve a better basis for discussion, the obtained results motivate the study on the preparation of thin layers of nanoparticles with a much higher degree of in-plane order.
  The results for the preparation of long-range ordered monolayers is presented in the following chapter.

\end{document}