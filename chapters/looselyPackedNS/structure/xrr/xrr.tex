\providecommand{\main}{../../../..}
\documentclass[\main/dresen_thesis.tex]{subfiles}
\begin{document}
  \label{sec:looselyPackedNS:layers:pnr}
  \begin{figure}[tb]
    \centering
    \includegraphics{looselyPackedNP_VerticalStructure_SC-IOS-11_XRR}
    \includegraphics{looselyPackedNP_VerticalStructure_SC-IOS-7_XRR}
    \caption{\label{fig:looselyPackedNP:layer:xrr}X-ray reflectometry from SC-IOS-11 (left) and SC-IOS-7 (right). The inset shows the scattering length density model of the fitted reflectometry curve (black).}
  \end{figure}

  The depth-resolved average electron density of the layers SC-IOS-11 and SC-IOS-7 is resolved with sub-nm precision using X-ray reflectometry measured on a Bruker D8 Advance.
  The observed reflectivity is shown in \reffig{fig:looselyPackedNP:layer:xrr}, where a footprint correction has been applied with an equidistributed beam profile.
  The reflectivity is fit by assuming that the sample is compromised of multiple stacked spheres, which rest on a silicon substrate.
  The separate sphere layers are allowed each to have a variable two dimensional packing $\eta_i$ and their position can be shifted by $\Delta z_i$ from the ideal position that is calculated for the spheres if they had a hard oleic acid shell and do not penetrate each other.
  The determined parameters for each layer are tabulated in \reftab{tab:looselyPackedNP:nanoparticle:xrr} for both IOS-11 and IOS-7.

  \begin{table}[!htbp]
    \centering
    \caption{\label{tab:looselyPackedNP:nanoparticle:xrr}Parameters for the layer of multiple nanospheres shown in \reffig{fig:looselyPackedNP:layer:xrr}. The parameters $\eta$ are the two dimensional packing density for each layer and $\Delta z$ are the shifts of the layer center from the pitch $\sqrt{8/3} (R+D_\mathrm{OA}$). The other parameters are the thickness of the spacer layer on the substrate $d_\mathrm{spacer}$, its SLD $\rho_\mathrm{spacer}$, the substrate roughness $\sigma$, the rate of roughness increase with layer height $\Delta \sigma$, and the wavelength spread of the instrument $\sigma_\lambda / \lambda$. The nanosphere parameters are fixed from SAXS.}
    \begin{tabular}{ c | l | l }
      \rule{0pt}{2ex} \textbf{XRR}  & \textbf{SC-IOS-11} & \textbf{SC-IOS-7} \\
      \hline
       $\eta_1     \, / \unit{\%}$                                  & $50.9(4)$         & $34(1)$    \\
       $\eta_2     \, / \unit{\%}$                                  & $55.4(4)$         & $32(1)$    \\
       $\eta_3     \, / \unit{\%}$                                  & $52.9(5)$         & $30(2)$    \\
       $\eta_4     \, / \unit{\%}$                                  & $49.8(6)$         & $33(3)$    \\
       $\eta_5     \, / \unit{\%}$                                  & $53.4(8)$         & $38(3)$    \\
       $\eta_6     \, / \unit{\%}$                                  & $28.6(1.0)$       & $40(3)$    \\
       $\eta_7     \, / \unit{\%}$                                  &                   & $41(3)$    \\
       $\eta_8     \, / \unit{\%}$                                  &                   & $44(3)$    \\
       $\eta_9     \, / \unit{\%}$                                  &                   & $43(7)$    \\
       $\eta_{10}     \, / \unit{\%}$                               &                   & $34(9)$    \\
       $\eta_{11}     \, / \unit{\%}$                               &                   & $8(10)$    \\
       \hline
       $\Delta z_1 \, / \unit{nm} $                                 & $-1.04(3)$        & $-1.33(6)$ \\
       $\Delta z_2 \, / \unit{nm} $                                 & $-2.60(3)$        & $-1.33(7)$ \\
       $\Delta z_3 \, / \unit{nm} $                                 & $-2.68(3)$        & $-1.13(8)$ \\
       $\Delta z_4 \, / \unit{nm} $                                 & $-2.95(4)$        & $-1.22(6)$ \\
       $\Delta z_5 \, / \unit{nm} $                                 & $-2.67(4)$        & $-1.40(6)$ \\
       $\Delta z_6 \, / \unit{nm} $                                 & $-1.82(10)$       & $-1.46(7)$ \\
       $\Delta z_7 \, / \unit{nm} $                                 &                   & $-1.41(9)$ \\
       $\Delta z_8 \, / \unit{nm} $                                 &                   & $-1.2(1)$   \\
       $\Delta z_9 \, / \unit{nm} $                                 &                   & $-0.4(5)$    \\
       $\Delta z_{10} \, / \unit{nm} $                              &                   & $-2.4(8)$   \\
       $\Delta z_{11} \, / \unit{nm} $                              &                   & $-1.5(19)$  \\
       \hline
       $d_\mathrm{spacer}   \, / \unit{nm} $                        & $5.15(3)$         & $5.27(2)$  \\
       $\rho_\mathrm{spacer}\, / \unit{10^{-6} \angstrom^{-2}} $    & $11.9(2)$         & $2.2(1)$ \\
       $\sigma     \, / \unit{nm} $                                 & $0.65(1)$         & $0.70(1)$  \\
       $\Delta \sigma$                                              & $0.034(1)$        & $0.036(1)$ \\
       $\sigma_\lambda / \lambda\, / \unit{\%}$                     & \multicolumn{2}{c}{$1.1(2)$} \\
      \hline
       $R             \, / \unit{nm}$                               & $5.41$         & $3.54$ \\
       $D_\mathrm{OA} \, / \unit{nm}$                               & $1.82$         & $1.69$ \\
       $\sigma_R      \, / \unit{\%}$                               & $5.45$         & $7.52$ \\
       $\rho_\mathrm{core}\, / \unit{10^{-6} \angstrom^{-2}}      $ & \multicolumn{2}{c}{$40.5$}\\
       $\rho_\mathrm{shell}\, / \unit{10^{-6} \angstrom^{-2}}     $ & \multicolumn{2}{c}{$8.52$}\\
       $\rho_\mathrm{substrate}\, / \unit{10^{-6} \angstrom^{-2}} $ & \multicolumn{2}{c}{$20.1$}\\
      \hline
    \end{tabular}
  \end{table}

  The illustrative model is able to describe the characteristics of the two obtained reflectivity curves for most parts.
  The SLD profile of SC-IOS-11 shows five layers of nanospheres with an additional sixth layer of decreased packing fraction.
  The two dimensional packing fractions $\eta$ are in the order of $50 \unit{\%}$ for the five layers and $29 \unit{\%}$ for the upper layer.
  $\Delta z$ is in all cases negative in the order of $1 - 3 \unit{nm}$.
  For SC-IOS-7 the same observation in $\Delta z$ can be made with a lesser degree of inter-mixing as the values are in the order of $1 - 1.5 \unit{nm}$.
  Furthermore in the case of SC-IOS-7 the packing fraction $\eta$ are lower being in an order of magnitude of $30 - 40 \unit{\%}$.
  The spacer thickness is in both cases approximately $5 \unit{nm}$ and the additional roughness parameter in an order of $0.7 \unit{nm}$ at the substrate.
  In both cases, the roughness increases with a slope of $0.035$ with respect to the layer height, yielding a roughness of approximately $2.5 \unit{nm}$ near the sample surface.

  The fitted roughness parameter is to be understood in this context as a model of the spread in the respective layer positions across the measured area within the reflectivity experiment.
  The increasing roughness with layer height is therefore to be understood as a larger uncertainty in the positions of the specific layer across the sample.

  The parameters, excluding the roughness, are used to generate the depiction of the sphere stacking shown in \reffig{fig:looselyPackedNP:layer:xrrDepiction}.
  Here, the two-dimensional packing fraction $\eta$ are translated into an average particle distance in the layer by
  \begin{align}
    r_\mathrm{pp} \eq \sqrt{\frac{\eta_\mathrm{CP}}{\eta}} 2 (R + D_\mathrm{OA}),
  \end{align}
  where $\eta_\mathrm{CP}$ is the dense circle packing fraction given by $\eta_\mathrm{CP} \eq \pi / \sqrt{12} \approx 90.7 \%$.
  From the depiction it becomes visible for IOS-SC-11 that the shift down from the calculated pitch correlates with the lowered lateral packing density as the spheres fill the voids.
  Additionally it is visible that the oleic acid shells of stacked particles overlap to some part.
  For IOS-SC-7 the particles are more spaced relative to each other and the density of the packing increases from the lower to the higher parts of the layer slightly.

  \begin{figure}[tb]
    \centering
    \includegraphics{looselyPackedNP_VerticalStructure_SC-IOS-11_XRRDepiction}
    \includegraphics{looselyPackedNP_VerticalStructure_SC-IOS-7_XRRDepiction}
    \caption{\label{fig:looselyPackedNP:layer:xrrDepiction}Depiction generated from the fitted parameters in \reftab{tab:looselyPackedNP:nanoparticle:xrr} showing the average particle distance and layer packing.}
  \end{figure}

\end{document}