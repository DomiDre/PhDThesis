\providecommand{\main}{../../../..}
\documentclass[\main/dresen_thesis.tex]{subfiles}
\begin{document}
  \label{sec:looselyPackedNS:layers:gisaxs}
  GISAXS is used to estimate the the packing density and average particle-to-particle distance by evaluating the structure factor.
  In \reffig{fig:looselyPackedNP:layer:gisaxs} GISAXS for SC-IOS-11 and SC-IOS-7 is shown, as well as the intensity in the Yoneda band and the nanoparticle form factor obtained from SAXS.
  The GISAXS patterns are in both cases dominated by the form factor of the nanospheres, visible by the rings.
  Slight modulations of the scattering in the rings are visible, giving rise to a structure factor.
  Due to the weak intensity of those modulations, the order from the structure factor is short-ranged.

  By using the parametrization of the form factor from SAXS determined in \refsec{sec:looselyPackedNS:nanoparticle:sas}, a hard-sphere structure factor is determined from the data to quantitatively evaluate the short-range order, which is also shown in \reffig{fig:looselyPackedNP:layer:gisaxs}.
  For SC-IOS-11, the hard-sphere radius is determined to $5.655(2) \unit{nm}$ and the packing fraction to $43.88(3) \%$.
  For SC-IOS-7, a hard-sphere radius of $3.872(4) \unit{nm}$ and packing fraction of $34.20(9) \%$ is observed.

  \begin{table}[!htbp]
    \centering
    \caption{\label{tab:looselyPackedNP:nanoparticle:gisaxs}Parameters for the hard-sphere structure factor in Percus-Yervick approximation shown in \reffig{fig:looselyPackedNP:layer:gisaxs} for both SC-IOS-11 and SC-IOS-7. $R_\mathrm{HS}$ is the hard-sphere radius and $\eta$ the packing fraction of the structure factor.}
    \begin{tabular}{ c | l | l }
      \rule{0pt}{2ex} \textbf{GISAXS}  & \textbf{SC-IOS-11} & \textbf{SC-IOS-7} \\
      \hline
      \rule{0pt}{2ex} $R_\mathrm{HS} \, / \unit{nm}$          & $5.655(2)$           & $3.872(4)$\\
      \rule{0pt}{2ex} $\eta          \, / \unit{\%}$          & $43.88(3)$           & $34.20(9)$\\
      \hline
    \end{tabular}
  \end{table}
  % \rule{0pt}{2ex} $I_0           \, / \unit{a.u.}$        & $2550(4)$            & $22405(53)$\\
  % \rule{0pt}{2ex} $I_\mathrm{bg} \, / \unit{a.u.}$        & $411(10)$            & $147(10)$\\
  \begin{figure}[tb]
    \centering
    \includegraphics{looselyPackedNP_GISAXS_SC-IOS-11}
    \includegraphics{looselyPackedNP_GISAXS_StructureFactor_SC-IOS-11}
    \includegraphics{looselyPackedNP_GISAXS_SC-IOS-7}
    \includegraphics{looselyPackedNP_GISAXS_StructureFactor_SC-IOS-7.png}
    \caption{\label{fig:looselyPackedNP:layer:gisaxs}GISAXS detector images (left) of IOS-11 (upper) and IOS-7 (lower) measured at the BM26B beam line at the ESRF under an incident angle of $\alpha_i \eq 0.2 \unit{^\circ}$. The  integrated data in the Yoneda band (right, blue curve), marked by the white stripe on the detector images, is fitted to the hard-sphere structure factor in Percus-Yervick approximation, with parameters listed in \reftab{tab:looselyPackedNP:nanoparticle:gisaxs}. The SAXS form factor of the nanoparticles in dispersion is shown for comparison (red curve). The black area around $q_y \eq 0 \unit{nm^{-1}}$ is the beam stop, whereas the small black rectangles in the detector images are gaps due to inactive areas of the detector. }
  \end{figure}

  In comparison to the particle radius of IOS-11 and IOS-7 obtained by SAXS ($5.4 \unit{nm}$ and $3.6 \unit{nm}$ respectively), the hard-sphere radius is increased by approximately $0.3 \unit{nm}$ in both cases.
  This is less than the estimated length of an oleic acid chain, which is in the order of $2.1 \unit{nm}$.
  This can be interpreted as the particles penetrating the oleic acid shells of one another, and that the oleic acid shells do not act as hard repulsive shells.

  The packing fractions on the other hand are in comparison to a dense spherical packing ($\pi / \sqrt(18) \approx 74 \%$) strongly reduced.
  The observed values are even lower than the expected packing density for loosely random packed spheres ($58 \%$ \cite{Tory_1973_Simul, Shi_2008_Simul}).
  This suggests that additional voids are present in the particle layer and that particles have on average a larger spacing than the hard-sphere radius.
  Estimating the volume an particle occupies on average by
  \begin{align}
    \bar{r} \eq \frac{R_\mathrm{HS}}{\sqrt[3]{\eta}},
  \end{align}
  a mean radius of $7.44 \unit{nm}$ is obtained for SC-IOS-11 and $5.5 \unit{nm}$ for SC-IOS-7.
  These values are approximately $2 \unit{nm}$ larger than the particle radius obtained by SAXS, from which it can be concluded that even though the oleic acid shell can be penetrated, on average the particles are separated by their oleic acid shells as it would be expected.

\end{document}