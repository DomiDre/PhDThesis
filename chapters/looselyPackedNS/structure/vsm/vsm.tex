\providecommand{\main}{../../../..}
\documentclass[\main/dresen_thesis.tex]{subfiles}
\begin{document}
  \label{sec:looselyPackedNS:layer:vsm}
  \begin{figure}[tb]
    \centering
    \includegraphics{looselyPackedNP_VSM_SC-IOS-11}
    \includegraphics{looselyPackedNP_VSM_SC-IOS-7}
    \caption{\label{fig:looselyPackedNP:layer:vsm}Room temperature vibrating sample magnetometry for SC-IOS-11 (left) and SC-IOS-7 (right). In black a Langevin behaviour is fit and the parameters are given in \reftab{tab:looselyPackedNP:layer:vsm}}
  \end{figure}

  For the magnetic properties of the loosely packed nanospheres, vibrating sample magnetometry provides a method to study the the macroscopic magnetization of SC-IOS-11 and SC-IOS-7.
  The field-dependent magnetization of both samples at room temperature is shown in \reffig{fig:looselyPackedNP:layer:vsm}.
  In both cases a superparamagnetic behaviour is observed with no excess susceptibility.
  The magnetization are well described by a Langevin curve with the parameters given in \reftab{tab:looselyPackedNP:layer:vsm}.

  \begin{table}[!htbp]
    \centering
    \caption{\label{tab:looselyPackedNP:layer:vsm}Parameters for the Langevin function shown in \reffig{fig:looselyPackedNP:layer:vsm}, with $\mu$ the magnetic moment and $M_s$ the spontaneous magnetization. Additionally given is the magnetic moment determined for the nanospheres in dispersion $\mu_\mathrm{disp.}$ in \refsec{sec:looselyPackedNS:nanoparticle:vsm}.}
    \begin{tabular}{ c | l | l }
      \rule{0pt}{2ex} \textbf{VSM \@ 300K} & SC-IOS-11 & SC-IOS-7 \\
      \hline
      \rule{0pt}{2ex} $\mu \, / \, \mu_B$           & $13650(46)$   & $4374(16)$\\
      \rule{0pt}{2ex} $M_s \, /  \unit{kAm^{-1}}$   & $192.4(1)$    & $219.8(3)$\\
      \hline
      $\mu_\mathrm{disp.} \, / \, \mu_B$            & $11354(58)$   & $3609(8)$\\
      \hline
    \end{tabular}
  \end{table}
  The determined magnetic moment from the Langevin curve is in both cases approximately $20 \%$ larger than the magnetic moment obtained for the samples measured in dispersion in \refsec{sec:looselyPackedNS:nanoparticle:vsm}.
  % For SC-IOS-11 it is possible that the nanospheres on the silicon substrate have oxidated partially in comparison to the nanospheres in dispersion and therefore a higher magnetization per particle is achieved, and for SC-IOS-7 the obtained value is still in agreement with the result from SANSPOL for IOS-7 (\refsec{sec:looselyPackedNS:nanoparticle:sas}).

  \begin{figure}[tb]
    \centering
    \includegraphics{looselyPackedNP_VSM_ZFC_FC_SC-IOS-11}
    \includegraphics{looselyPackedNP_VSM_ZFC_FC_SC-IOS-7}
    \caption{\label{fig:looselyPackedNP:layer:vsmZFCFC}Temperature dependent magnetization measured at $10 \unit{mT}$ after cooling in zero-field and at a field of $10 \unit{mT}$ for SC-IOS-11 (left) and SC-IOS-7 (right).}
  \end{figure}

  Furthermore, temperature-dependent magnetization measurements of the spin-coated layers are shown in \reffig{fig:looselyPackedNP:layer:vsmZFCFC}.
  The blocking temperature of SC-IOS-11 is determined from the ZFCw measurement to $102(1) \unit{K}$ and for SC-IOS-7 to $47(1) \unit{K}$.
  For SC-IOS-7 the value is well in agreement with the blocking temperature measured for the nanospheres IOS-7 in dispersion (\refsec{sec:looselyPackedNS:nanoparticle:vsm}).
  But for SC-IOS-11 the value is significantly increased in comparison to a blocking temperature of $95.5(5) \unit{K}$ obtained for IOS-11.

  The increased magnetic moment and blocking temperature of SC-IOS-11 can either be the result from a coupling between the spheres, which could conceivable result in increased size of the coherent magnetic domains and a higher thermal energy required to turn the sample into the superparamagnetic phase.
  Or it can be the result from a further progressed oxidation of the nanoparticles in the loosely packing on the silicon substrate in comparison to the nanospheres in organic dispersion.
  Due to the ambiguity of the observation, the macroscopic magnetization measurement of the nanospheres are insufficient to conclude on magnetic nanoparticle interaction.

  \begin{figure}[tb]
    \centering
    \includegraphics{looselyPackedNP_VSM30K_SC-IOS-11}
    \includegraphics{looselyPackedNP_VSM30K_SC-IOS-7}
    \caption{\label{fig:looselyPackedNP:layer:vsm30K}Temperature dependent magnetization measured at $10 \unit{mT}$ after cooling in zero-field and at a field of $10 \unit{mT}$ for SC-IOS-11 (left) and SC-IOS-7 (right).}
  \end{figure}
  In a next step, polarized neutron reflectometry is therefore used to study the nanostructure magnetization with depth-resolution.
  For the discussion of the reflectivity curves measured at low temperatures, the analogue measurements have been performed on the VSM in \reffig{fig:looselyPackedNP:layer:vsm30K}.
  Both samples have been zero-field cooled and field cooled, before a hysteresis up to a field of $730 \unit{mT}$ is measured at $30 \unit{K}$.
  In both cases a small hysteresis with a coercive field below $20 \unit{mT}$ is visible and the field cooled hysteresis is shifted with respect to the zero-field curve.
  The shift is typically observed for magnetite nanoparticles from the oleate synthesis and comes from an exchange-bias effect that originates either from the core-shell structure of the nanospheres or also possibly between the anti-phase boundaries \cite{Wetterskog_2013_Anoma}.




  % At low temperatures the magnetization of IOS-11 and IOS-7 in \reffig{fig:looselyPackedNP:nanoparticle:vsm10} show a hysteretic behaviour, where IOS-11 has a coercive field of $40 \unit{mT}$ and IOS-7 a field of $25 \unit{mT}$.
  % The temperature dependent curves furthermore show the blocking temperature of IOS-11 to be at $100 \unit{K}$ and for IOS-7 at $45 \unit{K}$.
  % The smaller values observed for IOS-7 is connected to the larger population of small-sized nanoparticles, which even at low temperatures are more susceptible to thermal fluctuations.

\end{document}