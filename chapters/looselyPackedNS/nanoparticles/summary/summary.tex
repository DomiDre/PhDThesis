\providecommand{\main}{../../../..}
\documentclass[\main/dresen_thesis.tex]{subfiles}

\begin{document}
  \label{sec:looselyPackedNS:nanoparticle:discussion:summary}
  The obtained nanoparticles are characterized structurally and magnetically by multiple complimentary experiments.

  Using TEM and SAS the spherical shape is confirmed and the particle diameter of the nanospheres IOS-11 is determined to be $10.9(1) \unit{nm}$.
  Using XRD in combination with SAS, a core-shell structure with a w\"ustite core and magnetite shell could be confirmed.
  The particle size distribution is in the order of $5.4 \unit{\%}$.
  It became apparent by the comparison of the results from SAXS, SANS and XRD that the exact phase of the nanospheres varies greatly as the magnetite shells continues to oxidize with time and contact to oxygen, turning the w\"ustite core slowly to magnetite, which is in accordance with current literature on iron oxide nanoparticles .
  The magnetization of the nanosphere IOS-11 was determined to be in the order of $200 \unit{kA \, m^{-1}}$ if the magnetic moment is averaged on the whole particle volume.
  From SANSPOL, it can further be deduced that the magnetization comes primarily from the shell, which is close to the magnetite bulk value with $453(14) \unit{kA m^{-1}}$.

  For IOS-7, the same characterization has been performed in parallel.
  The spherical shape and size is also confirmed by TEM and SAS, where additionally a bimodal distribution is observed by both experiments.
  The primary mode of the distribution has a diameter of $7.0(1) \unit{nm}$ and size distribution of $7.5 \unit{\%}$, whereas the second mode consists of small nanoparticles in the order of $1 \unit{nm}$ with a large size distribution of $60 \%$.
  The nanospheres of IOS-7 appear to be fully oxidized, as far as can be told from the SAS data evaluation and the nanosphere magnetization is determined to $199(14) \unit{kA \, m^{-1}}$ by SANSPOL.

  The order of magnitude for the average nanosphere magnetization is in both cases reproduced by VSM of the dispersions.
  From low-temperature magnetization measurements, a blocking temperature of $96 \unit{K}$ and $47 \unit{K}$ and a coercive field of $15(1) \unit{mT}$ and $10(1) \unit{mT}$ is observed for IOS-11 and IOS-7 at $30 \unit{K}$ respectively.

  The results obtained from the nanospheres in dispersion are in the following the reference properties for the characterization of samples of the same nanospheres spin-coated on silicon substrates.
  As the results show significant deviation from the magnetic properties of bulk magnetite in the nanospheres, this pre-characterization is indispensable for a proper discussion of higher order nanostructures.

\end{document}