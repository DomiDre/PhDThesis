\providecommand{\main}{../../../..}
\documentclass[\main/dresen_thesis.tex]{subfiles}

\begin{document}
  \label{sec:looselyPackedNS:nanoparticle:discussion:summary}
  The obtained nanoparticles are characterized structurally and magnetically by multiple complimentary experiments.

  Using TEM and SAS the spherical shape is confirmed and the particle diameter of the nanospheres IOS-11 is determined to be $10.58 \unit{nm}$  with a particle size distribution in the order of $5.4 \unit{\%}$.
  Using XRD in combination with SAS, it was confirmed that the nanoparticles are in an inverse spinell phase but show effects emerging from the topotaxial oxidation from an initial w\"ustite phase.
  This becomes apparent in the discrepancy of the reflections in XRD, where the peaks associated with the tetrahedral sublattice are broader than the peaks associated with the octahedral sublattice.
  Furthermore, the particles show a reduced magnetization in SANSPOL, which is for the average particle in the order of $220(14) \unit{kA \, m^{-1}}$, and $344(21) \unit{kA \, m^{-1}}$ if furthermore a magnetically dead surface layer in the order of $0.7(1) \unit{nm}$ is accounted for in the magnetic profile.
  The complimentary room temperature magnetization measurement of the nanosphere IOS-11 determines a particle magnetization of $208(1) \unit{kA \, m^{-1}}$, scaled to the total particle volume, and is therefore in excellent agreement with SANSPOL.
  \\

  For IOS-7, the same characterization has been performed in parallel.
  The spherical shape and size is also confirmed by TEM and SAS, where additionally a bimodal distribution is observed in TEM.
  The primary mode of the distribution has a diameter of $6.97(6) \unit{nm}$ in TEM and the evaluation of the SAS data by a spherical form factor yields the nearly exact same particle diameter of $6.98(1) \unit{nm}$.
  The size distribution is determined by SAS to $9.6(2) \unit{\%}$.
  The nanospheres of IOS-7 appear also to be fully oxidized by the SAS data evaluation and the nanosphere magnetization is determined to $151(10) \unit{kA \, m^{-1}}$ by SANSPOL.
  The VSM evaluation on the other hand shows a slight discrepancy with a result of $216(1) \unit{kA \, m^{-1}}$ as magnetization if scaled to the nanoparticle volume.
  This systematic deviation is argued from the approximation where the second mode observed in TEM is neglected in the analysis of the IOS-7 data.
  \\

  From low-temperature magnetization measurements, a blocking temperature of $96 \unit{K}$ and $47 \unit{K}$ and a coercive field of $15(1) \unit{mT}$ and $10(1) \unit{mT}$ is observed for IOS-11 and IOS-7 at $30 \unit{K}$.
  Furthermore, a shift of the hysteresis is observed for both samples, depending on whether the sample is field cooled or zero-field cooled.
  This effect is known in literature for oleate based nanoparticles and is also observed after full oxidation, which indicates that it originates from the anti-phase boundaries in the nanoparticle volume that result from the topotaxial oxidation \cite{Wetterskog_2013_Anoma}.

  The results obtained from the nanospheres in dispersion are in the following the reference properties for the characterization of samples of the same nanospheres spin-coated on silicon substrates.
  As the results show significant deviation from the magnetic properties of bulk magnetite in the nanospheres, this pre-characterization is indispensable for a proper discussion of higher order nanostructures.

\end{document}