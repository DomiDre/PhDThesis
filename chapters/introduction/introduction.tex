\providecommand{\main}{../..}
\documentclass[\main/dresen_thesis.tex]{subfiles}

\begin{document}
\chapter{Introduction}\label{ch:introduction}
  Nanoscience and nanotechnologies are currently seen in the early 21st century as huge potential to multiple technological advancements and breakthroughs for example in the fields of medicine \cite{Popat_2011_Nanot, Thanh_2012_Magne}, energy production and storage \cite{Huebler_2009_Digit, Shinn_2012_Nucle} or electronics and information technology \cite{Waser_2012_Nanoe, Duart_2006_Nanot}.
  The hype and prospects of nanotechnology are so large that funding by government agencies, venture capitalists and private businesses across the globe exceeds billions of dollar annually \cite{McCray_2005_Wills}, as everyone hopes to spark and be part of the next industrial revolution \cite{Guston_2010_Encyc}.

  Information technology is possibly the most famous field where nanotechnology has impacted everyone's daily life already.
  With the miniaturization of electronics to the nanometer scale, modern day circuit boards found in every personal computer and mobile phone contain billions of transistors with size extensions of mere $10 \unit{nm}$ (2017) \cite{Courtland_2017_Moores} and single bits of information can be stored nowadays in areas extending only $22 \times 22 \unit{nm^2}$ (2014) \cite{Fontana_2015_Volum}.
  Reducing the size to the nanometer range brought increase in clock speed, higher energy efficiency and reduced the price per computing  and storage unit dramatically \cite{Courtland_2017_Moores}.
  The down scaling in electronics has been an with exponential pace ongoing process for over half a century, doubling the number of silicon based transistors per microchip since the 1960s, which is famously known as Moore's law.
  This law is nearing it's end as atomic sizes and thus physical limits are being reached \cite{Waldrop_2016_Moret}.
  As the advancement in electronic devices can not be done by reducing the size much further, new designs are necessary \ie by designing three-dimensional circuits or by using alternative materials to silicon to achieve new physical properties \cite{Waldrop_2016_Moret}.
  For example, the developing field of spintronics studies electronic devices that make additional use of the spin degree of freedom in electron currents instead of only the charge \cite{Wolf_2001_Spint}.
  Typically, devices studied in spintronics are layers of different materials with varying physical properties where the combination leads to the new emergent physics \cite{Zutic_2004_Spint}.
  The preparation of such systems is nowadays commonly performed by top-down processes such as lithography, molecular beam epitaxy or vapor deposition.

  An alternative that mimics nature's way of building materials are bottom up processes, where nanometer sized building blocks are synthesized from the atomic or molecular level by chemical processes and higher order nanostructures are prepared by self-assembly methods \cite{Hannink_2006_Nanos, Whitesides_2002_Selfa}.
  Self-organized materials inherit the properties of their small components and can emerge collective properties from the long-range order and interacting forces.
  Furthermore, it also allows to create multi-functional materials by combinations of different physical properties, \eg in binary nanoparticle superlattices \cite{Redl_2003_Three}.
  Bottom-up processes have the promise to fabricate structures with precision and cheaper production cost than those conventionally associated with top-down approaches \cite{Hannink_2006_Nanos}.
  The precise control of the self-assembly process requires a deep understanding of the underlying driving forces and is still an active field of study \cite{Whitelam_2008_Thero, Powers_2016_Track, Josten_2017_Super}.

  Once nanoparticular building blocks are self-assembled, the system presents an ideal model to study the interparticle interaction on the length scales defined by the nanostructure.
  As the nanoparticles can be studied separately in a non-interacting state in dispersion, it is possible by direct comparison of the single-particle properties with the properties of the nanostructure to deduce on emergent effects that result from the interparticle interaction.
  Magnetic nanoparticles \cite{Lu_2007_Magne, Gubin_2005_Magne} are hereby interesting for the study of dipolar interparticle interaction.
  As dipolar interaction is dependent on the relative distance and orientation of two magnetic moments, the degree of structural order is directly correlated to the nature of the interparticle interaction.
  In relevant technical applications such as high-density magnetic recording media on the nanometer scale, dipolar interaction is an unwanted feature as one would want to have the information stored in the magnetic state to be non-interacting.
  Exact knowledge on the strengh and modifications upon reducing single magnetic islands to ever closer sizes and to ever closer distance to one another are therefore indispensable in the future design of magnetic storage devices.

  \paragraphNewLine{Concept}
    In this thesis four types of magnetic nanostructures are studied with focus on determining effects that result from collective magnetism.
    For this purpose, the nanoparticles are studied in each case separately in a dispersion to determine the non-interacting state.
    Subsequently the nanostructures are characterized structurally to determine the relevant length scales and assess the degree of order in the sample.
    With this knowledge the magnetism of the samples is discussed and compared to the single nanoparticle properties as well as with expectations from dipolar interparticle interaction.

    The four nanostructures are primarily differentiated by their increasing degree of long-range order.
    In \refch{ch:looselyPackedNS}, loosely packed thin layers of nanospheres are discussed that are prepared by spin coating and where no long-range degree of in-plane order is expected.
    Then, in \refch{ch:monolayers} a method to prepare long-range ordered monolayers of nanoparticles in a two-dimensional lattice is presented and studied.
    The monolayer preparation is extended in \refch{ch:doublelayers} to double layers, which are separated by a non-magnetic spacer material to furthermore study interlayer dipolar interaction with respect to the spacer thickness.
    And finally in \refch{ch:colloidalCrystals}, the discussion is lead on three-dimensional long-range ordered nanocubes in colloidal crystals of varying thickness.

    In each study ferrite nanoparticles of either iron oxide or cobalt ferrite are used.
    As this thesis grew organically and the monolayer preparation method had essentially to be developed for this thesis first, each nanostructure is prepared from separate batches of nanoparticles.
    But as a detailed nanoparticle characterization is performed for every batch respectively, this should not provide an issue in the goal of determining effects from dipolar interaction.

    The nanoparticles are characterized in each case by small-angle X-ray and neutron scattering, as well as X-ray diffraction, electron microscopy and vibrating sample magnetometry to obtain detailed information on the structural and magnetic properties.
    The latter two are then also applied to the subsequent study of the nanostructures, as well as grazing-incident scattering and reflectometry by X-rays and polarized neutrons to resolve the three-dimensional structure and magnetism on the nanometer length scale.
\end{document}
