\providecommand{\main}{../../..}
\documentclass[\main/dresen_thesis.tex]{subfiles}

\begin{document}
  Studying the dipolar interaction of magnetic nanoparticles quantitatively requires in an experiment the knowledge of the single-particle properties, the relative distance of the particles and their relative orientation.
  In the previous chapter of monolayers it was presented how the distance of the particles and their relative orientation are set by the self-assembled formation of a long-range ordered square lattice that enables the discussion of the interaction.
  To take the discussion further, it is interesting to study how such an ordered collection of nanoparticles in a monolayer interact with another monolayer.
  A natural idea is therefore to extend the two-dimensional layer structure into the third dimension by adding a second layer on top.
  The distance between the two layers can be varied by introducing an intermediate spacer layer with variable thickness.
  A comparison of the magnetization observed for such double layered structures with respect to the monolayer magnetization.
  Despite the simplicity of this approach, a thorough literature research was not able to yield a systematic study on double layers of nanoparticles.
\end{document}