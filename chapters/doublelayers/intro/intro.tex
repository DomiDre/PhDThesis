\providecommand{\main}{../../..}
\documentclass[\main/dresen_thesis.tex]{subfiles}

\begin{document}
  Studying the dipolar interaction of magnetic nanoparticles quantitatively requires in an experiment the knowledge of the single-particle properties, the relative distance of the particles and their relative orientation.
  In the previous chapter of monolayers the distance of the particles and their relative orientation are set by the formation of a long-range ordered square lattice that enables the discussion of the interaction.
  To take the discussion further, it is interesting to study how such an ordered collection of nanoparticles interact with one another.
  A natural idea is to extend the two-dimensional layer structure into the third dimension by adding a second layer on top.
  Despite the simplicity of this approach, a thorough literature research was not able to yield a systematic study on double layers of nanoparticles.

  One popular approach 
\end{document}