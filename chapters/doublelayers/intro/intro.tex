\providecommand{\main}{../../..}
\documentclass[\main/dresen_thesis.tex]{subfiles}

\begin{document}
  In the chapter of the loosely packed nanospheres (\refch{ch:looselyPackedNS}), it became apparent that a complex layer structure as obtained by spin-coating makes it challenging to resolve the vertical structure of magnetic nanoparticle multilayers that are separated by non-magnetic spacers.
  The preparation procedure for monolayers in the last chapter (\refch{ch:monolayers}) provides a method to obtain structurally well defined layers.
  Adding a non-magnetic spacer on top and subsequently repeating the monolayer drop-casting procedure on that, samples of separated single nanoparticle layers can be prepared.
  It is conceivable that a double layer material could obtain new properties in comparison to a monolayer arising from interlayer interactions.
  Furthermore, layering of the monolayers allows to enhance the signal from properties of the two-dimensional system by increasing the magnetic volume.
  Despite the simplicity of this approach and many studies on monolayer samples, a systematic study on double layers of magnetic nanoparticles has not been reported to date.
  Conceivable reasons are possible limitations in either producing reproducible homogeneous monolayers \cite{Mishra_2015_Polar, Bodnarchuk_2010_Large}, or monolayers prepared by a method such as Langmuir-Schaefer \cite{Ukleev_2017_Polar} are unpractical for usage in a layer-by-layer deposition procedure.
  \\

  The interaction of magnetic layers with non-magnetic spacers is studied in literature for layered systems prepared by molecular-beam epitaxy.
  A famous system hereof is the Fe/Cr/Fe sandwich structure studied by Binasch and Gr\"unberg \etal \cite{Binasch_1989_Enhan}, where $12 \unit{nm}$ thick iron layers are separated by a non-magnetic chromium layer in the order of  $1 \unit{nm}$.
  In such systems, the Nobel prize winning giant magneto-resistance was discovered, where the electrical resistance depends on whether the magnetizations in the two adjacent magnetic layers are parallel or antiparallel.
  It was found that the coupling between the iron layers oscillates between ferromagnetic and anti-ferromagnetic as a function of the distance between the layers due to the complex nature of the interlayer exchange coupling \cite{Demokritov_1998_Biqua}.
  % The interlayer exchange coupling between the two ferromagnetic layers with magnetization $\vec{m}_1$, $\vec{m}_2$ that are separated by a non-magnetic spacer is phenomenologically given by \cite{Demokritov_1998_Biqua}
  % \begin{align}
  %   E_\mathrm{IEC} \eq -J_1 \vec{m}_1 \cdot \vec{m}_2 - J_2 (\vec{m}_1 \cdot \vec{m}_2)^2.
  % \end{align}
  % This included ferromagnetic coupling of the layers ($J_1 > 0$), antiferromagnetic coupling ($J_1 <0$) and higher order effects by the biquadratic exchange coupling $J_2$.
  % Nonetheless, the study shows how a simple double layer system could open the door to previously unknown physical effects.

  To study systematically double layers of long-range ordered nanocubes, samples are prepared by sequentially performing the drop casting procedure and the interlayer distance is controlled by adding non-magnetic PMMA as spacer material of varied thickness in between.
  Doing this, a study comparing the magnetization for such double layered nanostructures is performed with respect to the varied interlayer distance and any changes in the magnetic properties can be attributed to the interlayer interaction.

  As naturally interlayer distances are several $\unit{nm}$ thick by the oleic acid surfactant of the nanoparticles, it should not be expected to observe interlayer exchange coupling between nanoparticle double layers such as in the thin layer systems of Fe/Cr/Fe.
  However, it is still possible that the layers interact via dipolar coupling.
  For an in-plane magnetization, dipolar interlayer coupling prefers the antiparallel alignment between two homogeneously magnetized layers with a non-magnetic spacer \cite{Labrune_2002_Dipol}.
  Due to the complex square array arrangement of the nanocubes, however, more complex and therefore richer ground states are conceivable, for which no theoretical study could be found in literature yet.
\end{document}