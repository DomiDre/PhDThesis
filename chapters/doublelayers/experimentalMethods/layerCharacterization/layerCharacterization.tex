\providecommand{\main}{../../../..}
\documentclass[\main/dresen_thesis.tex]{subfiles}

\begin{document}
  \paragraphNewLine{Scanning Electron Microscopy}
    The double layers with varied spacer thickness are characterized qualitatively by scanning electron microscopy.
    A Neon Zeiss 40 (\refsec{ch:instruments:laboratoryInstruments:sem}) is operated to obtain top-view and cross-sectional micrographs of each sample.
    For the cross-sectional views, the samples are cut on two opposing sides with a diamond cutter and broken downward.
    The micrographs are measured at $5 \unit{kV}$ and the data from the back-scattering electron detector is shown.

  \paragraphNewLine{X-Ray Reflectometry}
    Using a Bruker D8 Advanced at the \textsc{Forschungszentrum J\"ulich} \refsec{ch:instruments:laboratoryInstruments:xrr}, XRR of all double layers is measured.
    The samples are measured with a Cu-K$\alpha$ source ($\lambda \eq 1.54 \unit{\angstrom}$) and a $q$-range of $0 \ldots 0.15 \unit{\angstrom^{-1}}$ is evaluated by measuring $2 \theta \eq 0 \ldots 1 \unit{^\circ}$ in $0.005 \unit{^\circ}$ steps over an integrated time of approximately $1 \unit{h}$.
    To perform the footprint correction (\refsec{ch:methods:xrr}) of the sample, the collimation slit of the instrument with $0.2 \unit{mm}$ and samples width of $10 \unit{mm}$ is considered.
    An equidistributed beam intensity profile is assumed for the correction.

  \paragraphNewLine{Polarized Neutron Reflectometry}
    The samples DL-0.125\%, DL-0.25\%, DL-1.25\% and DL-2.5\% are measured by polarized neutron reflectometry on the D17 (\refsec{ch:lss:d17}) at the Institute Laue-Langevin.
    The instrument is operated in the time-of-flight mode and the samples are measured at three incident angles of $0.50^\circ ,\, 1.80^\circ ,\, 4.00^\circ$ each to discuss a $q$-range of $0 \ldots 0.2 \unit{\angstrom^{-1}}$.
    The data is reduced with the COSMOS software, which is maintained by the instrumental scientists \cite{Gutfreund_2018_Towar}.
    Each sample is measured after zero-field cooling at $\mathrm{T} \eq 5 \unit{K}$ with polarized neutrons at a guide field of $B \eq 10 \unit{mT}$, then at saturation of $6 \unit{T}$ and finally at a negative field of $-100 \unit{mT}$.

    For data evaluation the instrumental resolution is used as provided by the COSMOS reduction software.

  \paragraphNewLine{Vibrating Sample Magnetometry}
    The macroscopic magnetization of the double layers with varied spacer thickness are each measured field- and temperature-dependent using a PPMS Evercool II (\refsec{ch:instruments:laboratoryInstruments:vsm}).
    Each sample is cut with a diamond cutter to a size of approximately $5 \times 5 \unit{mm^2}$ and stuck on a quartz sample holder by a low temperature varnish (GE 7031).

    The samples are measured field-dependent in a range of $\pm 9 \unit{T}$ at $300 \unit{K}$ and $10 \unit{K}$ with a sweeping rate of $5 \unit{mT \, s^{-1}}$.
    The measurements at $10 \unit{K}$ are performed after cooling in a field of $10 \unit{mT}$.
    The sample magnetization is furthermore measured temperature-dependent after zero-field cooling and field cooling at $10 \unit{mT}$ from $10 \unit{K}$ to $350 \unit{K}$ while warming the sample with a rate of $1.5 \unit{K \, s^{-1}}$.

    The samples vary slightly in size, the mass of the silicon wafers is measured by a precision scale to estimate the measured area. As the thickness of the silicon does not vary, but is by a micrometer scale measurably given by $0.52 \unit{mm}$, and as the nanoparticle and PMMA layer only contributes a mass in the order of $\musf g$, the sample mass is directly proportional to the surface area.
    The determined masses of the samples are tabulated in \reftab{tab:doubleLayers:layerCharacterization:ppmsMasses}.
    The values are used to estimate the validness of the determined susceptibility and spontaneous magnetization obtained by fitting the room temperature magnetization in the range of $5 - 9 \unit{T}$.

    \begin{table}[!htbp]
      \centering
      \caption{\label{tab:doubleLayers:layerCharacterization:ppmsMasses}Mass of the samples used for the VSM measurements.}
      \begin{tabular}{ l | l}
        \rule{0pt}{2ex} \textbf{Sample}  & $m \, / \unit{mg}$ \\
        \hline
        \rule{0pt}{2ex} DL-0.125\%   & $29.55(2)$ \\
        \rule{0pt}{2ex} DL-0.25\%    & $27.44(2)$ \\
        \rule{0pt}{2ex} DL-1.25\%    & $26.73(2)$ \\
        \rule{0pt}{2ex} DL-2.5\%     & $29.51(2)$ \\
        \rule{0pt}{2ex} DL-5\%       & $30.27(2)$ \\
        \hline
      \end{tabular}
    \end{table}

\end{document}