\providecommand{\main}{../../..}
\documentclass[\main/dresen_thesis.tex]{subfiles}

\begin{document}
  In this chapter the first systematic study on the preparation and properties of magnetic nanocube double layers is presented.
  Using the knowledge acquired from the monolayers chapter, single layers of nanocubes in square arrays are prepared on a silicon substrate, coated with a PMMA layer of tuneable thickness and finished with a second layer of nanocubes.

  The successful and reproducible preparation by this method is shown by SEM, XRR and NR, where the total sample thickness is directly correlated to the amount of PMMA spin coated between the double layer sandwich.
  The presence of the two nanocube layers is visible in the SEM cross-sectional micrographs, as well as by the correlation peak at $1 \unit{nm^{-1}}$ visible in both XRR and NR.

  Using the successfully prepared double layers, the magnetic structure is studied by both VSM and PNR at low temperatures.
  From dipolar interlayer coupling, an antiferromagnetic coupling between the two layers might be expected, which however shows no strong signature that is directly correlated to the interlayer distance in either of the experimental characterizations.
  The temperature-dependent magnetization measurements of the samples closely resemble one another.
  For the field-dependent magnetization measurements a difference in a jump in magnetization around zero applied magnetic field is visible, which is however has no direct systematic with the interlayer distance.

  The polarized neutron reflectometry measurements are qualitatively discussed and show no strong signature of an anti-parallel but a parallel alignment of the magnetization in the two layers.
  \\

  It is concluded that the nanocube layers primarily act independently from one another in the double layer samples and interlayer dipolar coupling has no strong effect on the sample's magnetic properties.
\end{document}