\providecommand{\main}{../../..}
\documentclass[\main/dresen_thesis.tex]{subfiles}

\begin{document}
  Iron oxide nanocubes have been prepared following a literature known synthesis route from iron oleate.
  The cubical shape is confirmed by transmission electron microscopy and the typical core-shell structure of w\"ustite/magnetite for nanoparticles synthesized according to the oleate route is confirmed by XRD.

  In the XRD experiment, a relatively large w\"ustite fraction is estimated, whereas the crystallite length of the magnetite phase is comparably small.
  The dispersion is actively oxidized by heating it for half an hour in cyclooctane at $150 ^\circ C$ for one hour, which becomes visible in a brown coloring of the dispersion and a significantly enhanced measured magnetization in VSM but results in an unstable dispersion.
  As the nanocubes are supposed to be used for self-assembly experiments, the work is continued with the fraction that is not actively oxidized but stable in dispersion.
  This has the downside that the oxidation state of the nanocubes is not well defined across experiments.

  Using small-angle scattering the particle size is determined to be ... and the size distribution to ... , which is in accordance with the results from TEM of to $13.42 \unit{nm}$, with a size distribution of $7.3(5) \%$.

  VSM of the nanocubes shows a relatively weak spontaneous magnetization of $ \unit{kA \, m^{-1}}$ at room temperature.
  In SANSPOL ...
  \\

  The nanocubes in dispersion is used to self-assemble three-dimensional long-range ordered layers of nanocubes on silicon, which are characterized in the following section.

\end{document}