\providecommand{\main}{../../../..}
\documentclass[\main/dresen_thesis.tex]{subfiles}
\begin{document}
  \label{sec:doublelayers:structuralCharacterization:summary}

  Double layers have been prepared from Ac-CoFe-C nanoparticles with a spin-coated intermediate PMMA layer of varied thickness.
  From the cross-sectional SEM micrographs it is visible that the prepared PMMA thicknesses range from near zero to approximately $300 \unit{nm}$ and is linearly tuneable by the PMMA volume concentration used during the spin coating process.
  The varied thickness reflects optically by the color of the samples.
  \\

  Furthermore, the four samples DL-0.125\%, DL-0.25\%, DL-1.25\% and DL-2.5\% are studied by X-ray and neutron reflectometry, and one thicker sample DL-5\% by XRR only.
  From the visible Kiessig fringes, the sample thickness is determined for two samples in XRR (DL-1.25\% and DL-2.5\%) and three samples in NR (DL-0.25\%, DL-1.25\% and DL-2.5\%).
  The thickness determined from the Kiessig fringes is for both X-rays and neutrons observed to be larger as determined from SEM micrographs, which is explained by the invasive nature of the electron beam on the PMMA layer, which results in an underestimation of the sample thickness.
  The thickness determined by XRR at room temperature is $20 - 30\%$ larger than the thickness from neutron reflectometry at $5 \unit{K}$ in direct comparison, which is a larger reduction than should be expected from thermal contraction alone.
  The difference is therefore explained by the different scattering length density contrasts with which the sample is perceived by X-rays and neutrons.
  As for example neutrons are insensitive to an organic layer on top of the sample due to the poor contrast to air.
  An exact determination of the of the scattering length density profile for either X-rays or neutrons was not possible within the scope of this thesis and will be part for further future evaluation of the data.
  \\

  As main takeaway of the structure determination, order of magnitudes for the layer separations are obtained, which can be used to discuss in the following the magnetic properties of the double layer samples.


\end{document}