\providecommand{\main}{../../../..}
\documentclass[\main/dresen_thesis.tex]{subfiles}
\begin{document}
  \label{sec:doublelayers:structuralCharacterization:summary}

  Double layers have been prepared from Ac-CoFe-C nanoparticles with a spin-coated intermediate PMMA layer of varied thickness.
  From the cross-sectional SEM micrographs it is visible that the prepared PMMA thicknesses range from near zero to approximately $300 \unit{nm}$ and is linearly tuneable by the PMMA volume concentration used during the spin coating process.
  The varied thickness reflects optically by the color of the samples.
  \\

  Furthermore, the four samples DL-0.125\%, DL-0.25\%, DL-1.25\% and DL-2.5\% are studied by X-ray and neutron reflectometry, and one thicker sample DL-5\% by XRR only.
  From the visible Kiessig fringes, the sample thickness is determined for two samples in XRR (DL-1.25\% and DL-2.5\%) and three samples in NR (DL-0.25\%, DL-1.25\% and DL-2.5\%).
  The thickness determined from the Kiessig fringes is for both X-rays and neutrons observed to be larger than determined from SEM micrographs, which is explained by the invasive nature of the electron beam on the PMMA layer, which results in an underestimation of the sample thickness.
  Furthermore, a second length scale is visible in both XRR and NR that corresponds to the thickness of the nanocube layers including the oleic acid surfactant shell and is determined to $16(1) \unit{nm}$ and $13(2) \unit{nm}$, respectively.
  By subtracting the nanocube thickness from the sample thickness, the interlayer distance of each sample is estimated.
  The PMMA thickness determined by XRR at room temperature is $17(1) - 23(2)\%$ larger than the thickness from neutron reflectometry at $5 \unit{K}$ in direct comparison, which is a larger reduction than should be expected from thermal contraction that is estimated to be less than $3 \%$ from the literature properties of PMMA.
  The difference is therefore explained by the different scattering length density contrasts with which the sample is perceived by X-rays and neutrons.
  As for example neutrons are insensitive to an organic layer on top of the sample due to the poor contrast to air.
  An exact determination of the scattering length density profile for either X-rays or neutrons will be part for further future evaluation of the data.
  The order of magnitudes for the interlayer distance are useful in the following to discuss the magnetic properties of the double layer samples and the expected interlayer coupling strengh.
\end{document}