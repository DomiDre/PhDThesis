\providecommand{\main}{../../../..}
\documentclass[\main/dresen_thesis.tex]{subfiles}
\begin{document}
  \label{sec:doublelayers:structuralCharacterization:summary}

  Double layers have been prepared from Ac-CoFe-C nanoparticles with a spin-coated intermediate PMMA layer of varied thickness.
  The prepared PMMA thicknesses range from near zero to approximately $300 \unit{nm}$ and is linearly tuneable by the PMMA volume concentration used during the spin coating process as is shown by the evaluation of cross-sectional SEM micrographs.
  The varied thickness reflects optically by the color of the samples and the reproducibility is shown for one example by X-ray reflectometry.
  \\

  Four samples, DL-0.125\%, DL-0.25\%, DL-1.25\% and DL-2.5\%, with varied PMMA thickness are studied by X-ray and neutron reflectometry, and one thicker sample DL-5\% by XRR only.
  From the visible Kiessig fringes, the sample thickness is determined for two samples in XRR (DL-1.25\% and DL-2.5\%) and three samples in NR (DL-0.25\%, DL-1.25\% and DL-2.5\%).
  The thickness determined from the Kiessig fringes is for both X-rays and neutrons observed to be larger as determined from scanning electron microscopy, which is explained by the invasive nature of the electron beam on the PMMA layer.
  The thickness determined by XRR at room temperature is $20 - 30\%$ larger than the thickness from neutron reflectometry at $5 \unit{K}$ in direct comparison, which is a larger reduction than should be expected from thermal contraction alone.
  \\

  Scattering length density profiles of the double layers have not been determined yet.

\end{document}