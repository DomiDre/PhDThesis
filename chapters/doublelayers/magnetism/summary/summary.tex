\providecommand{\main}{../../../..}
\documentclass[\main/dresen_thesis.tex]{subfiles}
\begin{document}
  \label{sec:doublelayers:magnetism:summary}
  The magnetism of double layered structures has been tackled by two approaches.
  First, the macroscopic magnetization of five samples with varied interlayer distance are studied by vibrating sample magnetometry.
  Correcting for slight differences in the sample observed at room temperature, which originate from the variation in measured sample area and small variations inherited from the drop-casting procedure, low-temperature field-dependent and temperature-dependent magnetization measurements are performed and directly compared to one another.
  From the temperature-dependent magnetization, the magnetic behaviour that is also observed for monolayers is closely reproduced in each case and no significant effects of interlayer coupling are visible.
  The low-temperature hysteresis measurements show no modification of the coercivity but a small variation in the magnitude of the magnetization jump around $\pm 100 \unit{mT}$.

  With polarized neutron reflectometry the magnetic structure of the samples is approached with depth-resolution to gain a better view of the relative magnetization direction in the two respective layers.
  By simulating double layer structures on basis of the monolayer SLD profile, the expectation for the magnetic splitting from a parallel and anti-parallel direction are discussed.
  The most striking difference is observed in whether the peaks in the Kiessig fringes are seperated in the case of a parallel alignment or touching in the case of an anti-parallel alignment.
  This is transferred to qualitatively discuss the polarized neutron reflectivities of four double layer samples at saturation and at a negative field of $-100 \unit{mT}$ below the jump in the hysteresis.
  In each case a clear separation of the Kiessig fringes is observed and therefore a majorly parallel alignment of the two layers is concluded.

  Thus, by those two studies no strong signatures for an antiferromagnetic dipolar interlayer coupling could be observed.
  Instead from theoretical calculations and the experimental observations, the considered interlayer distances are large enough to consider the magnetic properties of the double layers closely to that of independent monolayers.
\end{document}