\providecommand{\main}{../../..}
\documentclass[\main/dresen_thesis.tex]{subfiles}
  \renewcommand{\thisPath}{\main/chapters/theoreticalBackground/ferrites}
\begin{document}

  \section{Iron Spinels}
  In this work, nanoparticles are prepared chemically from iron and cobalt salts for the study of ordered magnetic structures on the nanoscale.
  Iron is the fourth most common element in the Earth's crust and ferromagnetic at room temperature in its elemental state.
  Iron oxide forms naturally through the weathering of Fe-containing rocks on land and in the oceans as they interact with the most common element in the Earth's crust, oxygen \cite{Parkinson_2016_Irono}.
  In the iron oxides, oxygen anions \ch{O^{2-}} form a close packed lattice with octahedral and tetrahedral interstices.
  The oxidation state and abundance of the iron cations in the interstices determines the type of material and it's structural and magnetic properties.

  \subsection{Magnetite, Maghemite and W\"ustite}\label{ch:theoreticalBackground:maghemite}
    \subfile{\thisPath/maghemite/maghemite}

  \subsection{Cobalt Ferrite}\label{ch:theoreticalBackground:cobaltferrite}
    \subfile{\thisPath/cobaltferrite/cobaltferrite}
\end{document}
