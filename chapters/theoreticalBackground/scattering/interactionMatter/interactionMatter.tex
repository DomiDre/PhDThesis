\providecommand{\main}{../../../..}
\documentclass[\main/dresen_thesis.tex]{subfiles}

\renewcommand{\thisPath}{\main/chapters/theoreticalBackground/scattering/interactionMatter}
\begin{document}
  \subsection{Interaction of X-rays and Neutrons with Matter}\label{sec:theoreticalBackground:scattering:interactionWithMatter}
    To calculate the differential cross section, it is necessary to know how the scattering wave and the sample interact with each other, which is represented by the potential $V$.
    Even though in the previous chapter, the differential cross section is derived for quantum mechanical particles, it is shown in \refapp{ch:appendix:calculations:scatteringTheoryElectromagneticWaves} that a similar formula results for X-rays from classical electrodynamics.
    In the case of X-rays, the dominant coupling to consider is the electromagnetic interaction of the X-ray photons with the electron shells of the atoms making up the material.
    When an electron cloud is considered to oscillate in phase with the incoming X-ray, the differential cross section is modeled as
    \begin{align}
      \frac{\dint \sigma}{\dint \Omega} \eq |\hat{e}_i \cdot \hat{e}_f|^2 \biggl| \int \dint V e^{-i \vec{q} \cdot \vec{r}}  r_e \rho_e (\vec{r}) \biggr|^2.
    \end{align}
    Here $|\hat{e}_i \cdot \hat{e}_f|^2$ is a polarization factor depending on the geometry and the source of the experiment, $r_e \approx 2.8 \unit{fm}$ is the classical electron radius and $\rho_e$ the density distribution of the electron cloud.
    This result is known as Thomson scattering.
    In general the electronic structure and response of materials is very complex and highly dependent on the energy of the incoming photon.
    In the derivation of the formula above it is assumed that the electron cloud follows the incoming wave via Newton's law $\vec{F} \eq m \vec{a} \eq -e\vec{E}$.
    This neglects resonance, absorption and dispersion effects.
    Those become important whenever the energy of the X-rays is close to electronic transitions in the material or is high enough that it  ejects electrons from their respective positions.
    However in the frame of this work, such effects are negligible and further discussions about it can be found in literature \cite{AlsNielsen_2011_Eleme}.

    For neutrons on the other hand, two fundamental interactions with a material exist.
    On the one hand the neutron scatters via the short-ranged residual strong interaction with the atomic nuclei and on the other the magnetic moment of the neutron contributes to a scattering via the electromagnetic interaction with the internal magnetic field of the sample.
    For the nuclear scattering, the length scale of the residual strong force is on the order of $\mathcal{O} (10^{-15} \unit{m})$, whereas the wavelength of thermal neutrons $\mathcal{O} (10^{-10} \unit{m})$ is much larger.
    The nucleus can therefore be considered point-like and the potential for the scattering of a free neutron from a collection of nuclei positioned at $\vec{r}_j$ can be modeled by a sum of Fermi pseudopotentials
    \begin{align}
      V(r) \eq \frac{2 \pi \hbar^2}{m} \sum_j b_j \delta(\vec{r} - \vec{r}_j),\label{eq:theoreticalBackground:scattering:scatteringTheory:neutronScatteringPotentialSum}
    \end{align}
    where $b$ is the bound scattering length for said nucleus, which is different for every element and isotope and therefore includes the  information to differentiate between them.
    The \textit{ab initio} calculation of the scattering length for an isotope is a hard task, due to the complicated nature of the strong force, which is described by the theory of quantum chromodynamics, and needs detailed knowledge of the subatomic structure.
    However, the experimental values of the bound scattering length have been tabulated for most isotopes and can be combined to model an investigated material of known composition \cite{Sears_1992_Neutr}.
    The bound scattering length is in general a complex number $b \eq b^\prime - i b^{\prime \prime}$, where the complex part describes neutron absorption due to nuclear reactions.
    Furthermore the bound scattering length comprises a coherent $b_c$ and incoherent $b_i$ cross section, where the incoherent cross section depends on the relative orientation of the neutron spin $\vec{s}$ to the nucleus angular momentum $\vec{I}$
    \begin{align}
      b \eq b_c + \frac{2 b_i}{\sqrt{I(I+1)}} \vec{s} \cdot \vec{I}.
    \end{align}
    In the framework of the experiments and materials discussed in this thesis, neutron absorption by the material can be neglected.
    Also, the orientation of the nuclei angular momentum relative to the neutron spin can be considered random and therefore the incoherent scattering contains no information about the structure of the samples but is a constant background.
    When the exact atomic structure is not of interest, the continuum limit can be taken for the sum in \refeq{eq:theoreticalBackground:scattering:scatteringTheory:neutronScatteringPotentialSum} and it can be replaced by a scattering length density
    \begin{align}
      \sum_j b_j \delta(\vec{r} - \vec{r}_j) \rightarrow \rho(\vec{r}),
    \end{align}
    where the value of $\rho(\vec{r})$ is determined by summing over the bound scattering lengths of all atoms in a unit cell of volume $V_{uc}$ for a given material at position $\vec{r}$
    \begin{align}
      \rho(\vec{r}) \eq \frac{1}{V_{uc}} \sum_i^n b_i.
    \end{align}
    The differential cross section is thus obtained for neutrons from evaluating
    \begin{align}
      \frac{\dint \sigma}{\dint \Omega} \eq \bigg| \int \dint \vec{r} e^{-i\vec{q} \cdot \vec{r}} \rho (\vec{r}) \bigg|^2.
      \label{eq:theoreticalBackground:scattering:scatteringTheory:differentialCrossSectionBornApproximation}
    \end{align}
    It is interesting to note here that for X-rays, up to the polarization factor, the same formula applies when the X-ray scattering length density is identified from the electron density as $\rho \eq r_e \rho_e$.
    Therefore, every following discussion starting from this equation includes both the scattering of X-rays from the electron clouds of matter and the scattering of neutrons from the nuclear structure.

    The magnetic scattering of neutrons from the magnetic structure of the material can be included by adding a Zeeman potential to the Hamiltonian
    \begin{align}
      V_m (\vec{r}) \eq - \vec{\mu}_n \cdot \vec{B},
    \end{align}
    where $\vec{\mu}_n \eq \mu_n \hat{\sigma}$ is the magnetic moment of the neutron, which has a magnitude of $\mu_n \eq 9.662 \cdot 10^{-27} \unit{JT^{-1}}$, and is quantum mechanically given by the spin operator $\hat{\sigma}$, and $\vec{B}$ is the magnetic field generated by the sample due to the bound electron orbital motion and spins.
    In \refapp{ch:appendix:calculations:magneticScatteringTheory} it is derived that the contribution by the spin to the differential cross section can be modeled similar to the nuclear scattering as Fourier transform over a magnetic scattering length density
    \begin{align}
      f_M (\vec{q}) \eq & \hat{\mu}_n \cdot \hat{s}_\perp \int \dint \vec{r} e^{- i \vec{q} \cdot \vec{r}} \rho_\mathrm{mag}(\vec{r}),
    \end{align}
    where the magnetic scattering length density is defined by the spin density as
    \begin{align}
      \rho_\mathrm{mag} (\vec{r}) \eq \frac{m_n}{\hbar^2}  \frac{\mu_0}{2\pi} \mu_B \mu_n s_e (\vec{r}),
      \label{eq:theoreticalBackground:scattering:magneticSLD}
    \end{align}
    and an additional prefactor $\hat{\mu}_n \cdot \hat{s}_\perp$, where $\hat{s}_\perp$ is the fraction of the spins that are directed in the plane perpendicular to the scattering vector $\vec{q}$.
    Thus, polarized neutron scattering allows to measure the spin density of a sample in that plane.

    The polarization of the incoming neutron beam in an experiment is defined by a polarizer and stabilized by a weak magnetic guide field.
    When $\hat{s}_\perp$ is parallel (antiparallel) to this direction, $f_M$ contributes with a positive (negative) sign to the differential cross section and the direction of the spin is conserved during the scattering.
    On the other hand, when $\hat{s}_\perp$ is perpendicular to this direction, the spin operator results in a flip of the neutron spin during the scattering.
    This allows, through careful analysis of the non-spin flip and spin flip channels, to measure additionally the direction of the magnetization in the plane perpendicular to the scattering vector.
\end{document}