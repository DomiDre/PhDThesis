\providecommand{\main}{../../..}
\documentclass[\main/dresen_thesis.tex]{subfiles}
\renewcommand{\thisPath}{\main/chapters/theoreticalBackground/scattering}
\begin{document}
  \section{Scattering}\label{sec:theoreticalBackground:scattering}
    Scattering describes the general physical process, where radiation changes its straight path due to interaction with another object.
    In a broad perspective this includes all various types of radiation - light, X-ray, electron, neutron, \etc .
    For example it includes the very daily process of seeing, where after a light wave is emitted from a source like the sun or a light bulb, the wave is scattered from an object in the room before it is finally detected within ones eye.
    And scattering also includes the process where after a neutron is generated in a nuclear reactor, it interacts with a sample in a experimental hall before it is measured with a sophisticated detector.

    In this work, multiple X-ray and neutron scattering techniques are applied to study the nuclear and magnetic structure of nanoparticles and their manufactured assemblies.
    To understand the rich information that is obtained by those techniques, \refsec{sec:theoreticalBackground:scattering:scatteringTheory} presents in the following a brief introduction to the general scattering theory and \refsec{sec:theoreticalBackground:scattering:interactionWithMatter} to the interaction of X-ray and neutrons with matter.
    The length scales that coherently scatter due to finite collimation and wavelength spreads are discussed in \refsec{sec:theoreticalBackground:scattering:CoherenceInstrumentalResolution}.
    Then the theory behind the mainly applied techniques are discussed: small-angle scattering in \refsec{sec:theoreticalBackground:scattering:SASNanoparticles}, reflectometry in \refsec{sec:theoreticalBackground:scattering:reflectometry} and grazing-incidence small-angle scattering in \refsec{sec:theoreticalBackground:scattering:GISAS}.

    \subfile{\thisPath/introduction/introduction}
    \subfile{\thisPath/interactionMatter/interactionMatter}
    \subfile{\thisPath/coherence/coherence}
    \subfile{\thisPath/sas/sas}
    \subfile{\thisPath/reflectometry/reflectometry}
    \subfile{\thisPath/gisas/gisas}
\end{document}