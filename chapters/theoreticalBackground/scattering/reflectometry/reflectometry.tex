\providecommand{\main}{../../../..}
\documentclass[\main/dresen_thesis.tex]{subfiles}
\renewcommand{\thisPath}{\main/chapters/theoreticalBackground/scattering/reflectometry}
\begin{document}
  \subsection{Reflectometry}\label{sec:theoreticalBackground:scattering:reflectometry}
    Nanoparticles that are deposited on a substrate self-assemble to higher order structures under certain conditions that are elaborated in later chapters of this work.
    It is of high interest to study the physical properties of the ensemble and whether they differ from the single particle properties and how this depends on the superstructure.
    A technique to study the vertical structure of a thin sample on a substrate is reflectometry.
    X-ray reflectometry allows to study the average electron density in a sample with depth resolution, whereas neutron reflectometry allows to study the nuclear structure.
    Additionally, polarized neutron reflectometry allows to resolve the magnetic density with depth resolution.
    The general technique is described in the following in the frame of the study of nanoparticles on a substrate, as it is applied in later parts of this thesis.
\end{document}