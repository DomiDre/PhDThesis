\providecommand{\main}{../../../..}
\documentclass[\main/dresen_thesis.tex]{subfiles}
\renewcommand{\thisPath}{\main/chapters/theoreticalBackground/scattering/gisas}
\begin{document}
  \subsection{Grazing Incidence Small-Angle Scattering from a Surface}\label{sec:theoreticalBackground:scattering:GISAS}
    Additionally to the vertical structure, it is interesting to study the lateral structure of an ensemble of nanoparticles to obtain the complete three dimensional information.
    For this purpose grazing-incidence small-angle scattering (GISAS) has proven an efficient technique to measure the off-specular scattering from which in-plane order within a sample can be extracted with high resolution \cite{Renaud_2009_Probi}.
    As in the case of reflectometry, it is possible to obtain information about the lateral electron density using GISAS with x-rays (GISAXS), and about the lateral nuclear and magnetic structure using (polarized) neutrons (GISANS, polGISANS).
    The experimental setup ...

    To model the off-specular scattering measured in grazing incidence small angle scattering, the problem can be discussed in distorted wave born approximation (DWBA). Here, the potential $V(\vec{r})$ of a thin layer is decomposed into a sum of the laterally averaged potential $V_1(\vec{r})$ and the in-plane fluctuations $\delta V(\vec{r})$, which are treated as small perturbation $|\delta V| \ll |V_1|$
    \begin{align}
      V(\vec{r}) \eq V_1(\vec{r}) + \delta V(\vec{r}).
    \end{align}

\end{document}