\providecommand{\main}{../../../..}
\documentclass[\main/dresen_thesis.tex]{subfiles}
\renewcommand{\thisPath}{\main/chapters/theoreticalBackground/scattering/coherence}
\begin{document}
  \subsection{Coherence}\label{sec:theoreticalBackground:scattering:CoherenceInstrumentalResolution}

    In the derived theory, x-ray and neutrons are infinitely extending plane waves of a single wavelength $\lambda$ and the complete sample is radiating spherical waves in phase with the incoming wave with the scattering angle $\theta$.
    In reality, however, the beam has a finite width defined by collimation, which leads to a divergence of the beam of $\Delta \theta$ and the beam is not perfectly monochromatic, but has a finite distribution of wavelengths $\Delta \lambda$.
    This leads to a superposition of multiple patterns in a real experiment and possibly to a loss of information due to the finite instrumental resolution that destroys the interference pattern.
    To quantify what area of the sample is radiating in phase, such that the interference is not destroyed, the coherence of the beam needs to be evaluated.

    The wavelength spread $\Delta \lambda$ defines the longitudinal (or temporal) coherence of the beam. It can be quantified by considering on which length scale $L_L \eq n \lambda$ two waves with wavelengths $\lambda$ and $\lambda + \Delta \lambda$ start to interfere destructively
    \begin{align}
      n\lambda \eq& \biggl(n - \frac{1}{2} \biggr)(\lambda + \Delta \lambda),
    \end{align}
    which can be rearranged to
    \begin{align}
      L_L \eq& \frac{\lambda}{2} \biggl(1 + \frac{\lambda}{\Delta \lambda} \biggr) \approx \frac{\lambda^2}{2 \Delta \lambda},
    \end{align}
    where $\Delta \lambda / \lambda$ is typically in the order of $5\, \%$ for neutron experiments, which means the longitudinal coherence length is for neutrons with wavelength $\lambda \eq 5 \angstrom$ in the order of $L_L \eq 50 \angstrom$.

    The finite beam size results in a transversal coherence, which destroys the interference pattern when the angular separation $\Delta \theta$ leads to a destructive interference for the wavelength $\lambda$ on the length scale $L_T$
    \begin{align}
      L_T \eq \frac{\lambda}{2 \Delta\theta}.
    \end{align}
    The transversal coherence length might be different in the horizontal and vertical plane and depends strongly on the experimental geometry.
    It depends on the chosen slits and distances between the slits, sample and detector.
    For a typical beam divergence in the order of $\delta \theta \eq 0.3^\circ$, a transversal coherence length of $L_T \eq 500 \angstrom$ is obtained for $\lambda \eq 5 \angstrom$.

    The three coherence length scales define a coherence volume, for which a sample radiates in phase and produces interference pattern.
    This has to be kept in mind when designing models for the evaluation of scattering experiments, as over larger length scales the contributions add incoherently, and instead of the amplitudes, the intensities need to be added.

    % \& Instrumental Resolution
    % To include furthermore the smearing of the intensities due to the wavelength spread and beam divergence, an effective spread in the wave vector is obtained by error propagation
    % \begin{align}
    %   \frac{\Delta q}{q} \eq \sqrt{\biggl( \frac{\Delta \lambda}{\lambda} \biggr)^2 + \biggl( \cot (\theta) \Delta \theta \biggr)^2}.
    % \end{align}
    % The effective smearing is modelled by a Gaussian function with width $\Delta q$ that is convoluted with the calculated model of the differential cross section for a given $q$ range.
\end{document}