\providecommand{\main}{../../..}
\documentclass[\main/dresen_thesis.tex]{subfiles}
\renewcommand{\thisPath}{\main/chapters/theoreticalBackground/magnetism}
\begin{document}
\section{Magnetism}\label{ch:theoreticalBackground:magnetism}

  Classically, magnetic fields are understood to be produced by moving electric charges \cite{Jackson_1999_Class, Blundell_2001_Magne}.
  Within the framework of special relativity, magnetic fields emerge when the frame of reference is transformed from the rest frame of an electric charge to a relatively moving one via a Lorentz transformation.
  From this point on, the propagation and interaction of the magnetic field with charge and current distributions is fully described by Maxwell's equations.
  Here, it is customary and helpful for a localized current distribution $\vec{j}(\vec{r})$ to be associated with its magnetic moment defined by
  \begin{align}
    \vec{\mu} \eq \frac{1}{2} \int \vec{r} \times \vec{j}(\vec{r}) \dint V .
  \end{align}
  In the case of a closed loop of current $I$, the magnetic moment is just given by the product of the current and the enclosed area $A$, $\mu \eq I A$, with the direction pointing parallel to the surface normal.

  The magnetism of matter that is neutrally charged on the macroscopic scale, is the net sum produced mainly by the motion of electrons in the atoms that make up the matter (the contribution of the nucleus is approximately three orders of magnitude weaker).
  For this purpose, a magnetic moment can be associated with each electrons spin and orbital motion.
  Due to the quantum mechanical nature of the electron, the magnetic moments has also to be treated quantum mechanically\footnote{In fact, the Bohr-van-Leeuwen theorem shows that for a purely classical system in thermal equilibrium, the magnetization is always zero.}.
  The various types of magnetism that are observed within different materials in nature can then be understood by the structure and the coupling of the electrons within the matter.
  The most popular types that are used throughout this work are described in the following.

  \subsubsection{Diamagnetism}
    Diamagnetism is a weak effect that is exhibited by every material and measurable when it is not overshadowed by another type of magnetism.
    When a material enters a magnetic field, Faraday's law of induction states that currents are induced which by Lenz's law are counter directed to the external field.
    For materials with $n$ atoms per volume that have closed shells with $Z$ electrons, the diamagnetic magnetization can be described quantum mechanically in first order perturbation theory by the Langevin diamagnetism \cite{Blundell_2001_Magne}
    \begin{align}
      M \eq \underbrace{- \frac{Z n e^2}{6 m_e} \braket{r^2}}_{\eq \chi} B,
    \end{align}
    where $\braket{r^2}$ is the mean square distance of the electrons from the nucleus.
    The diamagnetic susceptibility $\chi$ is always negative and for common materials $\mu_0 chi$ is in the order of $10^{-5}$.
    In magnetometry experiments, it is often observed from background materials such as a silicon substrate that carries the actual sample of interest or a solvent.
    As the Langevin diamagnetism is a linear function in $B$ that is largely temperature independent, it can in general be easily separated from other magnetic contributions if it is not negligible in strength.

  \subsubsection{Paramagnetism}
    Paramagnetism is observed in materials that have atoms with unpaired electrons.
    At zero field, the magnetic moment of the unpaired electron have random orientation if no further coupling to other magnetic moments in the material is given.
    The magnetic moment try to align with external magnetic field and thus paramagnetic materials exhibit a positive susceptibility.

  \subsubsection{Ferromagnetism}
  \subsubsection{Antiferromagnetism}
  \subsubsection{Ferrimagnetism}
  \subsubsection{Superparamagnetism}

  \subsection{Cobalt Ferrite and Maghemite}

  \subsection{Magnetic Anisotropy}

  \subsection{Dipolar Interaction}
  For a point-like magnetic moment the magnetic moment produced at a distance $\vec{r}$ is described by a dipolar field
  \begin{align}
    \vec{B}(\vec{r}) \eq \frac{\mu_0}{4 \pi} \frac{3 \vec{r} (\vec{\mu} \cdot \vec{r}) - \vec{\mu} r^2}{r^5}.
  \end{align}

  The interaction of a magnetic moment with an external magnetic field  $\vec{B}$ is then described by the Zeeman potential
  \begin{align}
    E \eq - \vec{\mu} \cdot \vec{B},
  \end{align}
  from which the forces and the torque by the field on the moment can then be determined by the gradient.
  \FloatBarrier
\end{document}